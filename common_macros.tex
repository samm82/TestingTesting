\newif\ifnotpaper

%------------------------------------------------------------------------------
% Reused in seminar slides
%------------------------------------------------------------------------------

\def\rqatext{What testing approaches do the literature describe?}
\def\rqbtext{Are these descriptions consistent?}
\def\rqctext{Can we systematically resolve any of these inconsistencies?}

\def\addTextEx{For example, \citetISTQB{} \multiAuthHelper{cite}
    \citet{GerrardAndThompson2002} as the original source
    for \ifnotpaper their \else its \fi definition of ``scalability'' (see
    \Cref{scal-test-rec}); we verified\ptq{} this by
    looking at this original source.}

\def\supers{Dr.~Spencer Smith and Dr.~Jacques Carette}
\def\supersAck{\supers{} have been great supervisors and valuable sources of
    guidance and feedback}

%------------------------------------------------------------------------------
% Spacing Options
%------------------------------------------------------------------------------

\newcommand{\thesisForceSingleSpacing}{\singlespacing}
\newcommand{\thesisForceDoubleSpacing}{\doublespacing}

%------------------------------------------------------------------------------
% Portable HREFs
%------------------------------------------------------------------------------

% Common variant
\newcommand{\porthref}[2]{\href{#2}{#1}\printOnlyFootnote{\url{#2}}}

% Custom URLs
\newcommand{\porthreft}[3]{\href{#3}{#1}\printOnlyFootnote{\href{#3}{#2}}}
% Inside of some environments, footnote marks aren't registered properly, so we
% need to manually write the "text" part
\newcommand{\porthreftm}[2]{\href{#2}{#1\printOnlyFootnoteMark}}

\newcommand{\formatPaper}[2]{%
    \ifnotpaper #1{#2}%
    \else \underline{#2}%
    \fi
}

\def\refHelper{\ifnotpaper\else Reference \fi}
\newcommand\multiAuthHelper[1]{\ifnotpaper #1\else #1s\fi}
\def\docType{\ifnotpaper thesis%
    \else paper%
    \fi}

\newcommand\flawref[1]{%
    \ifnotpaper \labelcref{#1}%
    \else \Cref{#1}%
    \fi}

\newcommand\ifblind[2]{\IfEndWith*{\jobname}{_blind}{#1}{#2}}

%------------------------------------------------------------------------------
% Generic "chunks" that get reused
%------------------------------------------------------------------------------

\def\oneSrcDistinct{we make a distinction between ``self-contained'' flaws and
    ``internal'' flaws\thesisissueref{137,138}.}

\def\highLvlScope{a high-level overview of what is in scope\ifnotpaper; see
    \Cref{app-scope} for more detailed discussion on what we include and
    exclude\fi}

\def\listAllSrcs{\ifnotpaper\ and list all sources in each tier
        in \Cref{app-src-tiers}\fi}

\newcommand\defRel[3]{We can formally define the #1 relation $#2$ on the set
    $T$ of terms used by the literature to describe test approaches based on #3.}

\NewDocumentCommand{\approachFields}{s}{
    definitions, categories\IfBooleanTF{#1}{}{\ (see \Cref{cats-def})},
    synonyms\IfBooleanTF{#1}{}{\ (see \Cref{syn-rels})}, and
    parents\IfBooleanTF{#1}{}{\ (see \Cref{par-chd-rels})}
}

\def\displayNL{\\
    $\,\hookrightarrow\,$\quad }

\DeclareDocumentCommand\seeSrcCode{ m m m g }{%
    (see the \href{
        https://github.com/samm82/TestingTesting/blob/#1/#2\#L#3%
        \IfNoValueF{#4}{-L#4}}
    {relevant source code})%
}

\def\ourApproachGlossary{\ifblind{our test approach glossary}{\href{
            https://github.com/samm82/TestingTesting/blob/main/ApproachGlossary.csv
        }{our test approach glossary}}}

\def\accelTolTest{astronauts \citep[p.~11]{MorgunEtAl1999}, aviators
    \citep[pp.~27, 42]{HoweAndJohnson1995}, or catalysts
    \citep[p.~1463]{LiuEtAl2023}}
\NewDocumentCommand{\orthTestIntro}{s}{%
    \IfBooleanTF{#1}{s}{S}ome test approaches appear to be combinations of
    other (seemingly orthogonal) approaches}
\def\impKeywords{``implied'', ``can be'', ``sometimes'', ``should be'',
    ``ideally'', ``usually'', ``most'', ``likely'', ``often'', ``if'', and
    ``although''\utd{}}

\def\recFigs{\Cref{fig:recoveryGraphs,fig:scalGraphs,fig:perf-graph}}

% Define common footnotes about IEEE testing terms for reuse
\newcommand{\distinctIEEE}[1]{distinct from the notion of ``test #1'' described
    in \Cref{tab:ieeeCats}.}
\newcommand{\notDefDistinctIEEE}[1]{\footnote{Not formally defined, but
        \distinctIEEE{#1}}}
% Defined with help from GitHub Copilot
\NewDocumentCommand{\gerrardDistinctIEEE}{s m}{%
    \IfBooleanTF{#1}{}{\footnote}%
    {``Each type of test addresses a different risk area''
        \citep[p.~12]{Gerrard2000a}, which is \distinctIEEE{#2}}
}

\def\multiCatIntro{we automatically detect test approaches with more than one
    category}

\NewDocumentCommand\parSynIntro{s}{%
    \IfBooleanTF#1{t}{T}here are also pairs of synonyms where one is described
    as a subapproach of the other, abusing the meaning of ``synonym'' and
    causing confusion. \IfBooleanTF#1{Below are all}{We identify} \parSynCount{}
    of these pairs\IfBooleanTF#1{\ that we identify}{} through automatic
    analysis of the generated graphs\ifnotpaper\ as described in
        \Cref{auto-flaw-analysis}\fi}

% Examples of flaws
\def\bugPattonFlaw{\citet[pp.~13\==14]{Patton2006} ``just call[s] it what it
    is and get[s] on with it'', abandoning these four terms, ``problem'',
    ``incident'', ``anomaly'', ``variance'', ``inconsistency'', ``feature'' (!),
    and ``a list of unmentionable terms'' in favour of ``bug''; after all,
    ``there's no reason to dice words''!}
\NewDocumentCommand\tourFlaw{s}{%
    \IfBooleanTF#1{t}{T}he structure of tours can be defined as either quite
    general \citep[p.~34]{IEEE2022} or ``organized around a special focus''
    \citepISTQB{}\IfBooleanTF#1{}{.}}
\def\alphaFlaw{Alpha testing is performed by ``users within the organization
    developing the software'' \citep[p.~17]{IEEE2017}, ``a small, selected
    group of potential users'' \citep[p.~5-8]{SWEBOK2024}, or ``roles outside
    the development organization'' conducted ``in the developer's test
    environment'' \citepISTQB{}.}
\def\loadFlaw{Load testing is performed with loads ``between anticipated
    conditions of low, typical, and peak usage'' \citep[p.~5]{IEEE2022} or
    loads that are as large as possible \citep[p.~86]{Patton2006}.}
\NewDocumentCommand{\seeRefMissing}{s}{%
    \IfBooleanTF#1{}{\refHelper} \citet[p.~42]{Kam2008} says ``See
    \emph{boundary value analysis},'' for the glossary entry of ``boundary
    value testing'' but does not include ``boundary value analysis'' in the
    glossary.}
\NewDocumentCommand{\tolTestFlaw}{s}{%
    \IfBooleanTF#1{t}{T}he terms ``acceleration tolerance testing'' and
    ``acoustic tolerance testing'' seem to only refer to software testing in
    \citep[p.~56]{Firesmith2015}; elsewhere, they seem to refer to
    testing the acoustic tolerance of rats \citep{HolleyEtAl1996} or the
    acceleration tolerance of \accelTolTest{}, which don't exactly seem
    relevant.}
\def\errorGuessFlaw{Since errors are distinct from defects/faults \ifnotpaper
    (\citealp[pp.~128, 140]{IEEE2010}; \citealp[p.~12\=/3]{SWEBOK2024};
    \citealp[pp.~399\==400]{vanVliet2000})\else
    \cite[p.~12\=/3]{SWEBOK2024}, \cite[pp.~128, 140]{IEEE2010},
    \cite[pp.~399\==400]{vanVliet2000}\fi, error guessing should
instead be called ``defect guessing'' if it is based on a ``checklist
of potential defects'' \citep[p.~29]{IEEE2021c} or ``fault guessing''
if it is a ``fault-based technique'' \citep[p.~4\=/9]{SWEBOK2014}
that ``anticipate[s] the most plausible faults in each \acs{sut}''
\citep[p.~5\=/13]{SWEBOK2024}. One (or both) of these proposed terms
may be useful in tandem with ``error guessing'', which would focus on
errors as traditionally defined; this would be a subapproach of
error-based testing (implied by \citealp[p.~399]{vanVliet2000}).}
\NewDocumentCommand{\perfSecParFlaw}{s}{%
    \IfBooleanTF#1{p}{P}erformance testing and security testing are given as subtypes of
    reliability testing by \citet{ISO_IEC2023a}, but these are all listed
    separately by \citet[p.~53]{Firesmith2015}.}
\def\parSheetTestFlaw{``Par sheet testing'' from \citepISTQB{} seems to refer
    to \ifnotpaper the specific example mentioned in \flawref{specific-istqb}
    \else some specific example they do not cite \fi and does not seem more
    widely applicable, since a ``PAR sheet'' is ``a list of all the symbols on
    each reel of a slot machine'' \citep{Bluejay2024}.}
\NewDocumentCommand{\redBoxFlaw}{s}{%
    \IfBooleanTF#1{t}{T}he incorrect claim that ``white-box
    testing'', ``grey-box testing'', and ``black-box testing'' are
    synonyms for ``module testing'', ``integration testing'', and
    ``system testing'', respectively, \ifnotpaper (see
        \flawref{dubious-syns}) \fi casts doubt on the claim that
    ``red-box testing'' is a synonym for ``acceptance testing''
    \citep[p.~18]{SneedAndGöschl2000}\todo{OG Hetzel88}\ifnotpaper\
        (see \flawref{dubious-red-box-syn})\fi.}

\NewDocumentCommand{\defLabelDistinct}{s}{%
    \IfBooleanTF#1{t}{T}erms can be thought of as definition-label pairs,
    but there is a meaningful distinction between definition flaws and label
    flaws}

% Used in parSyns tables
\def\ftrnote{Fault tolerance testing may also be a subapproach of
    reliability testing \ifnotpaper
        \citetext{\citealp[p.~375]{IEEE2017}; \citealp[p.~7-10]{SWEBOK2024}}%
    \else \cite[p.~375]{IEEE2017}, \cite[p.~7-10]{SWEBOK2024}%
    \fi, which is distinct from robustness testing \citep[p.~53]{Firesmith2015}.}
\def\specfn{See \flawref{spec-func-syn}.}
\def\ucstn{See \flawref{use-case-scenario}.}

%------------------------------------------------------------------------------
% For populating values from files
%------------------------------------------------------------------------------

\ExplSyntaxOn
\ior_new:N \g_hringriin_file_stream

\NewDocumentCommand{\ReadFile}{mm}
{
    \hringriin_read_file:nn { #1 } { #2 }
    \cs_new:Npn #1 ##1
    {
        \str_if_eq:nnTF { ##1 } { * }
        { \seq_count:c { g_hringriin_file_ \cs_to_str:N #1 _seq } }
        { \seq_item:cn { g_hringriin_file_ \cs_to_str:N #1 _seq } { ##1 } }
    }
}

\cs_new_protected:Nn \hringriin_read_file:nn
{
    \ior_open:Nn \g_hringriin_file_stream { #2 }
    \seq_gclear_new:c { g_hringriin_file_ \cs_to_str:N #1 _seq }
    \ior_map_inline:Nn \g_hringriin_file_stream
    {
        \seq_gput_right:cx
        { g_hringriin_file_ \cs_to_str:N #1 _seq }
        { \tl_trim_spaces:n { ##1 } }
    }
    \ior_close:N \g_hringriin_file_stream
}

\ExplSyntaxOff

% Define/read values for Undefined Terms methodology for reuse and calculation!
\ReadFile{\undefTermCounts}{assets/misc/undefTermCounts}

\newcount\TotalBefore
\newcount\UndefBefore
\newcount\TotalAfter
\newcount\UndefAfter

\TotalBefore=\undefTermCounts{1}
\UndefBefore=\undefTermCounts{2}
\TotalAfter=\undefTermCounts{3}
\UndefAfter=\undefTermCounts{4}

\def\approachCount{\undefTermCounts{3}}

\ReadFile{\qualityCounts}{build/qualityCount}
\def\qualityCount{\qualityCounts{1}}

\ReadFile{\multiCatCounts}{build/multiCatCounts}
\def\multiCatCount{\multiCatCounts{1}}
\def\multiCatMax{\multiCatCounts{2}}
\def\multiCatMaxCount{\multiCatCounts{3}}

\ReadFile{\multiSynCounts}{build/multiSynCounts}
\def\multiSynCount{\multiSynCounts{1}}

\ReadFile{\parSynCounts}{build/parSynCounts}
\def\parSynCount{\parSynCounts{1}}
\def\selfParCount{\parSynCounts{2}}

\ReadFile{\stdSources}{build/stdSources}
\ReadFile{\metaSources}{build/metaSources}
\ReadFile{\textSources}{build/textSources}
\ReadFile{\paperSources}{build/paperSources}

\def\srcCount{\the\numexpr\stdSources{3} + \metaSources{3} + \textSources{3} + \paperSources{3}}

\ReadFile{\stdFlawDmnBrkdwn}{build/stdFlawDmnBrkdwn}
\ReadFile{\metaFlawDmnBrkdwn}{build/metaFlawDmnBrkdwn}
\ReadFile{\textFlawDmnBrkdwn}{build/textFlawDmnBrkdwn}
\ReadFile{\paperFlawDmnBrkdwn}{build/paperFlawDmnBrkdwn}
\ReadFile{\totalFlawDmnBrkdwn}{build/totalFlawDmnBrkdwn}

\ReadFile{\stdFlawMnfstBrkdwn}{build/stdFlawMnfstBrkdwn}
\ReadFile{\metaFlawMnfstBrkdwn}{build/metaFlawMnfstBrkdwn}
\ReadFile{\textFlawMnfstBrkdwn}{build/textFlawMnfstBrkdwn}
\ReadFile{\paperFlawMnfstBrkdwn}{build/paperFlawMnfstBrkdwn}
\ReadFile{\totalFlawMnfstBrkdwn}{build/totalFlawMnfstBrkdwn}

\def\flawCount{\totalFlawMnfstBrkdwn{13}}

\def\stds{\nameref{stds}}
\def\metas{\nameref{metas}}
\def\texts{\nameref{texts}}
\NewDocumentCommand{\papers}{s}{%
    \IfBooleanTF{#1}{\hyperref[papers]{Papers and Others}}{\nameref{papers}}%
}

\def\srcCat{\hyperref[sources]{Source Tier}}

\input{build/FlawDmnMacros}
\input{build/FlawMnfstMacros}

% Assisted by GitHub Copilot
\newcommand\macro[2][]{\texttt{\textbackslash#2\{#1\}}}

%------------------------------------------------------------------------------
% TODOs
%------------------------------------------------------------------------------

% Generic Inlined TODOs
\newcommand{\intodo}[1]{\todo[inline]{#1}}

% Unimportant TODOs for "later" (i.e., finishing touches or changes immediately before submission)
\newcommand{\latertodo}[1]{\todo[backgroundcolor=Cyan]{\textit{Later}: #1}}

% "Important" TODOs
\newcommand{\imptodo}[1]{\todo[inline,backgroundcolor=Red]{\textbf{Important}: #1}}

% "Easy" TODOs
\newcommand{\easytodo}[1]{\todo[inline,backgroundcolor=SeaGreen]{\textit{Easy}: #1}}
\newcommand{\eztodo}[1]{\easytodo{#1}}

% "Tedious" TODOs
\newcommand{\tedioustodo}[1]{\todo[inline,backgroundcolor=PineGreen]{\textit{Needs time}: #1}}

% "Question" TODO Notes
\newcounter{todonoteQuestionsCtr}
\newcommand{\questiontodo}[1]{\stepcounter{todonoteQuestionsCtr}\todo[backgroundcolor=Lavender]{\textbf{Q \#\thetodonoteQuestionsCtr{}}: #1}}
\newcommand{\qtodo}[1]{\questiontodo{#1}}

% Specific categories of TODOs
\def\ptq{\todo{Present tense?}}
\def\utd{\latertodo{Ensure this is up to date}}

%------------------------------------------------------------------------------
% Link to Drasil issue
%------------------------------------------------------------------------------

\newcommand{\issueref}[1]{\href{https://github.com/JacquesCarette/Drasil/issues/#1}{\##1}}
\newcommand{\pullref}[1]{\href{https://github.com/JacquesCarette/Drasil/pull/#1}{\##1}}
\newcommand{\thesisissuerefhelper}[1]{\href{https://github.com/samm82/TestingTesting/issues/#1}{\##1}}

\ExplSyntaxOn

% Based on output from ChatGPT
\NewDocumentCommand{\mapthesisissueref}{m}
{
    % Clear temporary sequences to store transformed items
    \seq_clear:N \l_tmpa_seq
    \seq_clear:N \l_tmpb_seq

    \seq_set_split:Nnn \l_tmpa_seq { , } { #1 } % Split the input by commas
    \seq_map_inline:Nn \l_tmpa_seq
    {
        \seq_put_right:Nn \l_tmpb_seq {\thesisissuerefhelper{##1}}
    }
    \seq_use:Nnnn \l_tmpb_seq { ~and~ } { ,~ } { ,~and~ }
}

\ExplSyntaxOff

\newcommand{\thesisissueref}[1]{\todo[backgroundcolor=lightgray]{See \mapthesisissueref{#1}}}
