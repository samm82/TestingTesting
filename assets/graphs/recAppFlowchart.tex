% Based on https://www.overleaf.com/learn/latex/LaTeX_Graphics_using_TikZ%3A_A_Tutorial_for_Beginners_(Part_3)%E2%80%94Creating_Flowcharts

\tikzstyle{startend} = [rectangle, rounded corners,
minimum width=3cm,
minimum height=1cm,
text centered,
text width=3cm,
draw=black,
fill=red!30]

\tikzstyle{io} = [trapezium,
trapezium stretches=true, % A later addition
trapezium left angle=70,
trapezium right angle=110,
minimum width=3cm,
minimum height=1cm,
text centered,
text width=3cm,
draw=black, fill=blue!30]

\tikzstyle{process} = [rectangle,
minimum width=3cm,
minimum height=1cm,
text centered,
text width=4cm,
draw=black,
fill=orange!30]

\tikzstyle{decision} = [diamond,
aspect=2,
minimum width=0.25cm,
minimum height=0.25cm,
text centered,
text width=2cm,
draw=black,
fill=green!30]

\tikzstyle{arrow} = [thick,->,>=stealth]

\begin{figure}
    \begin{tikzpicture}[node distance=2cm]

        \node (start) [startend] {Start};
        \node (ident) [io, right of=start, xshift=3cm] {New Test Approach Identified};
        \node (present) [decision, right of=ident, xshift=3cm] {Present in Glossary?};

        \node (newrow) [process, below of=present, yshift=-0.5cm] {Create New Row in Glossary};
        \node (newapp) [process, below of=newrow]  {Record Approach Name and Category as ``Approach''};

        \node (catgiven) [decision, below of=present, xshift=-5cm, yshift=-0.5cm] {Category Given?};

        \node (out1) [io, below of=catgiven] {Output};
        \node (end) [startend, below of=out1] {End};

        \draw [arrow] (start) -- (ident);
        \draw [arrow] (ident) -- (present);
        \draw [arrow] (present) -- node[anchor=south] {yes} (catgiven);
        \draw [arrow] (present) -- node[anchor=east] {no} (newrow);
        \draw [arrow] (newrow) -- (newapp);
        \draw [arrow] (newapp.west) -- (catgiven);
        \draw [arrow] (catgiven) -- (out1);
        \draw [arrow] (out1) -- (end);

    \end{tikzpicture}
    \caption{Procedure for recording test approaches in our glossary.}
    \label{fig:recAppFlowchart}
\end{figure}