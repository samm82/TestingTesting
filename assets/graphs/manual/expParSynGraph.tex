\documentclass{article}
\usepackage{graphicx}
\usepackage[pdf]{graphviz}
\usepackage{tikz}
\usetikzlibrary{arrows,shapes}

\begin{document}
\digraph{expParSynGraph}{
rankdir=BT;

// Dummy node to push the legend to the top left
start [style="invis"];

ExhaustiveTesting [label=<Exhaustive<br/>Testing>];
FaultToleranceTesting [label=<Fault<br/>Tolerance<br/>Testing>];
PathTesting [label=<Path<br/>Testing>];
RobustnessTesting [label=<Robustness<br/>Testing>];
ScenarioTesting [label=<Scenario<br/>Testing>];
StaticAnalysis [label=<Static<br/>Analysis>];
StaticTesting [label=<Static<br/>Testing>];
UseCaseTesting [label=<Use<br/>Case<br/>Testing>];

FaultToleranceTesting -> RobustnessTesting[dir=none,color="blue"];
FaultToleranceTesting -> RobustnessTesting[color="blue"];

PathTesting -> ExhaustiveTesting[dir=none,color="maroon"];
PathTesting -> ExhaustiveTesting[color="maroon"];

StaticAnalysis -> StaticTesting[dir=none,color="maroon"];
StaticAnalysis -> StaticTesting[color="green"];

UseCaseTesting -> ScenarioTesting[dir=none,color="blue"];
UseCaseTesting -> ScenarioTesting[color="green"];

% From https://stackoverflow.com/a/64007295/10002168
{
rank = same;
edge[style=invis];
UseCaseTesting -> StaticAnalysis -> FaultToleranceTesting -> PathTesting;
rankdir = LR;
}

}
\end{document}
