\documentclass{article}
\usepackage{graphicx}
\usepackage[pdf]{graphviz}
\usepackage{tikz}
\usetikzlibrary{arrows,shapes}

\begin{document}
\digraph[scale=0.8]{recoveryGraphCurrent}{
rankdir=BT;
avail   [label=<Availability<br/>Testing>];
bar     [label=<Backup and<br/>Recovery<br/>Testing>];
br      [label=<Backup/Recovery<br/>Testing>];
dr      [label=<Disaster/Recovery<br/>Testing>];
fail    [label=<Failover<br/>Testing>];
failr   [label=<Failover/Recovery<br/>Testing>];
failt   [label=<Failure<br/>Tolerance<br/>Testing>];
ft      [label=<Fault<br/>Tolerance<br/>Testing>];
perf    [label=<Performance<br/>Testing>];
perfRel [label=<Performance-related<br/>Testing>];
recab   [label=<Recoverability<br/>Testing>];
recvr   [label=<Recovery<br/>Testing>];
rel     [label=<Reliability<br/>Testing>];
use     [label=<Usability<br/>Testing>];

avail -> { rel };
bar   -> { rel };
br    -> { dr };
fail  -> { failt };
fail  -> { failr }[dir=none, style="dashed"];
failr -> { dr };
ft    -> { avail rel };
dr    -> { ft recab };
perf  -> { perfRel };
recab -> { rel };
recab -> { use }[style="dashed"];
recvr -> { avail ft perfRel rel };
recvr -> { recab }[dir=none];
rel   -> { perf };

{ rank=same; fail failr };
{ rank=same; recab recvr };

// for formatting
failr -> { recvr }[style="invis"];
}
\end{document}