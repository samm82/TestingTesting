% makecell with new lines so VS Code doesn't freak out
\def\app{\makecell{Approach\\Category}}

\begin{table}[hbtp!]
    \centering
    % Duplication required because '\ourApproachGlossary{}' doesn't work in table of contents
    \caption[Selected entries from our test approach glossary with ``Notes'' column excluded for brevity.]%
    {Selected entries from \ourApproachGlossary{} with ``Notes'' column excluded for brevity.}
    \label{tab:approachGlossaryExcerpt}
    \begin{tabularx}{\linewidth}{|m{1.8cm}|>{\raggedright\arraybackslash}m{3.5cm}|>{\raggedright\arraybackslash}X|>{\raggedright\arraybackslash}m{4.25cm}|>{\raggedright\arraybackslash}m{4.25cm}|}
        \hline
        \thead{Name}         & \thead{\app}                                                              & \thead{Definition}                                                                                                                                                                                                                     & \thead{Parent(s)}                                                                               & \thead{Synonym(s)}                                                                                                             \\
        \hline
        A/B Testing          & Practice \citep[Fig.~2]{IEEE2022}, Type (inferred from usability testing) & Testing ``that allows testers to determine which of two systems or components performs better'' \citep[pp.~1, 36]{IEEE2022}                                                                                                            & Statistical Testing \citep[pp.~1, 36]{IEEE2022}, Usability Testing \citep[p.~58]{Firesmith2015} & Split-Run Testing \citep[pp.~1, 36]{IEEE2022}                                                                                  \\%[1.5cm]
        % All Combinations Testing & Technique (\citealp[p.~22]{IEEE2022}; \citeyear[pp.~2, 16]{IEEE2021c}; \citealp[p.~5\=/11]{SWEBOK2024}) & Testing that covers ``all unique combinations of P-V pairs'' \citep[p.~16]{IEEE2021c}                                                                                                    & Combinatorial Testing \citetext{\citealp[p.~22]{IEEE2022}; \citeyear[pp.~2, 16, Fig.~2]{IEEE2021c}; \citealp[p.~5\=/11]{SWEBOK2024}}                                                           & ---                                                                     \\[1cm]
        Back-to-Back Testing & Practice \citep[p.~22]{IEEE2022}                                          & Testing ``whereby an alternative version of the system is used to generate expected results for comparison from the same test inputs'' \citep[p.~2]{IEEE2022} (IEEE, 2022, p. 2) \dots{}                                               & Non-functional Testing \citep[p.~5\=/9]{SWEBOK2024}                                             & Differential Testing \citep[p.~2]{IEEE2022}                                                                                    \\[1.5cm]
        %                                                                                                                                                              … or where "two or more variants of a program are executed with the same inputs, the outputs are compared, and errors are analyzed in case of discrepancies" (2010, p. 30; similar in Hamburg and Mogyorodi, 2024; OG Spillner)
        % Big-Bang Testing     & Technique \citep[pp.~601, 603, 605\==606]{SharmaEtAl2021}, Level (inferred from integration testing) & ``Testing in which \dots{} [components of a system] are combined all at once into an overall system, rather than in stages'' \citep[p.~45]{IEEE2017}                                     & Integration Testing (\citealp[p.~45]{IEEE2017}; \citealp[p.~5\=/7]{SWEBOK2024}; \citealp[p.~603]{SharmaEtAl2021}; \citealp[p.~42]{Kam2008}; \citealp[p.~488, Tab.~12.8]{PetersAndPedrycz2000}) & Sometimes spelled without a hyphen \citep[p.~489]{PetersAndPedrycz2000} \\[1.5cm]
        % Classification Tree Method & Technique (\citealp[p.~22]{IEEE2022}; \citeyear[pp.~2, 12, Fig.~2]{IEEE2021c}; \citealpISTQB{}; \citealp[p.~47]{Firesmith2015}) & Testing ``based on exercising classes in a classification tree'' \citep[p.~2]{IEEE2021c}                                                                                                 & Specification-based Testing (\citealp[p.~22]{IEEE2022}; \citeyear[pp.~2, 12, Fig.~2]{IEEE2021c}; \citealpISTQB{}; \citealp[p.~47]{Firesmith2015}), Model-based Testing (\citealp[p.~13]{IEEE2022}; \citeyear[pp.~6, 12]{IEEE2021c}) & Classification Tree Technique \citepISTQB{}, Classification Tree Testing \citep[p.~47]{Firesmith2015} \\[1.5cm]
        % Data Flow Testing    & Technique (\citealp[p.~22]{IEEE2022}; \citeyear[pp.~3, 27]{IEEE2021c}; \citealp[p.~5\=/13]{SWEBOK2024}; \citealp[p.~43]{Kam2008}) & A ``class of \dots{} techniques based on exercising definition-use pairs'' \citep[p.~3; similar on p.~27]{IEEE2021c}                                                                     & Structure-based Testing (\citealp[p.~22]{IEEE2022}; \citeyear[pp.~3, 27, Fig.~2]{IEEE2021c}; \citealp[p.~43]{Kam2008}), Control Flow Testing (\citeyear[p.~27]{IEEE2021c}; implied by \citealp[p.~5\=/13]{SWEBOK2024}; \citealp[p.~101]{IEEE2017}), Model-based Testing (\citeyear[p.~27]{IEEE2021c}; implied by \citealp[p.~179]{DoğanEtAl2014}), Web Application Testing (p. 179; can be in \citealp[pp.~16\==17]{Kam2008}) & ---                                           \\
        Retesting            & Type \citepISTQB{}                                                        & Testing ``performed to check that modifications made to correct a fault have successfully removed the fault'' (\citealp[p.~8]{IEEE2022}; \citeyear[p.~3]{IEEE2021a}; similar in \citeyear[p.~386]{IEEE2017}; \citealpISTQB{}), \dots{} & Change-Related Testing \citepISTQB{}                                                            & Confirmation Testing (\citealp[pp.~8, 35]{IEEE2022}; \citeyear[p.~3]{IEEE2021a}; \citeyear[p.~386]{IEEE2017}; \citealpISTQB{}) \\
        %                                                                                                  , often by rerunning test cases that previously failed (IEEE, 2022, p. 35; 2017, p. 386; Kam, 2008, p. 47) that may be "supplemented by new test cases that provide improved coverage" (IEEE, 2022, p. 35)                                                                                                                                 Sometimes spelled with a hyphen \citepISTQB{}, 
        \hline
    \end{tabularx}
\end{table}