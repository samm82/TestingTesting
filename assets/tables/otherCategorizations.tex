\def\selecExs{Deterministic Testing\\ Random Testing}
\def\infoSrcExs{Specification-based Testing\\ Structure-based Testing\\
    Experience-based Testing}
\def\sdlcExs{Waterfall Testing\\ Incremental Testing\\ \acf{ct}}
\def\reasExs{Smoke Testing\\ Initial Testing\\ Regression Testing}
\def\execExs{Static Testing\\ Dynamic Testing}
\def\goalExs{Verification Testing\\ Validation Testing}
\def\reqExs{Functional Testing\\ Non-functional Testing}
\def\humInvExs{Manual Testing\\ Automated Testing}
\def\strExs{Scripted Testing\\ Exploratory Testing\TblrNote{f}}
\def\factExs{Correctness Testing\\ Response-Time Testing\\ Access Control Testing\\
    Compliance Testing\\ Reliability Testing\\ Maintainability Testing\\ Portability Testing\\
    Performance Testing\TblrNote{e}}
\def\adqCritExs{Coverage-based Testing\\ Fault-based Testing\\ Error-based Testing}

\def\parCat{Parent\TblrNote{a} IEEE\\ Category\TblrNote{b}}

\begin{longtblr}[
    note{a} = {Defined in \Cref{par-chd-rels}.},
    note{b} = {Defined in \Cref{cats-def}.},
    note{c} = {These approaches may instead be techniques
            \citep[implied by][p.~35; see \Cref{tab:multiCats}]{IEEE2022}.},
    note{d} = {We also consider this categorization meaningful (see \Cref{static-test}).},
    note{e} = {Experience-based testing may instead be a practice (\citealp[pp.~22, 34]{IEEE2022};
            \citeyear[p.~viii]{IEEE2021}; see \Cref{tab:multiCats}).},
    caption = {Alternate categorizations found in the literature.},
    label = {tab:otherCategorizations}
    ]{
    colspec = {|X[0.25,c,m]X[0.25,c,m]X[0.15,c,m]X[0.35,l,m]|},
    width = \linewidth, rowhead = 1
    }
    \hline
    \thead{Categorization Basis} & \thead{Example Approaches} & \thead{\parCat{}}     & \thead{Source}                         \\
    \hline
    Level of Automation % (\citealp[p.~44]{Firesmith2015}; \citealp[p.~214]{KuļešovsEtAl2013})
                                 & \humInvExs{}               & Practice\TblrNote{c}  & \citep[pp.~22, 35]{IEEE2022}           \\
    \hline
    Execution of Code\TblrNote{d}
    % (\citealp[p.~53]{Patton2006}; \citealp[p.~214]{KuļešovsEtAl2013}; \citealp[p.~12]{Gerrard2000a}
                                 & \execExs{}                 & Approach              & \citep[pp.~17\==18]{IEEE2022}          \\
    \hline
    Goal
    % (\citealp[pp.~69--70]{Perry2006}; \citealp[p.~214]{KuļešovsEtAl2013})
                                 & \goalExs{}                 & Approach              & \citep[p.~16]{IEEE2022}                \\
    \hline
    % Visibility of the \acs{sut}'s Internal Structure % (\citealp[p.~8]{IEEE2021}; \citealp[pp.~5\=/10, 5\=/16]{SWEBOK2024};
    % % \citealp[pp.~53, 218]{Patton2006}; \citealp[p.~69]{Perry2006}; \citealp[p.~601; OG {[8]}]{SharmaEtAl2021};
    % % \citealp[pp.~57--58]{AmmannAndOffutt2017}; \citealp[p.~213]{KuļešovsEtAl2013}; \citep[pp.~4--5]{Kam2008})                       
    %                    & \visibExs{}                & Technique
    % % (\citealp[pp.~4, 8]{IEEE2021}; \citealp[p.~46]{Firesmith2015})        
    %                    & \citep[pp.~4, 8]{IEEE2021}                                                                  \\
    % \hline
    Source of Information for Designing Tests % \citep[p.~8; one other source]{IEEE2021} 
                                 & \infoSrcExs{}              & Technique\TblrNote{e}
    % (\citealp[p.~22]{IEEE2022}; \citeyear[p.~4]{IEEE2021};
    % \citealp[pp.~5\=/10, 5\=/13]{SWEBOK2024}; \citealpISTQB{}; \citealp[p.~46]{Firesmith2015}) 
                                 & \citep[pp.~4, 8]{IEEE2021}                                                                  \\
    \hline
    % Source of Test Data
    % \citep[p.~440]{PetersAndPedrycz2000}    & \dataSrcExs{}              & Technique                                                                                                                                \\
    % \hline
    % Coverage Requirement % \citep[pp.~4--5]{Kam2008}             
    %                    & \covReqExs{}               & Technique             & \citep[p.~5\=/13]{SWEBOK2024}          \\
    % %    \citealp[p.~49]{Firesmith2015}                                         
    % \hline
    Selection Process % \citep[p.~5-16]{SWEBOK2024}            
                                 & \selecExs{}                & Technique             & \citep[pp.~5\=/12, 5\=/16]{SWEBOK2024} \\
    \hline
    % \questBase{}                             & \questExs{}                & Approach                                                                                                                                                               \\
    % \hline
    Stage of Lifecycle           & \sdlcExs{}                 & Practice (inferred)   & \citep[p.~29]{Firesmith2015}           \\
    \hline
    % Reason             & \reasExs{}                 & Technique (inferred)  & \citep[p.~34]{Firesmith2015}                   \\
    % \hline
    % Seems synonymous with testing based on software qualities
    % Test Factor (also called Quality Factor or Quality Attribute)
    % \citep[pp.~40--41]{Perry2006}           & \factExs{}                 & Type (\citealp[p.~22]{IEEE2022}; and/or implied by its quality and/or \citealp{Firesmith2015})                                           \\
    % \hline
    Adequacy Criterion           & \adqCritExs{}              & Technique             & \citep[pp.~398--399]{vanVliet2000}     \\
    \hline
    % Coverage Criteria
    % \citep[pp.~18--19]{AmmannAndOffutt2017} & \covCritExs{}              & Technique (\citealp[p.~22]{IEEE2022}; \citeyear[Fig.~2]{IEEE2021}; \citealp[p.~5\=/11]{SWEBOK2024}; \citealp[pp.~47--48]{Firesmith2015})                                             \\
    % \hline
    % Structuredness
    % \citep[p.~214]{KuļešovsEtAl2013}        & \strExs{}                  & Practice\TblrNote{g} \citep[pp.~20, 22]{IEEE2022}                                                                                                                                    \\
    % \hline
    % Requirement Type
    % % Target (\citealp[p.~213]{KuļešovsEtAl2013}; \citealp[pp.~4--5]{Kam2008})
    %                    & \reqExs{}                  & Ambiguous (see \Cref{func-test-flaw}) & \citep[p.~213]{KuļešovsEtAl2013}       \\
    % \hline
    % Risk Area \citep[p.~12]{Gerrard2000a}   & \typeCatExs{}              & Ambiguous                                                                                                                                                                            \\
    % \hline
    % Priority (in the context of testing e-business projects)
    % \citep[p.~13]{Gerrard2000a}             & \priorExs{}                & Type\TblrNote{h} (\citealp[p.~22]{IEEE2022}; \citeyear[Tab.~A.1]{IEEE2021}; and/or implied by \citealp[p.~53]{Firesmith2015})                                                        \\
    % \hline
    % Purpose \citep{Pan1999}                 & \purpExs{}                 & Type (\citealp[p.~22]{IEEE2022}; and/or implied by \citealp[p.~53]{Firesmith2015})                                                                                                   \\
    % \hline
\end{longtblr}
