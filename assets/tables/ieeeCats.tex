% Conversion to (longtblr) talltblr assisted by GitHub Copilot

\begin{center}
    \begin{talltblr}[
        note{a} = {Also called ``test phase'' \ifnotpaper (see
                \flawref{level-phase-syns}) \fi or ``test stage'' \ifnotpaper
                (see \flawref{stage-level-syns})\else (see relevant synonym
                flaws in \Cref{syns})\fi.},
        note{b} = {Also called ``test design technique'' \ifnotpaper
                (\citealp[p.~11]{IEEE2022}; \citeyear[p.~5]{IEEE2021a};
                \citealpISTQB{})\else \cite[p.~11]{IEEE2022},
                \cite[p.~5]{IEEE2021a}, \cite{ISTQB}\fi.},
        caption={Categories of test approaches given by ISO/IEC and IEEE.},
        label={tab:ieeeCats}
        ]{
        colspec={|X[0.09,c,m]X[0.575,m]X[0.285,m]|},
        width = \linewidth, rowhead = 1, hlines
        }
        \thead{Term}               & \thead{Definition}                           & \thead{Examples} \\
        Test Level\TblrNote{a}     & A stage of testing ``typically associated
        with the achievement of particular objectives and used to treat particular
        risks'', each performed in sequence \ifnotpaper (\citealp[p.~12]{IEEE2022};
        \citeyear[p.~6]{IEEE2021a}; \citeyear[p.~6]{IEEE2021c}) \else \cite[p.~12]{IEEE2022},
        \cite[p.~6]{IEEE2021c}, \cite[p.~6]{IEEE2021a}
        \fi with their ``own documentation and resources''
        \citeyearpar[p.~469]{IEEE2017} % ; more generally, ``designat[es] \dots\ the
        % coverage and detail'' \citeyearpar[p.~249]{IEEE2017} 
                                   & unit/component testing, integration testing,
        system testing, acceptance testing \ifnotpaper (\citeyear[p.~12]{IEEE2022};
        \citeyear[p.~6]{IEEE2021a}; \citeyear[p.~6]{IEEE2021c};
        \citeyear[p.~467]{IEEE2017}) \else \cite[p.~467]{IEEE2017}, \cite[p.~12]{IEEE2022},
        \cite[p.~6]{IEEE2021c}, \cite[p.~6]{IEEE2021a} \fi                                           \\
        Test Practice              & A ``conceptual framework that can be
        applied to \dots{} [a] test process to facilitate testing'' \ifnotpaper
        (\citeyear[p.~14]{IEEE2022}; \citeyear[p.~471]{IEEE2017})
        \else \cite[p.~471]{IEEE2017}, \cite[p.~14]{IEEE2022}
        \fi % ; more generally, a ``specific type of activity that contributes to
        % the execution of a process'' \citeyearpar[p.~331]{IEEE2017} 
                                   & scripted testing, exploratory testing,
        automated testing \citeyearpar[p.~20]{IEEE2022}                                              \\
        Test Technique\TblrNote{b} & A ``procedure used to create or select a
        test model \dots, identify test coverage items \dots, and derive
        corresponding test cases'' \ifnotpaper (\citeyear[p.~11]{IEEE2022};
        \citeyear[p.~5]{IEEE2021a}; similar in \citeyear[p.~467]{IEEE2017})
        \else \cite[p.~11]{IEEE2022}, \cite[p.~5]{IEEE2021a} (similar in
        \cite[p.~467]{IEEE2017}) \fi that ``generate evidence that test item
        requirements have been met or that defects are present in a test item''
        \citeyearpar[p.~vii]{IEEE2021c} ``typically used to achieve a required
        level of coverage'' \citeyearpar[p.~5]{IEEE2021a}
        % ; ``a variety \dots\ is typically
        % required to suitably cover any system'' \citeyearpar[p.~33]{IEEE2022} and
        % is ``often selected based on team skills and familiarity, on the format
        % of the test basis'', and on expectations \citeyearpar[p.~23]{IEEE2022}
                                   & equivalence partitioning,
        boundary value analysis, branch testing \ifnotpaper (\citeyear[p.~11]{IEEE2022};
        \citeyear[p.~5]{IEEE2021a}) \else \cite[p.~11]{IEEE2022}, \cite[p.~5]{IEEE2021a} \fi         \\
        Test Type                  & ``Testing that is focused on specific
        quality characteristics'' \ifnotpaper (\citeyear[p.~15]{IEEE2022};
        \citeyear[p.~7]{IEEE2021c}; \citeyear[p.~473]{IEEE2017})
        \else \cite[p.~473]{IEEE2017}, \cite[p.~15]{IEEE2022}, \cite[p.~7]{IEEE2021c}
        \fi                        & security testing, usability testing,
        performance testing \ifnotpaper (\citeyear[p.~15]{IEEE2022};
        \citeyear[p.~8]{IEEE2021a}; \citeyear[p.~473]{IEEE2017}) \else
        \cite[p.~473]{IEEE2017}, \cite[p.~15]{IEEE2022}, \citeyear[p.~8]{IEEE2021a} \fi              \\
    \end{talltblr}
\end{center}
