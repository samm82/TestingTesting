\documentclass{beamer}

% Manifest data
\input{manifest}

\usepackage{amsmath}
\usepackage{textcomp}
\usepackage{listings}
\usepackage{lmodern}
\usepackage[T1]{fontenc}
\usepackage{tikz}
\usepackage{tikzsymbols}
\usepackage{anyfontsize}

\usepackage{makecell, tabularx}
\newcolumntype{M}{>{\raggedright\arraybackslash}m}
\renewcommand\tabularxcolumn[1]{M{#1}}
\renewcommand\arraystretch{1.2}
\usetikzlibrary{shapes,shapes.geometric,arrows,graphs,graphdrawing,fit,calc,positioning,automata,decorations.pathreplacing}
\usepackage{pgf-pie,pgfplots}
\pgfplotsset{compat=1.9}

\usepackage{multicol}

\usepackage[disable]{todonotes}
% Disable footnotes; from https://tex.stackexchange.com/a/240494/192195
\renewcommand{\footnote}[2][]{\relax}

% From https://tex.stackexchange.com/a/283202/192195
\usepackage[shortcuts]{extdash}

% Required for biblatex, but also adds functionality for quotation
\usepackage{csquotes}

% Jason's bibliography format
% % Credit to Gabriel Devenyi for this bibliography cfg:
% % github.com/gdevenyi/mcmaster.latex
% \usepackage[
%   style=numeric-comp,
%   backend=biber,
%   sorting=none,
%   backref=true,
%   maxnames=99,
%   alldates=iso,
%   seconds=true
% ]{biblatex} % bibliography
% \addbibresource{references.bib}
\usepackage[round]{natbib}
\bibliographystyle{plainnat}
\setcitestyle{yysep={;}}
\defcitealias{ISTQB}{Hamburg and Mogyorodi}
\newcommand{\citetISTQB}{\citetalias{ISTQB} (\citeyear{ISTQB})}
\newcommand{\citepISTQB}{\citepalias[\citeyear{ISTQB}]{ISTQB}}
\newcommand{\citealpISTQB}{\citetalias{ISTQB}, \citeyear{ISTQB}}

\lstset{
    language=[latex]tex,
    breaklines=true}

\usetheme{Madrid}

\setbeamertemplate{caption}{\centering\insertcaption\par}

% Change block width
\addtobeamertemplate{block begin}{%
    \centering\large%
    \setlength{\textwidth}{0.9\textwidth}
}{}

% From https://tex.stackexchange.com/a/489625/192195
\BeforeBeginEnvironment{block}{\begin{adjustbox}{minipage={\linewidth}, center}}
    \AfterEndEnvironment{block}{\end{adjustbox}}

\usepackage{adjustbox}

\def\checkmark{\tikz\fill[scale=0.4](0,.35) -- (.25,0) -- (1,.7) -- (.25,.15) -- cycle;} 

% Extra functionality for command parsing
\usepackage{xparse}

\newif\ifnotpaper

%------------------------------------------------------------------------------
% Reused in seminar slides
%------------------------------------------------------------------------------

\def\rqatext{What testing approaches do the literature describe?}
\def\rqbtext{Are these descriptions consistent?}
\def\rqctext{Can we systematically resolve any of these inconsistencies?}

\def\expBasedCatMain{\citeauthor{IEEE2022} categorize experience-based testing
    as both a test design technique and a test practice on the same
    page---twice \citeyearpar[Fig.~2, p.~34]{IEEE2022}!}

\NewDocumentCommand{\perfAsFamily}{s}{%
    \IfBooleanTF#1{\citealp}{\citep}[p.~1187]{Moghadam2019}\footnote{
        The original source describes ``performance testing \dots\ as a family
        of performance-related testing techniques'', but it makes more sense to
        consider ``performance-related testing'' as the ``family'' with
        ``performance testing'' being one of the variabilities
        (see \Cref{perf-test-rec}).}%
}

%------------------------------------------------------------------------------
% Spacing Options
%------------------------------------------------------------------------------

\newcommand{\thesisForceSingleSpacing}{\singlespacing}
\newcommand{\thesisForceDoubleSpacing}{\doublespacing}

%------------------------------------------------------------------------------
% Portable HREFs
%------------------------------------------------------------------------------

% Common variant
\newcommand{\porthref}[2]{\href{#2}{#1}\printOnlyFootnote{\url{#2}}}

% Custom URLs
\newcommand{\porthreft}[3]{\href{#3}{#1}\printOnlyFootnote{\href{#3}{#2}}}
% Inside of some environments, footnote marks aren't registered properly, so we
% need to manually write the "text" part
\newcommand{\porthreftm}[2]{\href{#2}{#1\printOnlyFootnoteMark}}

\newcommand{\formatPaper}[2]{%
    \ifnotpaper
        #1{#2}%
    \else
        \underline{#2}%
    \fi
}

\def\refHelper{\ifnotpaper\else Reference \fi}
\newcommand\multiAuthHelper[1]{\ifnotpaper #1\else #1s\fi}

\newcommand\discrepref[1]{%
    \ifnotpaper
        \labelcref{#1-discrep}%
    \else
        \Cref{#1-discrep}%
    \fi}

\newcommand\ifblind[2]{\IfEndWith*{\jobname}{_blind}{#1}{#2}}

% For `TblrNote`s in the middle of a cell (i.e., with following content)
% From https://topanswers.xyz/tex?q=4758
\ExplSyntaxOn
\NewDocumentCommand \MidTblrNote { m }
{
    \cs_if_exist:NT \hypersetup { \ExpTblrTemplate { note-border }{ default } }
    {
        \__tblr_hyper_link:nn {#1}
        { \textsuperscript { \UseTblrFont { note-tag } #1 } }
    }
}
\ExplSyntaxOff

%------------------------------------------------------------------------------
% Generic "chunks" that get reused
%------------------------------------------------------------------------------

\newenvironment{bigLandscape}{
    \newgeometry{hmargin=1cm, vmargin=2.5cm}
    \begin{landscape}
        }{
    \end{landscape}
    \restoregeometry
    \newpage
}

\DeclareDocumentCommand\seeSrcCode{ m m m g }{%
    (see the \href
    {https://github.com/samm82/TestGen-Thesis/blob/#1/scripts/#2.py\#L#3%
        \IfNoValueF {#4} {-L#4}}
    {relevant source code})%
}

\newcommand{\accelTolTest}{astronauts \citep[p.~11]{MorgunEtAl1999}, aviators
    \citep[pp.~27,~42]{HoweAndJohnson1995}, or catalysts
    \citep[p.~1463]{LiuEtAl2023}}

\def\recFigs{\Cref{fig:recoveryGraphs,fig:scalGraphs,fig:perf-graph}}

% Define common footnotes about IEEE testing terms for reuse
\newcommand{\distinctIEEE}[1]{distinct from the notion of ``test #1'' described
    in \Cref{tab:ieeeTestTerms}.}
\newcommand{\notDefDistinctIEEE}[1]{\footnote{Not formally defined, but
        \distinctIEEE{#1}}}
\newcommand{\gerrardDistinctIEEE}[1]{\footnote{``Each type of test addresses a
        different risk area'' \citep[p.~12]{Gerrard2000a}, which is
        \distinctIEEE{#1}}}

% Examples of discrepancies
\NewDocumentCommand\tourDiscrep{s}{%
    \IfBooleanTF#1{t}{T}he structure of tours can be defined as either quite
    general \citep[p.~34]{IEEE2022} or ``organized around a special focus''
    \citepISTQB{}\IfBooleanTF#1{}{.}}
\def\alphaDiscrep{Alpha testing is performed by ``users within the organization
    developing the software'' \citep[p.~17]{IEEE2017}, ``a small, selected
    group of potential users'' \citep[p.~5-8]{SWEBOK2024}, or ``roles outside
    the development organization'' conducted ``in the developer's test
    environment'' \citepISTQB{}.}
\def\loadDiscrep{Load testing is performed with loads ``between anticipated
    conditions of low, typical, and peak usage'' \citep[p.~5]{IEEE2022} or
    loads that are as large as possible \citep[p.~86]{Patton2006}.}

\def\suggSrcs{\href
    {https://github.com/samm82/TestGen-Thesis/issues/14\#issuecomment-1839922715}
    {suggested by Dr.~Carette}}

% Used in parSyns tables
\def\ftrnote{Fault tolerance testing may also be a sub-approach of
    reliability testing \ifnotpaper
        \citetext{\citealp[p.~375]{IEEE2017}; \citealp[p.~7-10]{SWEBOK2024}}%
    \else \cite[p.~375]{IEEE2017}, \cite[p.~7-10]{SWEBOK2024}%
    \fi, which is distinct from robustness testing \citep[p.~53]{Firesmith2015}.}
\def\specfn{See \Cref{spec-func-test}.}
\def\ucstn{See \discrepref{use-case-scenario}.}

%------------------------------------------------------------------------------
% For populating values from files
%------------------------------------------------------------------------------

\ExplSyntaxOn
\ior_new:N \g_hringriin_file_stream

\NewDocumentCommand{\ReadFile}{mm}
{
    \hringriin_read_file:nn { #1 } { #2 }
    \cs_new:Npn #1 ##1
    {
        \str_if_eq:nnTF { ##1 } { * }
        { \seq_count:c { g_hringriin_file_ \cs_to_str:N #1 _seq } }
        { \seq_item:cn { g_hringriin_file_ \cs_to_str:N #1 _seq } { ##1 } }
    }
}

\cs_new_protected:Nn \hringriin_read_file:nn
{
    \ior_open:Nn \g_hringriin_file_stream { #2 }
    \seq_gclear_new:c { g_hringriin_file_ \cs_to_str:N #1 _seq }
    \ior_map_inline:Nn \g_hringriin_file_stream
    {
        \seq_gput_right:cx
        { g_hringriin_file_ \cs_to_str:N #1 _seq }
        { \tl_trim_spaces:n { ##1 } }
    }
    \ior_close:N \g_hringriin_file_stream
}

\ExplSyntaxOff

% Define/read values for Undefined Terms methodology for reuse and calculation!
\ReadFile{\undefTermCounts}{assets/misc/undefTermCounts}

\newcount\TotalBefore
\newcount\UndefBefore
\newcount\TotalAfter
\newcount\UndefAfter

\TotalBefore=\undefTermCounts{1}
\UndefBefore=\undefTermCounts{2}
\TotalAfter=\undefTermCounts{3}
\UndefAfter=\undefTermCounts{4}

\def\approachCount{\undefTermCounts{3}}

\ReadFile{\qualityCounts}{build/qualityCount}
\def\qualityCount{\qualityCounts{1}}

\ReadFile{\parSynCounts}{build/parSynCounts}
\def\parSynCount{\parSynCounts{1}}
\def\selfCycleCount{\parSynCounts{2}}

\ReadFile{\stdSources}{build/stdSources}
\ReadFile{\metaSources}{build/metaSources}
\ReadFile{\textSources}{build/textSources}
\ReadFile{\paperSources}{build/paperSources}

\def\srcCount{\the\numexpr\stdSources{3} + \metaSources{3} + \textSources{3} + \paperSources{3}}

\ReadFile{\stdSmntcDiscBrkdwn}{build/stdSmntcDiscBrkdwn}
\ReadFile{\metaSmntcDiscBrkdwn}{build/metaSmntcDiscBrkdwn}
\ReadFile{\textSmntcDiscBrkdwn}{build/textSmntcDiscBrkdwn}
\ReadFile{\paperSmntcDiscBrkdwn}{build/paperSmntcDiscBrkdwn}
\ReadFile{\totalSmntcDiscBrkdwn}{build/totalSmntcDiscBrkdwn}

\ReadFile{\stdSntxDiscBrkdwn}{build/stdSntxDiscBrkdwn}
\ReadFile{\metaSntxDiscBrkdwn}{build/metaSntxDiscBrkdwn}
\ReadFile{\textSntxDiscBrkdwn}{build/textSntxDiscBrkdwn}
\ReadFile{\paperSntxDiscBrkdwn}{build/paperSntxDiscBrkdwn}
\ReadFile{\totalSntxDiscBrkdwn}{build/totalSntxDiscBrkdwn}

\def\stds{\nameref{stds}}
\def\metas{\nameref{metas}}
\def\texts{\nameref{texts}}
\def\papers{\nameref{papers}}

\def\srcCat{\hyperref[sources]{Source Tier}}
\def\reduns{\ifnotpaper\nameref{redun}\else Redundancies\footnote{Section omitted for brevity.}\fi}

\def\totalDiscreps{\totalSmntcDiscBrkdwn{13}}

\def\cats{\hyperref[cats]{Categories}}
\def\syns{\hyperref[syns]{Synonyms}}
\def\pars{\hyperref[pars]{Parents}}
\def\defs{\hyperref[defs]{Definitions}}
\def\terms{\hyperref[terms]{Terminology}}
\def\cites{\hyperref[cites]{Citations}}

%------------------------------------------------------------------------------
% TODOs
%------------------------------------------------------------------------------

% Generic Inlined TODOs
\newcommand{\intodo}[1]{\todo[inline]{#1}}

% Unimportant TODOs for "later" (i.e., finishing touches or changes immediately before submission)
\newcommand{\latertodo}[1]{\todo[backgroundcolor=Cyan]{\textit{Later}: #1}}

% "Important" TODOs
\newcommand{\imptodo}[1]{\todo[inline,backgroundcolor=Red]{\textbf{Important}: #1}}

% "Easy" TODOs
\newcommand{\easytodo}[1]{\todo[inline,backgroundcolor=SeaGreen]{\textit{Easy}: #1}}
\newcommand{\eztodo}[1]{\easytodo{#1}}

% "Tedious" TODOs
\newcommand{\tedioustodo}[1]{\todo[inline,backgroundcolor=PineGreen]{\textit{Needs time}: #1}}

% "Question" TODO Notes
\newcounter{todonoteQuestionsCtr}
\newcommand{\questiontodo}[1]{\stepcounter{todonoteQuestionsCtr}\todo[backgroundcolor=Lavender]{\textbf{Q \#\thetodonoteQuestionsCtr{}}: #1}}
\newcommand{\qtodo}[1]{\questiontodo{#1}}

% Specific categories of TODOs
\def\ptq{\todo{Present tense?}}

%------------------------------------------------------------------------------
% Citations
%------------------------------------------------------------------------------

\newcommand{\exhInfCite}{(\citealp[p.~5-5]{SWEBOK2024}; \citealp[p.~4]{IEEE2022};
    \citealp[p.~421]{vanVliet2000}; \citealp[pp.~439, 461]{PetersAndPedrycz2000})}

%------------------------------------------------------------------------------
% Link to Drasil issue
%------------------------------------------------------------------------------

\newcommand{\issueref}[1]{\href{https://github.com/JacquesCarette/Drasil/issues/#1}{\##1}}
\newcommand{\pullref}[1]{\href{https://github.com/JacquesCarette/Drasil/pull/#1}{\##1}}
\newcommand{\thesisissuerefhelper}[1]{\href{https://github.com/samm82/TestGen-Thesis/issues/#1}{\##1}}

\ExplSyntaxOn

% Based on output from ChatGPT
\NewDocumentCommand{\mapthesisissueref}{m}
{
    % Clear temporary sequences to store transformed items
    \seq_clear:N \l_tmpa_seq
    \seq_clear:N \l_tmpb_seq

    \seq_set_split:Nnn \l_tmpa_seq { , } { #1 } % Split the input by commas
    \seq_map_inline:Nn \l_tmpa_seq
    {
        \seq_put_right:Nn \l_tmpb_seq {\thesisissuerefhelper{##1}}
    }
    \seq_use:Nnnn \l_tmpb_seq { ~and~ } { ,~ } { ,~and~ }
}

\ExplSyntaxOff

\newcommand{\thesisissueref}[1]{\todo[backgroundcolor=lightgray]{See \mapthesisissueref{#1}}}

%------------------------------------------------------------------------------
% Code
%------------------------------------------------------------------------------

% Command based on: https://tex.stackexchange.com/questions/266811/define-a-new-command-with-parameters-inside-newcommand
\newcommand{\codeName}[1]{\expandafter\newcommand\csname #1\endcsname{\inlineHs{#1}}}

% Used for showing what the blue-highlighted text is, in the reading notes section
\codeName{ExampleText}

% Defines commands to be used in poster and thesis

\newcommand{\swebokScalDef}{This seems to define ``usability
    testing'' with elements of functional and recovery testing}
\newcommand{\swebokElasRef}{only cites a single source
    \textbf{that doesn't contain the words ``elasticity'' or ``elastic''}!}

% for assets/code/example.tex...
\newcommand{\exampleCode}{\input{assets/code/example}}
\newcommand{\refExampleCode}{\Cref{lst:exampleCode}}

% for assets/code/examplePseudocode.tex...
\newcommand{\examplePseudocode}{\begin{pseudocode}{haskell}{Broken QuantityDict Chunk Retriever}{examplePseudocode}
retrieveQD :: UID -> ChunkDB -> Maybe QuantityDict
retrieveQD u cdb = do
    (Chunk expectedQd) <- lookup u cdb
    pure expectedQd
\end{pseudocode}
}
\newcommand{\refExamplePseudocode}{\Cref{lst:examplePseudocode}}

% for assets/code/mainInvalidInputTest.tex...
\newcommand{\mainInvalidInputTest}{\begin{codeSnippet}{python}{Tests for main with an invalid input file}{mainInvalidInputTest}{https://github.com/samm82/Drasil/blob/sysTests/code/stable/projectile/projectile_c_p_nol_b_u_v_d/src/python/test/Control_test.py\#L29-L53}
  # from https://stackoverflow.com/questions/54071312/how-to-pass-command-line-argument-from-pytest-to-code
  ## \brief Tests main with invalid input file
  # \par Types of Testing:
  # Dynamic Black-Box (Behavioural) Testing
  # Boundary Conditions
  # Default, Empty, Blank, Null, Zero, and None
  # Invalid, Wrong, Incorrect, and Garbage Data
  # Logic Flow Testing
  @mark.parametrize("filename", invalid_value_input_files)
  @mark.xfail
  def test_main_invalid(monkeypatch, filename):
      # from https://stackoverflow.com/questions/10840533/most-pythonic-way-to-delete-a-file-which-may-not-exist
      try:
          remove(output_filename)
      except OSError as e: # this would be "except OSError, e:" before Python 2.6
          if e.errno != ENOENT: # no such file or directory
              raise # re-raise exception if a different error occurred


      assert not path.exists(output_filename)


      with monkeypatch.context() as m:
          m.setattr(sys, 'argv', ['Control.py', str(Path("test/test_input") / f"{filename}.txt")])
          Control.main()
      
      assert not path.exists(output_filename)
\end{codeSnippet}
}
\newcommand{\refMainInvalidInputTest}{\Cref{lst:mainInvalidInputTest}}

% for assets/code/projManualViolationReq.tex...
\newcommand{\projManualViolationReq}{\begin{codeSnippet}{haskell}{Projectile's manually created input verification requirement}{projManualViolationReq}{https://github.com/JacquesCarette/Drasil/blob/afb6fb752b8364d2807ced7fc0c1dd6c6aba52b2/code/drasil-example/projectile/lib/Drasil/Projectile/Requirements.hs\#L31-L34}
verifyParamsDesc = foldlSent [S "Check the entered", plural inValue,
    S "to ensure that they do not exceed the" +:+. namedRef (datCon [] []) (plural datumConstraint),
    S "If any of the", plural inValue, S "are out of bounds" `sC`
    S "an", phrase errMsg, S "is displayed" `S.andThe` plural calculation, S "stop"]
\end{codeSnippet}
}
\newcommand{\refProjManualViolationReq}{\Cref{lst:projManualViolationReq}}

% for assets/code/projViolationChoice.tex...
\newcommand{\projViolationChoice}{\begin{codeSnippet}{haskell}{\acs{projectile}'s choice for constraint violation behaviour in generated code}{projViolationChoice}{https://github.com/JacquesCarette/Drasil/blob/afb6fb752b8364d2807ced7fc0c1dd6c6aba52b2/code/drasil-example/projectile/lib/Drasil/Projectile/Choices.hs\#L120}
    srsConstraints = makeConstraints Warning Warning,
\end{codeSnippet}
}
\newcommand{\refProjViolationChoice}{\Cref{lst:projViolationChoice}}

%------------------------------------------------------------------------------
% Graphs
%------------------------------------------------------------------------------

% Organization of files

% \newcommand{\parChdGraphs}{
%     % Only top or bottom to comply with IEEE guidelines
%     \begin{figure}[tb!]
%         \centering
%         \begin{subfigure}[b]{\linewidth}
%             \centering
%             \includegraphics[width=\linewidth]{assets/graphs/specBasedGraph.pdf}
%             \caption{``Superset'' relations.}
%             \label{fig:specBasedGraph}
%         \end{subfigure}
%         \begin{subfigure}[t]{.45\linewidth}
%             \centering
%             \includegraphics[width=\linewidth]{assets/graphs/parChdLegend.pdf}
%         \end{subfigure}
%         \begin{subfigure}[t]{.5\linewidth}
%             \centering
%             \includegraphics[width=\linewidth]{assets/graphs/subsumesGraph.pdf}
%             \caption{``Subsume'' relations.}
%             \label{fig:subsumesGraph}
%         \end{subfigure}
%         \caption{Visualizations of different classes of parent-child relations.}
%         \label{fig:parChdGraphs}
%     \end{figure}
% }

\newcommand{\ExampleParChdGraphs}{
    % Only top or bottom to comply with IEEE guidelines
    \begin{figure*}[tb!]
        \begin{subfigure}[b]{0.4\linewidth}
            \includegraphics[width=0.9\linewidth]{assets/graphs/ExampleGlossaryGraph.pdf}
            \vspace{-8mm}
            \caption{Visualization from \\ \Cref{tab:exampleGlossary}.}
            \label{fig:exampleGraph}
        \end{subfigure}
        \centering
        \begin{subfigure}[b]{0.575\linewidth}
            \centering
            \includegraphics[width=\linewidth]{assets/graphs/manual/manualParChdLegend.pdf}
            % \vspace{-4mm}  % Not working for some reason
            \includegraphics[width=0.8\linewidth]{assets/graphs/expExampleGlossaryGraph.pdf}
            \vspace{-8mm}
            \caption{Explicit visualization from \\ \Cref{tab:exampleGlossary}.}
            \label{fig:expExampleGraph}
        \end{subfigure}
        \vspace{2mm}
        \caption{Example generated visualizations of parent-child relations.}
        \label{fig:exampleParChdGraphs}
    \end{figure*}
}

\newcommand{\ExampleSynGraph}{
    \begin{figure*}
        \begin{subfigure}[c]{0.4\linewidth}
            \centering
            \includegraphics[width=\linewidth]{assets/graphs/SynExampleGlossaryGraph.pdf}
            \caption{Visualization from \Cref{tab:synExampleGlossary}.} %\label{fig:exampleSynGraph}
        \end{subfigure}
        \begin{subfigure}[c]{0.575\linewidth}
            \centering
            \includegraphics[width=\linewidth]{assets/graphs/manual/manualSynLegend.pdf}
        \end{subfigure}
        \caption{Example generated visualizations of synonym relations.}\label{fig:exampleSynGraph}
    \end{figure*}
}

% \begin{subfigure}[t]{0.25\linewidth}
%     \centering
%     \includegraphics[width=1.4\linewidth]{assets/graphs/StaticExampleGlossaryGraph.pdf}
%     \caption{Static graph.}
%     \label{fig:staticExampleGraph}
% \end{subfigure}

\newcommand{\ExampleFlawGraphs}{
    \begin{figure*}
        \begin{subfigure}[t]{0.19\linewidth}
            \centering
            \includegraphics[width=1.5\linewidth]{assets/graphs/SelfExampleGlossaryGraph.pdf}
            \caption{Visualization of a reflexive parent relation.}\label{fig:selfExampleGraph}
        \end{subfigure}
        \hfill
        \begin{subfigure}[t]{0.235\linewidth}
            \centering
            \includegraphics[width=\linewidth]{assets/graphs/ParSynExampleGlossaryGraph.pdf}
            \caption{Visualization of a pair of terms with a parent-child \emph{and}
                synonym relation.}\label{fig:parSynExampleGraph}
        \end{subfigure}
        \hfill
        \begin{subfigure}[t]{0.47\linewidth}
            \centering
            \includegraphics[width=\linewidth]{assets/graphs/manual/manualFlawLegend.pdf}
        \end{subfigure}
        \caption{Example generated visualizations containing flaws.}\label{fig:exampleFlawGraphs}
    \end{figure*}
}

\newcommand{\recoveryGraphs}{
    % Only top or bottom to comply with IEEE guidelines
    \begin{figure}[bt!]
        \centering
        \includegraphics[width=\linewidth]{assets/graphs/recoveryLegend.pdf}
        \begin{subfigure}[b]{.475\linewidth}
            \centering
            \includegraphics[width=\linewidth]{assets/graphs/recoveryGraph.pdf}
            \caption{Visualization of current relations.}
            \label{fig:rec-graph-current}
        \end{subfigure}
        \begin{subfigure}[b]{.475\linewidth}
            \centering
            \includegraphics[width=\linewidth]{assets/graphs/recoveryProposedGraph.pdf}
            \caption{Visualization of proposed relations.}
            \label{fig:rec-graph-proposed}
        \end{subfigure}
        \caption{Visualizations of relations between terms related to recovery testing.}
        \label{fig:recoveryGraphs}
    \end{figure}
}

\newcommand{\scalGraphs}{
    % Only top or bottom to comply with IEEE guidelines
    \begin{figure}[b!]
        \centering
        \includegraphics[width=\linewidth]{assets/graphs/scalabilityLegend.pdf}
        \begin{subfigure}[b]{.475\linewidth}
            \centering
            \includegraphics[width=\linewidth]{assets/graphs/scalabilityGraph.pdf}
            \caption{Visualization of current relations.}
            \label{fig:scal-graph-current}
        \end{subfigure}
        \begin{subfigure}[b]{.475\linewidth}
            \centering
            \includegraphics[width=\linewidth]{assets/graphs/scalabilityProposedGraph.pdf}
            \caption{Visualization of proposed \ifnotpaper \else \\ \fi relations.}
            \label{fig:scal-graph-proposed}
        \end{subfigure}
        \caption{Visualizations of relations between terms related to scalability testing.}
        \label{fig:scalGraphs}
    \end{figure}
}

\newcommand{\performanceGraph}{
    \begin{paperFigure}
        \centering
        \includegraphics[width=\linewidth]{assets/graphs/performanceProposedGraph.pdf}
        \caption{Visualization of proposed relations between terms related to
            performance-related testing.}
        \label{fig:perf-graph}
    \end{paperFigure}
}

%------------------------------------------------------------------------------
% Images & Figures
%------------------------------------------------------------------------------

\newcommand{\drasilLogo}{assets/images/drasil_logo.png}
\newcommand{\drasilLogoImg}{\input{assets/images/drasil_logo}}
\newcommand{\refDrasilLogoImg}{\Cref{fig:drasilLogo}}

%------------------------------------------------------------------------------
% Tables
%------------------------------------------------------------------------------

% Organization of files
\newcommand{\organizationTable}{\begin{longtable}[c]{|>{\raggedright}p{0.3\linewidth}|>{\raggedright\arraybackslash}p{0.54\linewidth}|}
    \caption{Template Organization}
    \label{tab:organization}                                              \\

    \hline

    \rowcolor{McMasterMediumGrey}
    \textbf{File/Folder}     & \textbf{Intended Usage \& Description}
    \\ \hline

    \texttt{thesis.tex} & Focal \LaTeX{} file that collects everything and is
    used to build your thesis/report document.
    \\ \hline

    \texttt{Makefile} & A basic \texttt{Makefile} configuration. See
    \texttt{make help} for a list of helpful commands. \\ \hline

    \texttt{build/} & When you build your \acs{pdf}, this folder is used as the
    working directory of LuaLaTeX. Using this allows us to quickly get rid of
    \LaTeX{} build files that can cause problems when we re-build documents. \\
    \hline

    \texttt{manifest.tex} & Basic options that you should certainly configure
    according to your needs.
    \\ \hline

    \texttt{chapters.tex} & All chapters of your thesis should be included here.
    \\ \hline

    \texttt{chapters/} & Enumeration of the chapters of your thesis. I prefer
    using a two-digit indexing pattern for the prefix of file names so that I
    can quickly open up by chapter number using VS Codium. \\ \hline

    \texttt{assets.tex} & Enumeration of the various kinds of ``assets'' in the
    \texttt{assets/} folder. See the file for examples on how you can write your
    extra utility macros. \\ \hline

    \texttt{assets/} & Enumeration of various kinds of ``assets,'' with
    subdirectories for images and figures, tables, and code snippets. \\ \hline

    \texttt{front.tex} & All front matter of your thesis should be included
    here. \\ \hline

    \texttt{front/} & Enumeration of the front chapters of your thesis. These
    chapters should all be numbered using Roman numerals. \\ \hline

    \texttt{back.tex} & All back matter of your thesis should be included here.
    \\ \hline

    \texttt{back/} & Enumeration of the back matter content.
    \\ \hline

    \texttt{acronyms.tex} & List of acronyms you intend to use in your thesis.
    This uses the ``acro'' \LaTeX{} package.
    \\ \hline

    \texttt{macros.tex} & Helpful macros!
    \\ \hline

    \texttt{unicode\_chars.tex} & At times, you might find issues with unicode
    characters, especially in verbatim environments, where you might need to
    manually define them using other font glyphs.
    \\ \hline

    \texttt{mcmaster\_colours.tex} & Macros for the McMaster colour palette.
    \\ \hline

    \texttt{README.md} & Read it!
    \\ \hline

    \texttt{.gitignore} & List of files in the working directory that should be
    ignored by git.
    \\ \hline

    \texttt{latexmkrc} & Used for setting the timezone for latexmk, but can be
    used for other options.
    \\ \hline
\end{longtable}
}

\newcommand{\ieeeCatsTable}{% Conversion to (longtblr) talltblr assisted by GitHub Copilot

\begin{center}
    \begin{talltblr}[
        note{a} = {Also called ``test phase'' \ifnotpaper (see
                \flawref{level-phase-syns}) \fi or ``test stage'' \ifnotpaper
                (see \flawref{stage-level-syns})\else (see relevant synonym
                flaws in \Cref{syns})\fi.},
        note{b} = {Also called ``test design technique'' \ifnotpaper
                (\citealp[p.~11]{IEEE2022}; \citealpISTQB{})\else
                \cite[p.~11]{IEEE2022}, \cite{ISTQB}\fi.},
        caption={Categories of test approaches given by ISO/IEC and IEEE.},
        label={tab:ieeeCats}
        ]{
        colspec={|X[0.09,c,m]X[0.575,m]X[0.285,m]|},
        width = \linewidth, rowhead = 1, hlines
        }
        \thead{Term}               & \thead{Definition}                           & \thead{Examples} \\
        Test Level\TblrNote{a}     & A stage of testing ``typically associated
        with the achievement of particular objectives and used to treat particular
        risks'', each performed in sequence \ifnotpaper (\citealp[p.~12]{IEEE2022};
        \citeyear[p.~6]{IEEE2021b}) \else \cite[p.~12]{IEEE2022}, \cite[p.~6]{IEEE2021b}
        \fi with their ``own documentation and resources''
        \citeyearpar[p.~469]{IEEE2017} % ; more generally, ``designat[es] \dots\ the
        % coverage and detail'' \citeyearpar[p.~249]{IEEE2017} 
                                   & unit/component testing, integration testing,
        system testing, acceptance testing \ifnotpaper (\citeyear[p.~12]{IEEE2022};
        \citeyear[p.~6]{IEEE2021b}; \citeyear[p.~467]{IEEE2017}) \else
        \cite[p.~467]{IEEE2017}, \cite[p.~12]{IEEE2022}, \cite[p.~6]{IEEE2021b} \fi                  \\
        Test Practice              & A ``conceptual framework that can be
        applied to \dots{} [a] test process to facilitate testing'' \ifnotpaper
        (\citeyear[p.~14]{IEEE2022}; \citeyear[p.~471]{IEEE2017}; OG IEEE 2013)
        \else \cite[p.~471]{IEEE2017}, \cite[p.~14]{IEEE2022}
        \fi % ; more generally, a ``specific type of activity that contributes to
        % the execution of a process'' \citeyearpar[p.~331]{IEEE2017} 
                                   & scripted testing, exploratory testing,
        automated testing \citeyearpar[p.~20]{IEEE2022}                                              \\
        Test Technique\TblrNote{b} & A ``procedure used to create or select a
        test model, identify test coverage items, and derive corresponding test
        cases'' \ifnotpaper (\citeyear[p.~11]{IEEE2022}; similar in
        \citeyear[p.~467]{IEEE2017}) \else \cite[p.~11]{IEEE2022} (similar in
        \cite[p.~467]{IEEE2017}) \fi that ``generate evidence that test item
        requirements have been met or that defects are present in a test item''
        \citeyearpar[p.~vii]{IEEE2021b} % ; ``a variety \dots\ is typically
        % required to suitably cover any system'' \citeyearpar[p.~33]{IEEE2022} and
        % is ``often selected based on team skills and familiarity, on the format
        % of the test basis'', and on expectations \citeyearpar[p.~23]{IEEE2022}
                                   & equivalence partitioning,
        boundary value analysis, branch testing \citeyearpar[p.~11]{IEEE2022}                        \\
        Test Type                  & ``Testing that is focused on specific
        quality characteristics'' \ifnotpaper (\citeyear[p.~15]{IEEE2022};
        \citeyear[p.~7]{IEEE2021b}; \citeyear[p.~473]{IEEE2017}; OG IEEE 2013)
        \else \cite[p.~473]{IEEE2017}, \cite[p.~15]{IEEE2022}, \cite[p.~7]{IEEE2021b}
        \fi                        & security testing, usability testing,
        performance testing \ifnotpaper (\citeyear[p.~15]{IEEE2022};
        \citeyear[p.~473]{IEEE2017}) \else \cite[p.~473]{IEEE2017},
        \cite[p.~15]{IEEE2022} \fi                                                                   \\
    \end{talltblr}
\end{center}
}
\newcommand{\otherCatsTable}{% Defined here so VS Code doesn't freak out
\def\ieeeEquiv{\makecell{IEEE\\Equivalent}}
\def\swebokLevel{{Level\\(objective-\\based)\TblrNote{a}}}

\begin{longtblr}[
    note{a} = {See \discrepref{stage-level-syns}.},
    note{b} = {Testing methods and guidances are omitted from this table
            since \citet{BarbosaEtAl2006} do not define or give examples of them.},
    note{c} = {Synonyms for these examples are used by
            \citet[p.~3; OG Mathur, 2012]{SouzaEtAl2017} and
            \citet[p.~3]{BarbosaEtAl2006}.},
    caption={Categories of Testing Given by Other Sources.},
    label={tab:otherCats}
    ]{
    colspec={|X[0.08,c,m]|X[0.43,m]|X[0.34,m]|Q[c,m]|},
    width = \linewidth, rowhead = 1
    }
    \hline
    \thead{Term}                           & \thead{Definition}           & \thead{Examples} & \thead{\ieeeEquiv{}} \\
    \hline
    % Guidance                               & none given
    % \citep[p.~3]{BarbosaEtAl2006}          & none given         & Technique?                              \\
    \swebokLevel{}                         & Test levels based on the
    purpose of testing \citep[p.~5\=/6]{SWEBOK2024} that ``determine
    how the test suite is identified \dots\ regarding its consistency
    \dots\ and its composition''
    \citetext{p.~5\=/2}                    & conformance testing,
    installation testing, regression testing, performance testing,
    security testing % reliability testing,
    \citep[pp.~5\=/7 to 5\=/9]{SWEBOK2024} & Type?                                                                  \\
    % Method                                 & none given
    % \citep[p.~3]{BarbosaEtAl2006}          & none given         & Practice?                               \\
    Phase                                  & none given
    %(\citealp[p.~221]{Perry2006}; \citealp[p.~3]{BarbosaEtAl2006})  
                                           & unit testing,
    integration testing, system testing, regression testing (\citealp[p.~221]{Perry2006};
    \citealp[p.~3]{BarbosaEtAl2006})       & Level                                                                  \\
    Procedure                              & The basis for how
    testing is performed that guides the process; ``categorized in[to] testing methods,
    testing guidances\TblrNote{b} and testing techniques''
    \citep[p.~3]{BarbosaEtAl2006}          & none given
    generally; see ``Technique''           & Approach                                                               \\
    Process                                & ``A sequence of
    testing steps'' \citep[p.~2]{BarbosaEtAl2006} ``based on a development technology and \dots\
    paradigm, as well as on a testing procedure''
    \citetext{p.~3}                        & none given                   & Practice                                \\
    Stage                                  & An
    alternative to the ``traditional \dots\ test stages'' %\footnote{See ``Level'' in \Cref{tab:ieeeCats}.}
    based on ``clear technical groupings''
    \citep[p.~13]{Gerrard2000a}            & desktop development testing,
    infrastructure testing,
    % system testing, large scale integration, and
    post-deployment monitoring
    \citep[p.~13]{Gerrard2000a}            & Level                                                                  \\
    Technique                              & ``Systematic
    procedures and approaches for generating or selecting the most suitable test suites''
    \citep[p.~5\=/10]{SWEBOK2024}          & specification-based testing,
    % ``on a sound theoretical basis'' \citep[p.~3]{BarbosaEtAl2006}
    structure-based testing, fault-based testing\TblrNote{c}
    % , experience-based testing, usage-based testing
    (\citealp[pp.~5\=/10, 5\=/13 to 5\=/15]{SWEBOK2024})
    % black-box, white-box, defect/fault-based, model-based testing
    % \citetext{\citealp[p.~3]{SouzaEtAl2017}; OG Mathur, 2012};
    % functional, structural, error-based, state-based testing \citep[p.~3]{BarbosaEtAl2006}
                                           & Technique                                                              \\
    \hline
\end{longtblr}
}
\newcommand{\otherCategorizationsTable}{\def\selecExs{Deterministic Testing\\ Random Testing}
\def\covCritExs{Input Space Partitioning\\ Graph Coverage\\ Logic Coverage\\ Syntax-based Testing}
\def\execExs{Static Testing\\ Dynamic Testing}
\def\goalExs{Verification Testing\\ Validation Testing}
\def\propExs{Functional Testing\\ Non-functional Testing}
\def\humInvExs{Manual Testing\\ Automated Testing}
\def\strExs{Scripted Testing\\ Exploratory Testing}
\def\covReqExs{Data Flow Testing\\ Control Flow Testing}

\begin{paperTable}
    \centering
    \begin{minipage}{\linewidth}
        \begin{longtblr}[
            note{\textrm{a}} = {See \Cref{par-chd-rels}.},
            note{\textrm{b}} = {We also consider this categorization meaningful (see \Cref{static-test}).},
            note{\textrm{c}} = {Functional testing is categorized ambiguously (see \Cref{func-test-discrep}) and non-functional testing is uncategorized.},
            note{\textrm{d}} = {May instead be a subset of the ``technique'' category \citep[implied by][p.~35; see \Cref{tab:multiCats}]{IEEE2022}.},
            note{\textrm{e}} = {Exploratory testing may instead be a ``technique'' (see \Cref{tab:multiCats}).},
            caption = {Alternate categorizations given by the literature.},
            label = {tab:otherCategorizations}
            ]{
            colspec = {|X[0.35,c,m]X[0.2,c,m]X[0.35,c,m]|}, width = \linewidth,
            rowhead = 1
            }
            \hline
            \thead{Test Basis}                                        & \thead{Example Approaches} & \thead{Parent\MidTblrNote{\textrm{a}} IEEE Category}                                                                                   \\
            \hline
            Selection Process \citep[p.~5-16]{SWEBOK2024}             & \selecExs{}                & Technique \citep[pp.~5-12, 5-16]{SWEBOK2024}                                                                                           \\
            \hline
            Coverage Criteria \citep[pp.~18--19]{AmmannAndOffutt2017} & \covCritExs{}              & Technique (\citealp[p.~22]{IEEE2022}; \citeyear[Fig.~2]{IEEE2021}; \citealp[p.~5-11]{SWEBOK2024}; \citealp[pp.~47--48]{Firesmith2015}) \\
            \hline
            Execution of Code\MidTblrNote{\textrm{b}} (\citealp[p.~214]{KuļešovsEtAl2013}; \citealp[p.~12]{Gerrard2000a};
            \citealp[p.~53]{Patton2006})                              & \execExs{}                 & Approach                                                                                                                               \\
            \hline
            Goal of Testing (\citealp[p.~214]{KuļešovsEtAl2013};
            \citealp[pp.~69--70]{Perry2006})                          & \goalExs{}                 & Approach                                                                                                                               \\
            \hline
            Property of Code \citep[p.~213]{KuļešovsEtAl2013}
            or Test Target \citep[pp.~4--5]{Kam2008}                  & \propExs{}                 & Approach\TblrNote{\textrm{c}}                                                                                                          \\
            \hline
            Human Involvement \citep[p.~214]{KuļešovsEtAl2013}        & \humInvExs{}               & Practice\MidTblrNote{\textrm{d}} \citep[p.~22]{IEEE2022}                                                                               \\
            \hline
            Structuredness \citep[p.~214]{KuļešovsEtAl2013}           & \strExs{}                  & Practice\MidTblrNote{\textrm{e}} \citep[pp.~20, 22]{IEEE2022}                                                                          \\
            \hline
            Coverage Requirement \citep[pp.~4--5]{Kam2008}            & \covReqExs{}               & Technique \citep[p.~5\=/13]{SWEBOK2024}                                                                                                \\
            \hline
        \end{longtblr}
    \end{minipage}
\end{paperTable}
}

\newcommand{\flawMnfstsTable}{\begin{paperTable}
    \centering
    \caption{Breakdown of identified flaws by manifestation and source tier.}\label{tab:flawMnfsts}
    % \begin{minipage}{\linewidth}
    \begin{tabular}{|r|*{6}{cc|}c|}
        \hline
                           & \multicolumn{2}{c|}{\thead{\wrong{}}} & \multicolumn{2}{c|}{\thead{\miss{}}} & \multicolumn{2}{c|}{\thead{\contra{}}} & \multicolumn{2}{c|}{\thead{\ambi{}}} & \multicolumn{2}{c|}{\thead{\over{}}} & \multicolumn{2}{c|}{\thead{\redun{}}} &                                                                                                                                                                                                \\
        \thead{\srcTier{}} & \thead{Obj}                           & \thead{Sub}                          & \thead{Obj}                            & \thead{Sub}                          & \thead{Obj}                          & \thead{Sub}                           & \thead{Obj}              & \thead{Sub}              & \thead{Obj}              & \thead{Sub}               & \thead{Obj}               & \thead{Sub}               & \thead{Total}             \\
        \hline
        \stds{}            & \stdFlawMnfstBrkdwn{1}                & \stdFlawMnfstBrkdwn{2}               & \stdFlawMnfstBrkdwn{3}                 & \stdFlawMnfstBrkdwn{4}               & \stdFlawMnfstBrkdwn{5}               & \stdFlawMnfstBrkdwn{6}                & \stdFlawMnfstBrkdwn{7}   & \stdFlawMnfstBrkdwn{8}   & \stdFlawMnfstBrkdwn{9}   & \stdFlawMnfstBrkdwn{10}   & \stdFlawMnfstBrkdwn{11}   & \stdFlawMnfstBrkdwn{12}   & \stdFlawMnfstBrkdwn{13}   \\
        \metas{}           & \metaFlawMnfstBrkdwn{1}               & \metaFlawMnfstBrkdwn{2}              & \metaFlawMnfstBrkdwn{3}                & \metaFlawMnfstBrkdwn{4}              & \metaFlawMnfstBrkdwn{5}              & \metaFlawMnfstBrkdwn{6}               & \metaFlawMnfstBrkdwn{7}  & \metaFlawMnfstBrkdwn{8}  & \metaFlawMnfstBrkdwn{9}  & \metaFlawMnfstBrkdwn{10}  & \metaFlawMnfstBrkdwn{11}  & \metaFlawMnfstBrkdwn{12}  & \metaFlawMnfstBrkdwn{13}  \\
        \texts{}           & \textFlawMnfstBrkdwn{1}               & \textFlawMnfstBrkdwn{2}              & \textFlawMnfstBrkdwn{3}                & \textFlawMnfstBrkdwn{4}              & \textFlawMnfstBrkdwn{5}              & \textFlawMnfstBrkdwn{6}               & \textFlawMnfstBrkdwn{7}  & \textFlawMnfstBrkdwn{8}  & \textFlawMnfstBrkdwn{9}  & \textFlawMnfstBrkdwn{10}  & \textFlawMnfstBrkdwn{11}  & \textFlawMnfstBrkdwn{12}  & \textFlawMnfstBrkdwn{13}  \\
        \papers*{}         & \paperFlawMnfstBrkdwn{1}              & \paperFlawMnfstBrkdwn{2}             & \paperFlawMnfstBrkdwn{3}               & \paperFlawMnfstBrkdwn{4}             & \paperFlawMnfstBrkdwn{5}             & \paperFlawMnfstBrkdwn{6}              & \paperFlawMnfstBrkdwn{7} & \paperFlawMnfstBrkdwn{8} & \paperFlawMnfstBrkdwn{9} & \paperFlawMnfstBrkdwn{10} & \paperFlawMnfstBrkdwn{11} & \paperFlawMnfstBrkdwn{12} & \paperFlawMnfstBrkdwn{13} \\
        \hline
        Total              & \totalFlawMnfstBrkdwn{1}              & \totalFlawMnfstBrkdwn{2}             & \totalFlawMnfstBrkdwn{3}               & \totalFlawMnfstBrkdwn{4}             & \totalFlawMnfstBrkdwn{5}             & \totalFlawMnfstBrkdwn{6}              & \totalFlawMnfstBrkdwn{7} & \totalFlawMnfstBrkdwn{8} & \totalFlawMnfstBrkdwn{9} & \totalFlawMnfstBrkdwn{10} & \totalFlawMnfstBrkdwn{11} & \totalFlawMnfstBrkdwn{12} & \totalFlawMnfstBrkdwn{13} \\
        \hline
    \end{tabular}
    % \end{minipage}
\end{paperTable}

% TODO: "Trace." abbreviation is done manually for thesis to keep it on the page; should we explain this?
}
\newcommand{\flawDmnsTable}{\begin{paperTable}
    \centering
    \caption{Breakdown of \ifnotpaper identified \fi flaws by domain (defined in \Cref{dmn-def})
        by source tier (defined in \Cref{source-tiers}).}\label{tab:flawDmns}
    % \begin{minipage}{\linewidth}
    \begin{tabular}{|r|*{7}{cc|}c|}
        \hline
                           & \multicolumn{2}{c|}{\thead{\cats{}}} & \multicolumn{2}{c|}{\thead{\syns{}}} & \multicolumn{2}{c|}{\thead{\pars{}}} & \multicolumn{2}{c|}{\thead{\defs{}}} & \multicolumn{2}{c|}{\thead{\labels{}}} & \multicolumn{2}{c|}{\thead{\scope{}}} & \multicolumn{2}{c|}{\thead{\ifnotpaper Trace. \else \trace{} \fi}} &                                                                                                                                                                                                             \\
        % \cline{2-10}
        \thead{\srcTier{}} & \thead{Obj}                          & \thead{Sub}                          & \thead{Obj}                          & \thead{Sub}                          & \thead{Obj}                            & \thead{Sub}                           & \thead{Obj}                                                        & \thead{Sub}            & \thead{Obj}            & \thead{Sub}             & \thead{Obj}             & \thead{Sub}             & \thead{Obj}             & \thead{Sub}             & \thead{Total}           \\
        \hline
        \stds{}            & \stdFlawDmnBrkdwn{1}                 & \stdFlawDmnBrkdwn{2}                 & \stdFlawDmnBrkdwn{3}                 & \stdFlawDmnBrkdwn{4}                 & \stdFlawDmnBrkdwn{5}                   & \stdFlawDmnBrkdwn{6}                  & \stdFlawDmnBrkdwn{7}                                               & \stdFlawDmnBrkdwn{8}   & \stdFlawDmnBrkdwn{9}   & \stdFlawDmnBrkdwn{10}   & \stdFlawDmnBrkdwn{11}   & \stdFlawDmnBrkdwn{12}   & \stdFlawDmnBrkdwn{13}   & \stdFlawDmnBrkdwn{14}   & \stdFlawDmnBrkdwn{15}   \\
        \metas{}           & \metaFlawDmnBrkdwn{1}                & \metaFlawDmnBrkdwn{2}                & \metaFlawDmnBrkdwn{3}                & \metaFlawDmnBrkdwn{4}                & \metaFlawDmnBrkdwn{5}                  & \metaFlawDmnBrkdwn{6}                 & \metaFlawDmnBrkdwn{7}                                              & \metaFlawDmnBrkdwn{8}  & \metaFlawDmnBrkdwn{9}  & \metaFlawDmnBrkdwn{10}  & \metaFlawDmnBrkdwn{11}  & \metaFlawDmnBrkdwn{12}  & \metaFlawDmnBrkdwn{13}  & \metaFlawDmnBrkdwn{14}  & \metaFlawDmnBrkdwn{15}  \\
        \texts{}           & \textFlawDmnBrkdwn{1}                & \textFlawDmnBrkdwn{2}                & \textFlawDmnBrkdwn{3}                & \textFlawDmnBrkdwn{4}                & \textFlawDmnBrkdwn{5}                  & \textFlawDmnBrkdwn{6}                 & \textFlawDmnBrkdwn{7}                                              & \textFlawDmnBrkdwn{8}  & \textFlawDmnBrkdwn{9}  & \textFlawDmnBrkdwn{10}  & \textFlawDmnBrkdwn{11}  & \textFlawDmnBrkdwn{12}  & \textFlawDmnBrkdwn{13}  & \textFlawDmnBrkdwn{14}  & \textFlawDmnBrkdwn{15}  \\
        \papers*{}         & \paperFlawDmnBrkdwn{1}               & \paperFlawDmnBrkdwn{2}               & \paperFlawDmnBrkdwn{3}               & \paperFlawDmnBrkdwn{4}               & \paperFlawDmnBrkdwn{5}                 & \paperFlawDmnBrkdwn{6}                & \paperFlawDmnBrkdwn{7}                                             & \paperFlawDmnBrkdwn{8} & \paperFlawDmnBrkdwn{9} & \paperFlawDmnBrkdwn{10} & \paperFlawDmnBrkdwn{11} & \paperFlawDmnBrkdwn{12} & \paperFlawDmnBrkdwn{13} & \paperFlawDmnBrkdwn{14} & \paperFlawDmnBrkdwn{15} \\
        \hline
        Total              & \totalFlawDmnBrkdwn{1}               & \totalFlawDmnBrkdwn{2}               & \totalFlawDmnBrkdwn{3}               & \totalFlawDmnBrkdwn{4}               & \totalFlawDmnBrkdwn{5}                 & \totalFlawDmnBrkdwn{6}                & \totalFlawDmnBrkdwn{7}                                             & \totalFlawDmnBrkdwn{8} & \totalFlawDmnBrkdwn{9} & \totalFlawDmnBrkdwn{10} & \totalFlawDmnBrkdwn{11} & \totalFlawDmnBrkdwn{12} & \totalFlawDmnBrkdwn{13} & \totalFlawDmnBrkdwn{14} & \totalFlawDmnBrkdwn{15} \\
        \hline
    \end{tabular}
    % \end{minipage}
\end{paperTable}}

\newcommand{\testReqsTable}{\begin{table}[hbtp!]
    \centering
    \caption{Testing Requirements}
    \label{tab:testReqs}
    % \begin{tabularx}{\textwidth}{|>{\hsize=0.65\hsize}X|>{\hsize=1.35\hsize}X|}
    \begin{tabularx}{\textwidth}{|l|X|l|l|}
        \hline
        \rowcolor{McMasterMediumGrey}
        \textbf{Testing Approach} & \textbf{Requirements} & \textbf{In Drasil?} & \textbf{Addable?} \\
        \hline
        test                      & req                   & in                  & addable           \\
        \hline
    \end{tabularx}
\end{table}}

\notpapertrue

\newenvironment{paperTable}{
    \begingroup
    \renewcommand*{\thefootnote}{\alph{footnote}}
    \begin{table}[hbtp!]
        }{
    \end{table}
    \endgroup
}

\renewcommand\stds{\stdSources{1}}
\renewcommand\metas{\metaSources{1}}
\renewcommand\texts{\textSources{1}}
\renewcommand\papers{\paperSources{1}}

\newcounter{methodCounter}

% Based on https://tex.stackexchange.com/a/182669/192195
\newcommand{\tikzmark}[1]{\tikz[overlay,remember picture] \node[baseline] (#1) {};}

\tikzset{bracestyle/.style={midway, left, xshift=-2ex, align=right, font=\small, draw=none, thin, text=black}}

\newcommand\VerticalBrace[5][]{%
    % #1 = draw options
    % #2 = top mark
    % #3 = bottom mark
    % #4 = amplitude
    % #5 = label
\begin{tikzpicture}[overlay,remember picture]
  \draw[decorate,decoration={brace, amplitude=#4ex}, #1] 
    ([yshift=0.5ex, xshift=-3ex]#3.south west) -- ([yshift=1ex, xshift=-3ex]#2.north west)
        node[bracestyle] {#5};
\end{tikzpicture}
}

\title[Committee Meeting 2]{Second Committee Meeting}
\subtitle{Updated Progress Report}
\author{\thesisAuthorName{}}

% Committee
% \begin{itemize}
%     \item \emph{Supervisor}: Dr.~Jacques Carette
%     \item \emph{Supervisor}: Dr.~Spencer Smith
%     \item Dr.~Ned Nedialkov
%     \item Dr.~Richard Paige
% \end{itemize}

\institute{McMaster University}
\date{Fall 2025}

\AtBeginSection[]
{
  \begin{frame}
    \frametitle{Table of Contents}
    \tableofcontents[currentsection]
  \end{frame}
}

\begin{document}

% From https://tex.stackexchange.com/a/42826/192195
\NewDocumentCommand{\ShowListingForLineNumber}{s O{1.0} m m}{
    \IfBooleanTF{#1}{\vspace{-#2\baselineskip}}{}
    \lstinputlisting[
        style=MyListStyle,
        linerange={#3-#3},
        firstnumber=#3,
    ]{#4}
}

%%%%%%%%%%%%%%%%%%%%%%%%%%%%%%%%%%%%%%%%%%%%%%%%%%%%%%%%%%%%%%%%%%%%%%%%%%%%%%%
%% TITLE PAGE
%%%%%%%%%%%%%%%%%%%%%%%%%%%%%%%%%%%%%%%%%%%%%%%%%%%%%%%%%%%%%%%%%%%%%%%%%%%%%%%
\frame{\titlepage}

%%%%%%%%%%%%%%%%%%%%%%%%%%%%%%%%%%%%%%%%%%%%%%%%%%%%%%%%%%%%%%%%%%%%%%%%%%%%%%%
%% TABLE OF CONTENTS
%%%%%%%%%%%%%%%%%%%%%%%%%%%%%%%%%%%%%%%%%%%%%%%%%%%%%%%%%%%%%%%%%%%%%%%%%%%%%%%

\begin{frame}
    \frametitle{Table of Contents}
    \tableofcontents
\end{frame}

%%%%%%%%%%%%%%%%%%%%%%%%%%%%%%%%%%%%%%%%%%%%%%%%%%%%%%%%%%%%%%%%%%%%%%%%%%%%%%%
%% INTRODUCTION
%%%%%%%%%%%%%%%%%%%%%%%%%%%%%%%%%%%%%%%%%%%%%%%%%%%%%%%%%%%%%%%%%%%%%%%%%%%%%%%
\section{Introduction}

\begin{frame}
    \frametitle{Introduction}
    \framesubtitle{Where Were We?}
    \begin{itemize}
        \item We wanted to generate test cases in \textbf{Drasil}, our software
              artifact generation framework
              \begin{itemize}
                  \item Started writing test cases manually
                  \item\pause We stopped to understand software testing to follow existing standards
                        %   \begin{itemize}
                        %       \item Understand the problem domain
                        %       \item Make use of all areas of the domain
                        %       \item Follow domain standards, including quality and terminology
                        %   \end{itemize}
              \end{itemize}

        \item What happened?
              \begin{itemize}
                  \item The domain of software testing is \emph{much} larger than we expected
                  \item Software testing terminology and standards are \emph{not} standardized
              \end{itemize}
    \end{itemize}
\end{frame}

\begin{frame}{Introduction}
    \framesubtitle{Existing Taxonomies?}
    \begin{itemize}
        \item Existing software testing taxonomies: \hfill Focus on: \hspace{0.25cm}
              \begin{itemize}
                  \item \citet{TebesEtAl2020a} \hfill {\small The Testing Process}
                  \item \citet{SouzaEtAl2017} \hfill {\small Organizing Terminology}
                  \item \citet{Firesmith2015} \hfill {\small Relations between Approaches}
                  \item \citet{UnterkalmsteinerEtAl2014} \hfill {\small Traceability between Stages}
              \end{itemize}
    \end{itemize}
\end{frame}

% \subsection{The Need for Standardized Terminology}

% \subsection{The Lack of Standardized Terminology}

\begin{frame}[c]{Introduction}
    \framesubtitle{Existing Taxonomies?}
    \begin{figure}
        \begin{center}
            \only{\includegraphics[height=0.65\textheight]{assets/images/system testing}
                \caption{\tiny \citep[p.~23]{Firesmith2015}}}<1>
            \only{\includegraphics[height=0.65\textheight]{assets/images/system testing hil}
                \caption{\tiny Adapted from \citep[p.~23]{Firesmith2015}}}<2>
            \only{\includegraphics[height=0.65\textheight]{assets/images/system testing sos}
                \caption{\tiny Adapted from \citep[p.~23]{Firesmith2015}}}<3>
            \only{\includegraphics[height=0.65\textheight]{assets/images/system testing self}
                \caption{\tiny Adapted from \citep[p.~23]{Firesmith2015}}}<4>
            \only{\includegraphics[height=0.65\textheight]{assets/images/system testing int}
                \caption{\tiny Adapted from \citep[p.~23]{Firesmith2015}}}<5>
            \only{\includegraphics[height=0.65\textheight]{assets/images/system testing int 2}
                \begin{columns}
                    \centering
                    \begin{column}{0.4\textwidth}
                        \caption{\tiny Adapted from \citepISTQB{}}
                    \end{column}
                    \begin{column}{0.4\textwidth}
                        \caption{\tiny Adapted from \citep[p.~23]{Firesmith2015}}
                    \end{column}
                \end{columns}
            }<6>
        \end{center}
    \end{figure}
\end{frame}

% \begin{frame}
%     \frametitle{Barriers to Effective Communication}
%     \framesubtitle{``The Problem'' (cont.)}
%     \vspace{-1cm}
%     \begin{columns}[T]
%         \begin{column}{.5\textwidth}
%             \begin{center}
%                 \huge Interorganizational

%                 \normalsize Schools, companies, etc.

%                 \vspace{5mm}

%                 % Based on code frustratingly generated by ChatGPT
%                 \begin{tikzpicture}

%                     % Define coordinates of the triangle's vertices
%                     \coordinate (A) at (0, 0);
%                     \coordinate (B) at (3, 0);
%                     \coordinate (C) at (1.5, 2.6);
%                     \coordinate (D) at (1.5, 0.86667);

%                     % Draw circles at each vertex without labels
%                     \draw[fill=blue!20] (A) circle [radius=0.5];
%                     \draw[fill=blue!20] (B) circle [radius=0.5];
%                     \draw[fill=blue!20] (C) circle [radius=0.5];

%                     % Draw arrows as arcs
%                     \draw[->, thick, shorten <= 20pt] (B) arc (0:48:3cm);
%                     \draw[->, thick, shorten <= 20pt] (C) arc (120:168:3cm);
%                     \draw[->, thick, shorten <= 20pt] (A) arc (240:288:3cm);

%                     % Small diagrams inside each circle
%                     \only<2->{
%                         \foreach \x in {A,B,C} {
%                                 \begin{scope}[shift={(\x)}, scale=0.2] % Scaling down and shifting to position A
%                                     \coordinate (D) at (-1.25, -0.75);
%                                     \coordinate (E) at (1.25, -0.75);
%                                     \coordinate (F) at (0, 1.5);
%                                     \draw[fill=blue!20] (D) circle [radius=0.5];
%                                     \draw[fill=blue!20] (E) circle [radius=0.5];
%                                     \draw[fill=blue!20] (F) circle [radius=0.5];
%                                     \draw[->, shorten <= 4pt] (E) arc (0:45:2.5cm);
%                                     \draw[->, shorten <= 4pt] (F) arc (120:165:2.5cm);
%                                     \draw[->, shorten <= 4pt] (D) arc (240:285:2.5cm);
%                                 \end{scope}
%                             }
%                     }
%                 \end{tikzpicture}
%             \end{center}
%         \end{column}
%         \begin{column}<2->{.5\textwidth}
%             \begin{center}
%                 \huge Intraorganizational \normalsize
%             \end{center}

%             ``Complete testing'' could require the tester to:
%             \begin{itemize}
%                 \item discover every bug,
%                 \item exhaust the time allocated,
%                 \item implement every planned test,
%                 \item \dots{} \tiny \citep[p.~7]{KanerEtAl2011}
%                       \normalsize \hspace{0pt} % For bullet spacing
%             \end{itemize}
%         \end{column}
%     \end{columns}
% \end{frame}

%%%%%%%%%%%%%%%%%%%%%%%%%%%%%%%%%%%%%%%%%%%%%%%%%%%%%%%%%%%%%%%%%%%%%%%%%%%%%%%
%% PROJECT
%%%%%%%%%%%%%%%%%%%%%%%%%%%%%%%%%%%%%%%%%%%%%%%%%%%%%%%%%%%%%%%%%%%%%%%%%%%%%%%

\section{Project}
\subsection{Research Questions}

\def\rqa{\begin{alertblock}{Research Question 1}
        \rqatext{}
    \end{alertblock}
}

\def\rqb{\begin{alertblock}{Research Question 2}
        \rqbtext{}
    \end{alertblock}
}

\def\rqc{\begin{alertblock}{Research Question 3}
        \rqctext{}
    \end{alertblock}
}

\begin{frame}{Research Questions}
    \onslide<1->\rqa{} \vspace*{\fill}
    \onslide<1>\rqb{} \vspace*{\fill}
    \onslide<1>\rqc{} \vspace*{\fill}
    \vspace{0.2cm}
\end{frame}

\subsection{Methodology}

\input{build/methodOverviewSem}

\begin{frame}{Methodology}
    \framesubtitle{Procedure}
    \begin{itemize}
        \item We build a glossary with a row for each test approach
    \end{itemize}
    \begin{center}
        \begin{table}
            \small
            \begin{tabularx}{\linewidth}{|M{1.1cm}|M{1.3cm}|X|M{1.5cm}|M{1.6cm}|}
                \hline
                \thead{\textbf{Name}} & \thead{\textbf{Category}} & \thead{\textbf{Definition}}                                                                                         & \thead{\textbf{Parent(s)}}                       & \thead{\textbf{Synonym(s)}}           \\
                \hline
                A/B Testing           & Practice {\tiny (Fig.~2)} & Testing ``that allows testers to determine which of two systems or components performs better'' {\tiny (pp.~1, 36)} & Statistical Testing {\tiny (pp.~1,~36)}, \dots{} & Split-Run Testing {\tiny (pp.~1,~36)} \\
                \hline
            \end{tabularx}
            \caption{\tiny Information from \citep{IEEE2022}}
        \end{table}
    \end{center}
    \pause \vspace{-0.5cm}
    \begin{itemize}
        \item We gather this information from sources by looking for:
              \begin{itemize}
                  \item Glossaries, taxonomies, hierarchies, etc.
                  \item Testing-related terms
                  \item Terms described \emph{by} other approaches
                  \item Terms that \emph{imply} other approaches
              \end{itemize}
    \end{itemize}
\end{frame}

% \begin{frame}{Methodology}
%     \framesubtitle{Procedure}
%     \begin{itemize}
%         \item It seems that the existence of a software quality implies the
%               existence of a test type associated with it \pause
%         \item Some test approaches use shared or complicated terminology \pause
%         \item For each of these, we record its
%               \begin{itemize}
%                   \item Name
%                   \item Definition
%                   \item Precedence for a related test type (only for qualities)
%                   \item Synonym(s)
%               \end{itemize}
%     \end{itemize}
% \end{frame}

% \begin{frame}{Methodology}
%     \framesubtitle{Procedure}
%     \begin{itemize}
%         \item Recording these data in a consistent format allows for visualizations to
%               be generated according to a certain logic \pause
%         \item It also allows for subsets of flaws to be identified \pause
%               \vspace{1cm}\rqb{}
%     \end{itemize}
% \end{frame}

\begin{frame}{Methodology}
    \framesubtitle{Sources}
    \begin{figure}
        \centering
        \begin{tikzpicture}
            \pie[sum=auto, after number=, text=legend, thick,
                scale=\ifnotpaper0.7\else0.5\fi,
                every label/.style={align=left, scale=0.7}]
            {\stdSources{3}/\stds{},
                \metaSources{3}/\metas{},
                \textSources{3}/\texts{},
                \paperSources{3}/\papers{}}

            \onslide<1>{
                \node[anchor=west, align=center] at (0, 3) {
                    In total, we investigate \srcCount{} sources
                };
            }

            \onslide<2>{
                \node[anchor=west, align=center] at (-2.2, 3) {
                    Textbooks used at McMaster were our ad hoc starting points\\
                    \tiny \citep{Patton2006, PetersAndPedrycz2000, vanVliet2000}
                };
                \draw[->, very thick] (-1.8, 2.9) -- (-1.55, 1.1);
            }

            % \onslide<3>{
            %     \node[anchor=west, align=left] at (2.4, -2) {
            %         Includes websites \citetext{\citealp{LambdaTest2024}; \\
            %             \quad\quad\citealp{Pandey2023}} and a booklet\\
            %         \quad\quad\citep{SPICE2022}};
            %     \draw[->, very thick] (2.25, -1.5) -- (1.25, -1.1);
            % }

        \end{tikzpicture}
    \end{figure}
\end{frame}

\begin{frame}[t]{Methodology}
    \framesubtitle{Categories}
    \vspace{-0.925cm}
    \centering
    \only<1>{\includegraphics[width=\linewidth]{assets/graphs/manual/catRels1.pdf}}
    \only<2>{\includegraphics[width=\linewidth]{assets/graphs/manual/catRels2.pdf}}
    \only<3>{\includegraphics[width=\linewidth]{assets/graphs/manual/catRels3.pdf}}
    \only<4>{\includegraphics[width=\linewidth]{assets/graphs/manual/catRels4.pdf}}
    \only<5>{\includegraphics[width=\linewidth]{assets/graphs/manual/catRels5.pdf}}

    \begin{minipage}{0.98\textwidth}
        % For weird spacing issue
        \only<1-3>{\vspace{-0.8cm}}
        \only<4>{\vspace{-0.475cm}}
        \only<5>{\vspace{0.02cm}}
        \only<1>{\textbf{Approach:} a ``high-level test implementation choice''
            \citep[p.~10]{IEEE2022} used to ``pick the particular test case
            values'' \citeyearpar[p.~465]{IEEE2017}}
        \only<2>{\textbf{Level:} a stage of testing with ``particular
            objectives and \dots{} risks'', each performed in sequence
            (\citealp[p.~12]{IEEE2022}; \citeyear[p.~6]{IEEE2021a};
            \citeyear[p.~6]{IEEE2021c})}
        \only<3>{\textbf{Practice:} a ``conceptual framework that can be applied
            to \dots{} [a] test process to facilitate testing''
            (\citealp[p.~14]{IEEE2022}; \citeyear[p.~471]{IEEE2017})}
        \only<4>{\textbf{Technique:} a
            % ``defined'' and ``systematic'' \citep[p.~464]{IEEE2017}
            ``procedure used to create or select a test
            model, identify test coverage items, and derive corresponding test cases''
            (\citeyear[p.~11]{IEEE2022}; \citeyear[p.~5]{IEEE2021a};
            similar in \citeyear[p.~467]{IEEE2017})}
        \only<5>{\textbf{Type:} ``Testing that is focused on specific quality
            characteristics'' (\citealp[p.~15]{IEEE2022}; \citeyear[p.~7]{IEEE2021c};
            \citeyear[p.~473]{IEEE2017})}
    \end{minipage}
\end{frame}

\begin{frame}[t]{Methodology}
    \framesubtitle{Visualization Notation}
    \vspace{-0.925cm}
    \centering
    \only<1>{\includegraphics[width=\linewidth]{assets/graphs/manual/catRels5.pdf}}
    \only<2>{\includegraphics[width=\linewidth]{assets/graphs/manual/catRels6.pdf}}
    \only<3>{\includegraphics[width=\linewidth]{assets/graphs/manual/catRels7.pdf}}

    \begin{minipage}{0.98\textwidth}
        \centering
        % For spacing issues
        \only<1>{\vspace{1.6cm}}
        \only<2>{\vspace{-2.2cm}}
        \only<3>{\vspace{-1.2cm}}
        \only<1>{Arrows point from a \emph{child} node to a \emph{parent} node.}
        \only<2>{Lines without arrowheads connect \emph{synonyms}.}
        \only<3>{Dashed lines indicate a relationship is \emph{implicit}.}
    \end{minipage}
\end{frame}

\begin{frame}[t]{Methodology}
    \framesubtitle{Visualization Notation}
    \begin{columns}[T]
        \begin{column}{.5\textwidth}
            \vspace{-0.5cm}
            \centering
            \includegraphics[width=\linewidth]{assets/graphs/manual/catRels8.pdf}
        \end{column}
        \begin{column}{.5\textwidth}
            \vspace{0.75cm}
            Dashed outlines indicate a term is \emph{implicit}.\\
            \vspace{1.3cm}
            Dotted outlines indicate a term is a \emph{synonym} to more than one term.
        \end{column}
    \end{columns}
\end{frame}

\begin{frame}{Graph of Test Approaches}
    \pause \large \centering \texttt{! Dimension too large.}
    % \includegraphics[width=0.75\textwidth, height=0.75\textheight,
    %     keepaspectratio]{assets/graphs/approachGraph.pdf}
\end{frame}

\begin{frame}{Graph of Test Levels}
    \includegraphics[width=\textwidth]{assets/graphs/levelGraph.pdf}
\end{frame}

\begin{frame}{Graph of Test Practices}
    \includegraphics[width=\textwidth]{assets/graphs/practiceGraph.pdf}
\end{frame}

\begin{frame}{Graph of Test Techniques}
    \includegraphics[width=\textwidth]{assets/graphs/techniqueGraph.pdf}
\end{frame}

\begin{frame}{Graph of Test Types}
    \includegraphics[width=\textwidth]{assets/graphs/typeGraph.pdf}
\end{frame}

% \begin{frame}[t]{Methodology}
%     \framesubtitle{Static Testing}
%     \begin{figure}
%         \centering
%         \includegraphics[height=0.65\textheight]{assets/images/test approach choices}
%         \caption{\tiny \citep[Fig.~2]{IEEE2022}}
%     \end{figure}
% \end{frame}

% \begin{frame}{Methodology}
%     \framesubtitle{Static Testing}
%     \begin{columns}[c]
%         \begin{column}{.3\textwidth}
%             \begin{figure}
%                 \centering
%                 \includegraphics[width=\linewidth]{assets/images/test approach static testing}
%                 \caption{\tiny Adapted from \citep[Fig.~2]{IEEE2022}}
%             \end{figure}
%         \end{column}
%         \begin{column}{.7\textwidth}
%             \begin{itemize}
%                 \item \citeauthor{IEEE2022} seem to describe ``static testing''
%                       as a separate category
%                 \item \pause Is ``static testing'' part of software testing?
%                       \vspace{-0.25cm}\begin{columns}[t]
%                           \begin{column}{.55\textwidth}
%                               \begin{center}
%                                   \textbf{Yes}
%                               \end{center} \tiny
%                               \hspace{0.75cm}\citep[pp.~16\==17]{IEEE2022} \\
%                               \hspace{0.75cm}\citep[p.~43]{IEEE2021b} \\
%                               \hspace{0.75cm}\citep[p.~5\=/2]{SWEBOK2025} \\
%                               \hspace{0.75cm}\citep[pp.~8\==9]{Gerrard2000a}
%                           \end{column}
%                           \begin{column}{.45\textwidth}
%                               \begin{center}
%                                   \hspace{-0.25cm}\textbf{No}
%                               \end{center} \tiny
%                               \citep[p.~427]{IEEE2017} \\
%                               \citep[pp.~5\=/1]{SWEBOK2025} \\
%                               \citep[p.~13]{Firesmith2015} \\
%                               \citep[p.~439]{PetersAndPedrycz2000} \\
%                               \citep[p.~222]{AmmannAndOffutt2017}
%                           \end{column}
%                       \end{columns}\vspace{0.25cm}
%                 \item \pause We record static test approaches for completeness
%                       % \item \pause Quite distinct but not necessarily orthogonal
%                       % \item \pause When considering static testing in isolation,
%                       %       related \emph{dynamic approaches} have grey backgrounds

%                       %       \vspace{-0.5cm}
%                       %       \includegraphics[width=\linewidth]{assets/graphs/manual/catRels9.pdf}
%             \end{itemize}
%         \end{column}
%     \end{columns}
% \end{frame}

% \begin{frame}{Graph of \emph{Static} Test Approaches}
%     \includegraphics[width=\textwidth]{assets/graphs/staticGraph.pdf}
% \end{frame}

\section{Results}
\begin{frame}{Overview}
    % \onslide<1->\rqa{} \vspace*{\fill}
    % \onslide<1>\rqb{} \vspace*{\fill}
    % \onslide<1>\rqc{} \vspace*{\fill}
    % \onslide<2>
    % \vspace{-4cm}
    \begin{columns}
        \begin{column}{0.45\textwidth}
            \vspace{-1cm}
            \begin{itemize}
                \item \approachCount{} test approaches $\rightarrow$
                \item<2-> \qualityCount{} software qualities \\ \small (may imply test approaches)
                \item<3-> \flawCount{} flaws in the software testing literature
            \end{itemize}
        \end{column}
        \begin{column}{0.55\textwidth}
            \centering
            \begin{tikzpicture}
                \pie[sum=100, text=legend, thick, scale=0.5,
                every label/.style={align=left, scale=0.7}]
                {{\the\numexpr 100 - 100 * \UndefAfter/\TotalAfter}/Defined,
                {\the\numexpr 100 * \UndefAfter/\TotalAfter}/{Not defined}}
            \end{tikzpicture}
        \end{column}
    \end{columns}
\end{frame}

\begin{frame}{Flaw Summary by Source Tier}
    \begin{center}
        \begin{figure}[bt!]
\centering
\begin{tikzpicture}
\begin{axis}[
width=0.8\textwidth, height=7.5cm,
yticklabels={\parbox{0.24\textwidth}{\raggedleft\papers{}},\parbox{0.24\textwidth}{\raggedleft\texts{}},\parbox{0.24\textwidth}{\raggedleft\metas{}},\parbox{0.24\textwidth}{\raggedleft\stds{}}},
ytick=data,
xlabel=\parbox{0.5\textwidth}{\centering Number of Flaws per Source Tier \\ \quad{}}, xbar, xmin=0,
nodes near coords,
every node near coord/.append style={font=\tiny},
]

\addplot[fill=blue!60] coordinates {(\the\numexpr\paperFlawMnfstBrkdwn{13},0) (\the\numexpr\textFlawMnfstBrkdwn{13},1) (\the\numexpr\metaFlawMnfstBrkdwn{13},2) (\the\numexpr\stdFlawMnfstBrkdwn{13},3)};
\end{axis}
\end{tikzpicture}

\end{figure}

    \end{center}
\end{frame}

\begin{frame}{Normalized Flaw Summary}
    \begin{center}
        \input{assets/graphs/normFlawBarsSummarySem}
    \end{center}
\end{frame}

\begin{frame}{Flaw Summary by Manifestation}
    \begin{center}
        \begin{figure}[bt!]
\centering
\begin{tikzpicture}
\begin{axis}[
width=0.8\textwidth, height=0.8\textheight,
symbolic y coords={Redundancies,Overlaps,Ambiguities,Contradictions,Omissions,Mistakes},
ytick=data,
xlabel=Number of Flaws, xbar, xmin=0,
nodes near coords,
every node near coord/.append style={font=\tiny},
]
\addplot[fill=blue!60] coordinates {(8,Redundancies) (16,Overlaps) (35,Ambiguities) (182,Contradictions) (13,Omissions) (54,Mistakes)};
\end{axis}
\end{tikzpicture}
\end{figure}

    \end{center}
\end{frame}

\begin{frame}{Flaw Summary by Domain}
    \begin{center}
        \begin{figure}[bt!]
\centering
\begin{tikzpicture}
\begin{axis}[
width=0.8\textwidth, height=7.5cm,
symbolic y coords={Traceability,Scope,Labels,Definitions,Parents,Synonyms,Categories},
ytick=data,
xlabel=Number of Flaws, xbar, xmin=0,
nodes near coords,
every node near coord/.append style={font=\tiny},
]
\addplot[fill=blue!60] coordinates {(7,Traceability) (5,Scope) (39,Labels) (71,Definitions) (57,Parents) (62,Synonyms) (65,Categories)};
\end{axis}
\end{tikzpicture}
\end{figure}

    \end{center}
\end{frame}

\begin{frame}{Automated Flaws}
    \begin{itemize}
        \item Some terms are given as a synonym to two (or more) disjoint,
              unrelated terms, making the relation between the given synonyms
              ambiguous \pause
        \item These are included in generated visualizations automatically
    \end{itemize}
    \vspace{-0.5cm}
    \begin{columns}[c]
        \begin{column}{.475\linewidth}
            \small
            \begin{table}[hbtp!]
                \centering
                \begin{tabularx}{\textwidth}{c>{\raggedleft\arraybackslash}X} \hline
                    Name & Synonym(s)                                   \\ \hline
                    E    & F (Author, 2022; implied by StdAuthor, 2021) \\
                    G    & F (Author, 2017), H (implied by 2022)        \\
                    H    & X (StdAuthor, 2021)                          \\ \hline
                \end{tabularx}
            \end{table}
        \end{column}
        \begin{column}{.5\linewidth}
            \includegraphics[width=\textwidth]{assets/graphs/SynExampleGlossaryGraph.pdf}
        \end{column}
    \end{columns} %\pause
    \vspace{-0.25cm}
\end{frame}

\begin{frame}{Automated Flaws}
    Prominent examples of these ``multi-synonyms'': \vspace{0.25cm}
    \begin{enumerate}
        % \item \textbf{Invalid Testing:} \hfill \textbf{Source(s)}
        %       \begin{itemize}
        %           \item Error Tolerance Testing {\hfill \tiny \citep[p.~45]{Kam2008}}
        %           \item Negative Testing {\hfill \tiny \citepISTQB{}}
        %       \end{itemize} \pause
        \item \textbf{Soak Testing:} \hfill \textbf{Source(s)}
              \begin{itemize}
                  \item Endurance Testing {\hfill \tiny \citep[p.~39]{IEEE2021c}}
                  \item Reliability Testing {\hfill \tiny (\citealp[Tab.~2]{Gerrard2000a};
                                \citeyear[Tab.~1, p.~26]{Gerrard2000b})}
              \end{itemize} \pause
        \item \textbf{Functional Testing:}
              \begin{itemize}
                  \item Behavioural Testing {\hfill \tiny \citep[p.~45]{Kam2008}}
                  \item Correctness Testing {\hfill \tiny \citep[p.~5\=/7]{SWEBOK2024}}
                  \item Specification-based Testing {\hfill \tiny (\citealp[p.~196]{IEEE2017}; \dots{})}
                        % \citealp[pp.~44\==45]{Kam2008}; \citealp[p.~399]{vanVliet2000})}
                        % implied by \citealp[p.~129]{IEEE2021c}; \citeyear[p.~431]{IEEE2017})}
              \end{itemize} \pause
        \item \textbf{Link Testing:}
              \begin{itemize}
                  \item Branch Testing {\hfill \tiny (implied by \citealp[p.~24]{IEEE2021c})}
                  \item Component Integration Testing {\hfill \tiny \citep[p.~45]{Kam2008}}
                  \item Integration Testing {\hfill \tiny (implied by \citealp[p.~13]{Gerrard2000a})}
              \end{itemize}
    \end{enumerate}
\end{frame}


%   \begin{figure}
%       %   \vspace{-1mm}
%       \includegraphics[width=.8\textwidth]{assets/stable.png}
%       %   \vspace{-3mm}
%       \caption{Contents of \texttt{stable}}
%       \vspace{-1mm}
%   \end{figure}

%   \lstinputlisting[
%       title=An example log,
%       captionpos=b,
%       language={},
%       basicstyle=\tiny, % TODO: reduce font size?
%       breakatwhitespace=true,
%       showstringspaces=false
%   ]{assets/log.txt}

% \onslide<7-|handout:1>\begin{block}{}
%     {"The information you have should be just as useful for generating
%         tests as it should be for manually running them."}
%     \vspace{3mm}
%     \hspace\fill{\small--- Dr.~Jacques Carette}
% \end{block}

%%%%%%%%%%%%%%%%%%%%%%%%%%%%%%%%%%%%%%%%%%%%%%%%%%%%%%%%%%%%%%%%%%%%%%%%%%%%%%%
%% NEXT STEPS
%%%%%%%%%%%%%%%%%%%%%%%%%%%%%%%%%%%%%%%%%%%%%%%%%%%%%%%%%%%%%%%%%%%%%%%%%%%%%%%

\section{Next Steps}
\begin{frame}
    \frametitle{Thesis Chapters}
    \begin{columns}
        \begin{column}{0.2\textwidth}
        \end{column}
        \begin{column}{0.8\textwidth}
            \begin{enumerate}
                \setcounter{enumi}{-1}
                \item\tikzmark{top 1}Abstract
                \item Introduction
                \item Terminology (including relevant appendices)
                \item\tikzmark{bottom 1}Methodology
                \item\tikzmark{top 2}Tools
                \item\tikzmark{bottom 2}Observed Flaws (including relevant appendix)
                \item\tikzmark{top 3}Recommendations
                \item Threats to Validity
                \item\tikzmark{bottom 3}Future Work
                \item\tikzmark{top 4}\tikzmark{bottom 4}Conclusion
            \end{enumerate}

            \VerticalBrace[ultra thick, blue]{top 1}{bottom 1}{1.50}{Complete}
            \VerticalBrace[ultra thick, blue]{top 2}{bottom 2}{1.46}{Under Review}
            \VerticalBrace[ultra thick, blue]{top 3}{bottom 3}{1.48}{In Progress}
            \VerticalBrace[ultra thick, blue]{top 4}{bottom 4}{1.15}{To Do Later}
        \end{column}
    \end{columns}
\end{frame}

\begin{frame}
    \frametitle{Scheduling Next Presentations}
    \begin{table}
        \begin{tabular}{|ll|ll|}
            \hline
            \multicolumn{2}{|c|}{\textbf{Seminar}} & \multicolumn{2}{c|}{\textbf{Defense}}                                       \\ \hline
            Oct.~27                                & 2:00--3:00                            & Nov.~10 & 2:00--5:00                \\
            Oct.~28                                & 2:30--3:30                            & Nov.~11 & 10:00--1:00 or 2:30--4:00 \\
            Oct.~29                                & 1:30--3:30                            & Nov.~17 & 2:00--5:00                \\
            Nov.~3                                 & 2:00--5:00                            & Nov.~18 & 11:00--1:00 or 2:30--4:00 \\
            Nov.~4                                 & 11:00--1:00 or 2:30--4:00             & Nov.~19 & 9:30--10:30 or 1:00--3:30 \\
            Nov.~5                                 & 9:30--11:30 or 1:00--3:30             &         &                           \\ \hline
        \end{tabular}
    \end{table}
\end{frame}

%%%%%%%%%%%%%%%%%%%%%%%%%%%%%%%%%%%%%%%%%%%%%%%%%%%%%%%%%%%%%%%%%%%%%%%%%%%%%%%
%% ACKNOWLEDGEMENT
%%%%%%%%%%%%%%%%%%%%%%%%%%%%%%%%%%%%%%%%%%%%%%%%%%%%%%%%%%%%%%%%%%%%%%%%%%%%%%%

\begin{frame}
    \frametitle{Acknowledgment}

    \begin{itemize}
        \item \supersAck{}
        \item The format of this presentation was \emph{heavily} based on
              a previous presentation by Jason Balaci, who also provided a
              great thesis template
        \item \codeAI*{}
    \end{itemize}
\end{frame}

%%%%%%%%%%%%%%%%%%%%%%%%%%%%%%%%%%%%%%%%%%%%%%%%%%%%%%%%%%%%%%%%%%%%%%%%%%%%%%%
%% REFERENCES
%%%%%%%%%%%%%%%%%%%%%%%%%%%%%%%%%%%%%%%%%%%%%%%%%%%%%%%%%%%%%%%%%%%%%%%%%%%%%%%

% From https://tex.stackexchange.com/a/457255/192195
\setbeamertemplate{page number in head/foot}{}

\begin{frame}[allowframebreaks,noframenumbering]
    \frametitle{References}

    \bibliography{references,seminar_images}
\end{frame}

\end{document}
