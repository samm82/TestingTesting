\chapter{Tools User Guide}\label{tools-guide}

Since we keep the description of our tools abstract in \Cref{tools},
we provide a more in-depth description of our tools as follows.

\section{Flaw Comment Syntax}\label{flaw-comment-syntax}
As described in \Cref{record-flaws}, we include comments alongside the flaws we
document so we can analyze them automatically (as described in
\Cref{flaw-comment-analysis}). These comments have the following format:
\begin{displayquote}
    \texttt{\% Flaw count (MNFST, DMN): \{A1\} \{A2\} \dots{} | \{B1\} % \{B2\}
        \dots{} | \{C1\} \dots}
\end{displayquote}
\texttt{MNFST} and \texttt{DMN} are placeholders for the ``keys'' given in
\Cref{tab:flawMnfstDefs,tab:flawDmnDefs}, respectively, that we use to
track a flaw's manifestation(s) and domain(s) (defined in \Cref{flaw-def}).
For example, the comment line for an incorrect synonym relation would start
with ``\texttt{\% Flaw count (WRONG, SYNS)}'' and one for a redundant label would
start with ``\texttt{\% Flaw count (REDUN, LABELS)}''. We omit these keys from
% constructed examples of these comments without associated flaws throughout / brevity
this chapter for simplicity. Finally, \texttt{A1}, \texttt{A2}, % \texttt{B2},
\texttt{B1}, and \texttt{C1} are each placeholders for a source involved in
this example flaw; in general, there can be arbitrarily many. We represent
each source by its \BibTeX{} key and wrap each one in curly braces (with
the exception of the \acs{istqb} glossary due to its use of custom commands
via \macro{citealias}) to mimic \LaTeX{}'s citation commands for ease of
parsing. We then separate each ``group'' of sources with a pipe symbol
(\texttt{|}) so we can compare each pair of groups; in general, a flaw can
have any number of groups of sources.

As mentioned in \Cref{one-src-flaws}, \oneSrcDistinct{} We track
self-contained flaws by recording the single source that the flaw is present
in such as in the first line below. In constrast, we track internal flaws
by recording the single source in multiple groups (as defined above). The
second line is a standard example of this, while the third is more complex;
in this case, source \texttt{Y} agrees with only one of the conflicting
sources of information in \texttt{X}.
\begin{displayquote}
    \texttt{\% Flaw count: \{X\}\\\% Flaw count: \{X\} | \{X\}\\
        \% Flaw count: \{X\} | \{X\} \{Y\}}
\end{displayquote}
We can also specify the ``explicitness'' (see \Cref{explicitness}) of a
flaw by inserting the phrase ``implied by'' after the sources of explicit
information and before those of implicit information, such as in the
following example:\qtodo{Is this clear enough?}
\begin{displayquote}
    \texttt{\% Flaw count: \{X\} \{Y\} | \{Z\} implied by \{X\}}
\end{displayquote}\phantomsection{}\label{less-cred-assert}%
Occasionally, we assert that a source from a less credible tier is more
correct than a source from a more credible tier\thesisissueref{184}.
% For example, \tolTestFlaw*{} This flaw is supported by these additional
% papers found via a miniature literature review (described in
% \Cref{undef-terms}) from a lower source tier than \citet{Firesmith2015}
% (which is a terminology collection; see \Cref{source-tiers}). However,
% this flaw is really based in \citet{Firesmith2015} and not in these
% additional papers, but this would be counted as a flaw in these papers if
% they were included as detailed above.
If we documented these flaws as above, they would incorrectly be counted as
flaws within the less credible tier! Therefore, we document these
``assertion'' sources separately to track them for traceability without
counting them incorrectly.
% As an example, the
% following is the comment line\utd{} for this specific example
% (\flawref{assert-truth}):
% \begin{displayquote}
%     \texttt{\% Flaw count (WRONG, LABELS): \{Firesmith2015\}\\
%         \% Assertion: \{LiuEtAl2023\} \{MorgunEtAl1999\} \{HolleyEtAl1996\}
%         \displayNL \{HoweAndJohnson1995\}}
% \end{displayquote}
For example, if we assert that a textbook \texttt{W} is correct and
indicates a flaw in established standards \texttt{X} and \texttt{Y}, we
would track this assertion separately from its associated flaw as follows:
\begin{displayquote}
    \texttt{\% Flaw count: \{X\} \{Y\}\\\% Assertion: \{W\}}
\end{displayquote}
