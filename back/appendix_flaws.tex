\chapter{Full Lists of Flaws}\label{flaws-full}

The following are the full lists of manually identified flaws, first grouped
by their manifestation (\Cref{flawMnfsts-full}), then by their domain
(\Cref{flawDmn-full}) as defined in \Cref{flaw-def}. We then present inferred
flaws (\Cref{infer-flaws}) as described in \Cref{infers} for completeness,
although these flaws do not contribute to any counts.
% since they require further analysis to determine if they are flaws at all!

\section{Full Lists of Flaws by Manifestation}\label{flawMnfsts-full}

We sort the following groups of flaws by their source tier (defined in
\Cref{source-tiers}) in descending order of credibility (defined in
\Cref{cred}).

\subsection{Full List of Mistakes}\label{wrong-full}
\input{build/FlawMnfstWrong}

\subsection{Full List of Omissions}\label{miss-full}
\input{build/FlawMnfstMiss}

\subsection{Full List of Contradictions}\label{contra-full}
\input{build/FlawMnfstContra}

\subsection{Full List of Ambiguities}\label{ambi-full}
\input{build/FlawMnfstAmbi}

\subsection{Full List of Overlaps}\label{over-full}
\input{build/FlawMnfstOver}

\subsection{Full List of Redundancies}\label{redun-full}
\input{build/FlawMnfstRedun}

\section{Full Lists of Flaws by Domain}\label{flawDmn-full}

The following sections provide all of the data that we automatically detect
(as described in \Cref{auto-flaw-analysis}) and summarize in \Cref{flawDmns}.

\subsection{Multiple Categorizations}\label{multiCats-full}

As mentioned in \Cref{multiCats}, \multiCatIntro{} and list them in
\Cref{tab:multiCats}.

\begin{landscape}
    \begin{paperTable}
    \centering
    \caption{Test approaches with more than one category.}\label{tab:multiCats}
    \begin{minipage}{\linewidth}
        \centering
        \begin{tabular}{|r|l|l|}
            \hline
            \thead{Approach}         & \thead{Category 1}                                                  & \thead{Category 2}                                                                                                                              \\
            \hline
            Ad Hoc Testing           & Practice \cite[p.~33]{IEEE2013}                                     & Technique \cite[p.~5\=/14]{SWEBOK2024}                                                                                                          \\
            Capacity Testing         & Technique \cite[pp.~38\==39]{IEEE2021c}                             & Type \cite[p.~22]{IEEE2022}, \cite[p.~2]{IEEE2013}                                                                                              \\
            Checklist-based Testing  & Practice \cite[p.~34]{IEEE2022}                                     & Technique \cite{ISTQB}                                                                                                                          \\
            Data-driven Testing      & Practice \cite[p.~22]{IEEE2022}                                     & Technique \cite[p.~43]{Kam2008}                                                                                                                 \\
            End-to-end Testing       & Type \cite{ISTQB}                                                   & Technique \cite[p.~47]{Firesmith2015}, \cite[pp.~601, 603, 605\==606]{SharmaEtAl2021}                                                           \\
            Endurance Testing        & Technique \cite[pp.~38\==39]{IEEE2021c}                             & Type \cite[p.~2]{IEEE2013}                                                                                                                      \\
            %                                                                                                                                                   pp.~iii\==iv, 4, 11, 29, 35, 122, 125, 
            Error Guessing           & Practice \cite[p.~33]{IEEE2013}                                     & Technique \cite[pp.~4, 34, Fig.~2]{IEEE2022}, \cite[Fig.~2, Tab.~A.2]{IEEE2021c}\footnote{Some sources omitted for brevity.}                    \\ %, \cite[pp.~3, 33]{IEEE2013}, \cite[p.~5\=/13]{SWEBOK2024}, \cite[p.~50]{Firesmith2015} \\
            Experience-based Testing & Technique \cite[pp.~46, 50]{Firesmith2015}, \cite[Fig.~2]{IEEE2022} & Practice \cite[Fig.~2]{IEEE2022}, \cite[p.~viii]{IEEE2021c}, \cite[pp.~iii, 31, 33]{IEEE2013}                                                   \\
            %                                                                                                                                                                                                   TODO: remove footnote duplication
            Exploratory Testing      & Technique \cite[p.~50]{Firesmith2015}, \cite[p.~5\=/14]{SWEBOK2024} & Practice \cite[pp.~11, 20, 34, Fig.~2]{IEEE2022}, \cite[p.~viii]{IEEE2021c}, \cite[p.~5]{IEEE2021a}\footnote{Some sources omitted for brevity.} \\ %, \cite[pp.~13, 33]{IEEE2013} \\
            Load Testing             & Technique \cite[pp.~38\==39]{IEEE2021c}                             & Type \cite[p.~253]{IEEE2017}, \cite[pp.~5, 20, 22]{IEEE2022}, \cite{ISTQB}                                                                      \\
            Performance Testing      & Technique \cite[pp.~38\==39]{IEEE2021c}                             & Type \cite[pp.~7, 22, 26\==27]{IEEE2022}, \cite[p.~7]{IEEE2021c}, \cite[pp.~2, 8]{IEEE2021a}                                                    \\
            Stress Testing           & Technique \cite[pp.~38\==39]{IEEE2021c}                             & Type \cite[p.~442]{IEEE2017}, \cite[pp.~9, 22]{IEEE2022}                                                                                        \\
            \hline
        \end{tabular}
    \end{minipage}
\end{paperTable}

\end{landscape}

\subsection{Intransitive Synonyms}\label{multiSyns}
There are also cases in which a term is given as a synonym to two (or more)
terms that are not synonyms themselves. Sometimes, these terms
\emph{are} synonyms; for example, \citetISTQB{} \multiAuthHelper{say}
``use case testing'', ``user scenario testing'', and ``scenario testing'' are
all synonyms (although there may be a slight distinction; see
\Cref{tab:parSyns} and \flawref{use-case-scenario}).
% Old explanation based on inconsistent citations from Kam2008/ISTQB
%
% use case testing, user scenario testing,
% and scenario testing are synonyms of each other, as shown in
% \Cref{fig:threeWaySyns}.
% \begin{figure}[hbtp!]
%     \centering
%     \begin{tikzpicture}
%
%         \node[ellipse, draw, align=center] (ust) at (10, 0) {User Scenario\\Testing};
%         \node[ellipse, draw, align=center] (st)  at (0, 0)  {Scenario\\Testing};
%         \node[ellipse, draw, align=center] (uct) at (5, 4)  {Use Case\\Testing};
%
%         \draw[thick] (ust) -- (st)  node [midway, align=center, below]     {\citealpISTQB{}};
%         \draw[thick] (st)  -- (uct) node [midway, align=center, left=12pt] {\citealpISTQB{};\\ \citealp[pp.~47--49]{Kam2008}\\ (see \flawref{use-case-scenario})\\};
%         \draw[thick] (uct) -- (ust) node [midway, align=center, right=4pt] {\citealp[p.~48]{Kam2008}\\};
%
%     \end{tikzpicture}
%     \caption{Visual representation of a three-way synonym relation.}
%     \label{fig:threeWaySyns}
% \end{figure}
However, this does not always make sense. We identify \multiSynCount{}
such cases through automatic analysis of the generated graphs\ifnotpaper,
listed below (test approaches in \emph{italics} are synonyms with each other,
but not with other terms not in italics\todo{Better way to handle/display
    this?})\else. The following three are the most prominent examples\fi:

\begin{enumerate}
    \item \textbf{Invalid Testing:}
\begin{itemize}
    \item Error Tolerance Testing \cite[p.~45]{Kam2008}
    \item Negative Testing \cite{ISTQB} (implied by \cite[p.~10]{IEEE2021b})
\end{itemize}
\item \textbf{Soak Testing:}
\begin{itemize}
    \item Endurance Testing \cite[p.~39]{IEEE2021b}
    \item Reliability Testing\footnote{Endurance testing is given as a child of
              reliability testing by \cite[p.~55]{Firesmith2015}, although the
              terms are not synonyms.} \cite[Tab.~2]{Gerrard2000a},
          \cite[Tab.~1, p.~26]{Gerrard2000b}
\end{itemize}
\item \textbf{Link Testing:}
\begin{itemize}
    \item Branch Testing (implied by \citealp[p.~24]{IEEE2021b})
    \item Component Integration Testing \citep[p.~45]{Kam2008}
    \item Integration Testing (implied by \citealp[p.~13]{Gerrard2000a})
\end{itemize}
\end{enumerate}

\subsection{Synonym and Parent-Child Overlaps}\label{parSyns-full}
As described in \Cref{parSyns}, \parSynIntro*{}.
\begin{landscape}
    \begin{paperTable}
    \centering
    \caption{Pairs of test approaches with both \hyperref[par-chd-rels]{parent-child} and \hyperref[syn-rels]{synonym} relations.}
    \label{tab:parSyns}
    \begin{minipage}{\linewidth}
        \centering
        \begin{tabular}{|rcl|l|l|}
            \hline
            \thead{``Child''}        & \thead{$\to$} & \thead{``Parent''}                              & \thead{Parent-Child Source(s)}                                        & \thead{Synonym Source(s)}                                                   \\
            \hline
            All Transitions Testing  & $\to$         & State Transition Testing                        & \citep[p.~19]{IEEE2021}                                               & \citep[p.~15]{Kam2008}                                                      \\
            Co-existence Testing     & $\to$         & Compatibility Testing                           & \cite[p.~3]{IEEE2022}, \cite[Tab.~A.1]{IEEE2021}, \cite{ISO_IEC2023a} & \citep[p.~37]{IEEE2021}                                                     \\
            Fault Tolerance Testing  & $\to$         & Robustness Testing\footnote{\ftrnote{}}         & \citep[p.~56]{Firesmith2015}                                          & \citepISTQB{}                                                               \\
            Functional Testing       & $\to$         & Specification-based Testing\footnote{\specfn{}} & \citep[p.~38]{IEEE2021}                                               & \cite[p.~196]{IEEE2017}, \cite[p.~399]{vanVliet2000}, \cite[p.~44]{Kam2008} \\
            Orthogonal Array Testing & $\to$         & Pairwise Testing                                & \citep[p.~1055]{Mandl1985}                                            & \cite[p.~5-11]{SWEBOK2024}, \cite[p.~473]{Valcheva2013}                     \\
            Path Testing             & $\to$         & Exhaustive Testing                              & \citep[pp.~466-467, 476]{PetersAndPedrycz2000}                        & \citep[p.~421]{vanVliet2000}                                                \\
            Performance Testing      & $\to$         & Performance-related Testing                     & \cite[p.~22]{IEEE2022}, \cite[p.~38]{IEEE2021}                        & \citep[p.~1187]{Moghadam2019}                                               \\
            Static Analysis          & $\to$         & Static Testing                                  & \cite[pp.~9, 17, 25, 28]{IEEE2022}, \cite{ISTQB}                      & \citep[p.~438]{PetersAndPedrycz2000}                                        \\
            % Omitted parent-child sources for static row :\cite[Fig.~4, p.~12]{Gerrard2000a}, \cite[p.~3]{Gerrard2000b}                                     
            Structural Testing       & $\to$         & Structure-based Testing                         & \citep[pp.~105\=/121]{Patton2006}                                     & \cite[p.~9]{IEEE2022}, \cite{ISTQB}, \cite[pp.~443\=/444]{IEEE2017}         \\
            Use Case Testing         & $\to$         & Scenario Testing\footnote{\ucstn{}}             & \cite[p.~20]{IEEE2021}\todo{OG Hass, 2008}                            & \cite{ISTQB}, \cite[pp.~47-49]{Kam2008}                                     \\
            \hline
        \end{tabular}
    \end{minipage}
\end{paperTable}

\end{landscape}

\section{Inferred Flaws}\label{infer-flaws}
Throughout our research, we infer many potential flaws as described in
\Cref{infers}. Some of these have a conflicting source while others do not.
Since these are more subjective and are based on our own judgement, we
exclude them from any counts of the numbers of flaws but give them here for
completeness.

\subsection{Inferred Synonym Flaws}\label{infMultiSyns}
See \Cref{multiSyns}.

\begin{enumerate}
    \input{build/infMultiSyns}
\end{enumerate}

\subsection{Inferred Parent Flaws}\label{infParSyns}
As discussed in \Cref{parSyns}, some pairs of synonyms also have a
parent-child relation, abusing the meaning of ``synonym'' and causing
confusion. While \Cref{tab:parSyns} gives the cases where both relations
are supported by the literature, some are less explicit. The
following automatically generated lists contain examples where at least
one of these conflicting relations is \emph{not} explicitly supported by the
literature but may, nonetheless, be correct. The relations in the first two
lists are explicitly given in the literature but may be incorrect, while
those in the third list are unsubtantiated by the literature and require
more thought before a recommendation can be made.

\input{build/infParSyns}

In addition to these flaws, \citep[Tab.~2]{Gerrard2000a} does
\emph{not} give ``functionality testing'' as a parent of ``end-to-end
functionality testing'' as we infer he should (see \Cref{orth-test}).

\subsection{Inferred Category Flaws}
See \Cref{multiCats}.

\input{build/infMultiCats}

\subsection{Other Inferred Flaws}
The following are flaws that, if were more concrete, would also be
included alongside the other flaws:
\begin{itemize}
    \item ``Fuzz testing'' is ``tagged'' (?) as ``artificial
          intelligence'' \citep[p.~5]{IEEE2022}.
    \item \citeauthor{Gerrard2000b}'s definition for ``security
          audits'' seems too specific, only applying to ``the products
          installed on a site'' and ``the known vulnerabilities for
          those products'' \citeyearpar[p.~28]{Gerrard2000b}.
\end{itemize}
