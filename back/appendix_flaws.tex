\chapter{Full Lists of Flaws}\label{flaws-full}

The following are the full lists of manually identified flaws, first grouped
by their manifestation, then by their domain as defined in \Cref{flaw-def}.

\section{Full Lists of Flaws by Manifestation}\label{flawMnfsts-full}

We sort the following groups of flaws by their source tier (defined in
\Cref{source-tiers}) in descending order of credibility (defined in
\Cref{cred}).

\subsection{Full List of Mistakes}\label{wrong-full}
\input{build/FlawMnfstWrong}

\subsection{Full List of Omissions}\label{miss-full}
\input{build/FlawMnfstMiss}

\subsection{Full List of Contradictions}\label{contra-full}
\input{build/FlawMnfstContra}

\subsection{Full List of Ambiguities}\label{ambi-full}
\input{build/FlawMnfstAmbi}

\subsection{Full List of Overlaps}\label{over-full}
\input{build/FlawMnfstOver}

\subsection{Full List of Redundancies}\label{redun-full}
\input{build/FlawMnfstRedun}

\section{Full Lists of Flaws by Domain}\label{flawDmn-full}

The following sections provide all of the data that we automatically detect
(as described in \Cref{auto-flaw-analysis}) and summarize in \Cref{flawDmns}.

\subsection{Intransitive Synonyms}\label{multiSyns}
There are also cases in which a term is given as a synonym to two (or more)
terms that are not synonyms themselves. Sometimes, these terms
\emph{are} synonyms; for example, \citetISTQB{} \multiAuthHelper{say}
``use case testing'', ``user scenario testing'', and ``scenario testing'' are
all synonyms (although there may be a slight distinction; see
\Cref{tab:parSyns} and \flawref{use-case-scenario}).
% Old explanation based on inconsistent citations from Kam2008/ISTQB
%
% use case testing, user scenario testing,
% and scenario testing are synonyms of each other, as shown in
% \Cref{fig:threeWaySyns}.
% \begin{figure}[hbtp!]
%     \centering
%     \begin{tikzpicture}
%
%         \node[ellipse, draw, align=center] (ust) at (10, 0) {User Scenario\\Testing};
%         \node[ellipse, draw, align=center] (st)  at (0, 0)  {Scenario\\Testing};
%         \node[ellipse, draw, align=center] (uct) at (5, 4)  {Use Case\\Testing};
%
%         \draw[thick] (ust) -- (st)  node [midway, align=center, below]     {\citealpISTQB{}};
%         \draw[thick] (st)  -- (uct) node [midway, align=center, left=12pt] {\citealpISTQB{};\\ \citealp[pp.~47--49]{Kam2008}\\ (see \flawref{use-case-scenario})\\};
%         \draw[thick] (uct) -- (ust) node [midway, align=center, right=4pt] {\citealp[p.~48]{Kam2008}\\};
%
%     \end{tikzpicture}
%     \caption{Visual representation of a three-way synonym relation.}
%     \label{fig:threeWaySyns}
% \end{figure}
However, this does not always make sense. We identify \multiSynCount{}
such cases through automatic analysis of the generated graphs\ifnotpaper,
listed below (test approaches in \emph{italics} are synonyms with each other,
but not with other terms not in italics\todo{Better way to handle/display
    this?})\else. The following three are the most prominent examples\fi:

\begin{enumerate}
    \item \textbf{Invalid Testing:}
\begin{itemize}
    \item Error Tolerance Testing \cite[p.~45]{Kam2008}
    \item Negative Testing \cite{ISTQB} (implied by \cite[p.~10]{IEEE2021b})
\end{itemize}
\item \textbf{Soak Testing:}
\begin{itemize}
    \item Endurance Testing \cite[p.~39]{IEEE2021b}
    \item Reliability Testing\footnote{Endurance testing is given as a child of
              reliability testing by \cite[p.~55]{Firesmith2015}, although the
              terms are not synonyms.} \cite[Tab.~2]{Gerrard2000a},
          \cite[Tab.~1, p.~26]{Gerrard2000b}
\end{itemize}
\item \textbf{Link Testing:}
\begin{itemize}
    \item Branch Testing (implied by \citealp[p.~24]{IEEE2021b})
    \item Component Integration Testing \citep[p.~45]{Kam2008}
    \item Integration Testing (implied by \citealp[p.~13]{Gerrard2000a})
\end{itemize}
\end{enumerate}

\subsection{Synonym and Parent-Child Overlaps}\label{parSyns-full}
As described in \Cref{parSyns}, \parSynIntro*{}.
\begin{landscape}
    \begin{paperTable}
    \centering
    \caption{Pairs of test approaches with both \hyperref[par-chd-rels]{parent-child} and \hyperref[syn-rels]{synonym} relations.}
    \label{tab:parSyns}
    \begin{minipage}{\linewidth}
        \centering
        \begin{tabular}{|rcl|l|l|}
            \hline
            \thead{``Child''}        & \thead{$\to$} & \thead{``Parent''}                              & \thead{Parent-Child Source(s)}                                        & \thead{Synonym Source(s)}                                                   \\
            \hline
            All Transitions Testing  & $\to$         & State Transition Testing                        & \citep[p.~19]{IEEE2021}                                               & \citep[p.~15]{Kam2008}                                                      \\
            Co-existence Testing     & $\to$         & Compatibility Testing                           & \cite[p.~3]{IEEE2022}, \cite[Tab.~A.1]{IEEE2021}, \cite{ISO_IEC2023a} & \citep[p.~37]{IEEE2021}                                                     \\
            Fault Tolerance Testing  & $\to$         & Robustness Testing\footnote{\ftrnote{}}         & \citep[p.~56]{Firesmith2015}                                          & \citepISTQB{}                                                               \\
            Functional Testing       & $\to$         & Specification-based Testing\footnote{\specfn{}} & \citep[p.~38]{IEEE2021}                                               & \cite[p.~196]{IEEE2017}, \cite[p.~399]{vanVliet2000}, \cite[p.~44]{Kam2008} \\
            Orthogonal Array Testing & $\to$         & Pairwise Testing                                & \citep[p.~1055]{Mandl1985}                                            & \cite[p.~5-11]{SWEBOK2024}, \cite[p.~473]{Valcheva2013}                     \\
            Path Testing             & $\to$         & Exhaustive Testing                              & \citep[pp.~466-467, 476]{PetersAndPedrycz2000}                        & \citep[p.~421]{vanVliet2000}                                                \\
            Performance Testing      & $\to$         & Performance-related Testing                     & \cite[p.~22]{IEEE2022}, \cite[p.~38]{IEEE2021}                        & \citep[p.~1187]{Moghadam2019}                                               \\
            Static Analysis          & $\to$         & Static Testing                                  & \cite[pp.~9, 17, 25, 28]{IEEE2022}, \cite{ISTQB}                      & \citep[p.~438]{PetersAndPedrycz2000}                                        \\
            % Omitted parent-child sources for static row :\cite[Fig.~4, p.~12]{Gerrard2000a}, \cite[p.~3]{Gerrard2000b}                                     
            Structural Testing       & $\to$         & Structure-based Testing                         & \citep[pp.~105\=/121]{Patton2006}                                     & \cite[p.~9]{IEEE2022}, \cite{ISTQB}, \cite[pp.~443\=/444]{IEEE2017}         \\
            Use Case Testing         & $\to$         & Scenario Testing\footnote{\ucstn{}}             & \cite[p.~20]{IEEE2021}\todo{OG Hass, 2008}                            & \cite{ISTQB}, \cite[pp.~47-49]{Kam2008}                                     \\
            \hline
        \end{tabular}
    \end{minipage}
\end{paperTable}

\end{landscape}
