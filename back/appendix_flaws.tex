\chapter{Full Lists of Flaws}\label{flaws-full}

The following are the full lists of manually identified flaws, first grouped
by their manifestation, then by their domain as defined in \Cref{flaw-def}.

\section{Full Lists of Flaws by Manifestation}\label{flawMnfsts-full}

\subsection{Full List of Mistakes}\label{wrong-full}
\input{build/FlawMnfstWrong}

\subsection{Full List of Omissions}\label{miss-full}
\input{build/FlawMnfstMiss}

\subsection{Full List of Contradictions}\label{contra-full}
\input{build/FlawMnfstContra}

\subsection{Full List of Ambiguities}\label{ambi-full}
\input{build/FlawMnfstAmbi}

\subsection{Full List of Overlaps}\label{over-full}
\input{build/FlawMnfstOver}

\subsection{Full List of Redundancies}\label{redun-full}
\input{build/FlawMnfstRedun}

\section{Full Lists of Flaws by Domain}\label{flawDmn-full}

\subsection{Full List of Synonym Flaws}

\subsubsection{Intransitive Synonyms}\label{multiSyns}
There are also cases in which a term is given as a synonym to two (or more)
terms that are not synonyms themselves. Sometimes, these terms
\emph{are} synonyms; for example, \citetISTQB{} \multiAuthHelper{say}
``use case testing'', ``user scenario testing'', and ``scenario testing'' are
all synonyms (although there may be a slight distinction; see
\Cref{tab:parSyns} and \flawref{use-case-scenario}).
% Old explanation based on inconsistent citations from Kam2008/ISTQB
%
% use case testing, user scenario testing,
% and scenario testing are synonyms of each other, as shown in
% \Cref{fig:threeWaySyns}.
% \begin{figure}[hbtp!]
%     \centering
%     \begin{tikzpicture}
%
%         \node[ellipse, draw, align=center] (ust) at (10, 0) {User Scenario\\Testing};
%         \node[ellipse, draw, align=center] (st)  at (0, 0)  {Scenario\\Testing};
%         \node[ellipse, draw, align=center] (uct) at (5, 4)  {Use Case\\Testing};
%
%         \draw[thick] (ust) -- (st)  node [midway, align=center, below]     {\citealpISTQB{}};
%         \draw[thick] (st)  -- (uct) node [midway, align=center, left=12pt] {\citealpISTQB{};\\ \citealp[pp.~47--49]{Kam2008}\\ (see \flawref{use-case-scenario})\\};
%         \draw[thick] (uct) -- (ust) node [midway, align=center, right=4pt] {\citealp[p.~48]{Kam2008}\\};
%
%     \end{tikzpicture}
%     \caption{Visual representation of a three-way synonym relation.}
%     \label{fig:threeWaySyns}
% \end{figure}
However, this does not always make sense. We identify \multiSynCount{}
such cases through automatic analysis of the generated graphs\ifnotpaper,
listed below (test approaches in \emph{italics} are synonyms with each other,
but not with other terms not in italics\todo{Better way to handle/display
    this?})\else. The following three are the most prominent examples\fi:

\begin{enumerate}
    \item \textbf{Invalid Testing:}
\begin{itemize}
    \item Error Tolerance Testing \cite[p.~45]{Kam2008}
    \item Negative Testing \cite{ISTQB} (implied by \cite[p.~10]{IEEE2021b})
\end{itemize}
\item \textbf{Soak Testing:}
\begin{itemize}
    \item Endurance Testing \cite[p.~39]{IEEE2021b}
    \item Reliability Testing\footnote{Endurance testing is given as a child of
              reliability testing by \cite[p.~55]{Firesmith2015}, although the
              terms are not synonyms.} \cite[Tab.~2]{Gerrard2000a},
          \cite[Tab.~1, p.~26]{Gerrard2000b}
\end{itemize}
\item \textbf{Link Testing:}
\begin{itemize}
    \item Branch Testing (implied by \citealp[p.~24]{IEEE2021b})
    \item Component Integration Testing \citep[p.~45]{Kam2008}
    \item Integration Testing (implied by \citealp[p.~13]{Gerrard2000a})
\end{itemize}
\end{enumerate}
