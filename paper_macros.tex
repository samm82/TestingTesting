%------------------------------------------------------------------------------
% Spacing Options
%------------------------------------------------------------------------------

\newcommand{\thesisForceSingleSpacing}{\singlespacing}
\newcommand{\thesisForceDoubleSpacing}{\doublespacing}

%------------------------------------------------------------------------------
% Portable HREFs
%------------------------------------------------------------------------------

% Common variant
\newcommand{\porthref}[2]{\href{#2}{#1}\printOnlyFootnote{\url{#2}}}

% Custom URLs
\newcommand{\porthreft}[3]{\href{#3}{#1}\printOnlyFootnote{\href{#3}{#2}}}
% Inside of some environments, footnote marks aren't registered properly, so we
% need to manually write the "text" part
\newcommand{\porthreftm}[2]{\href{#2}{#1\printOnlyFootnoteMark}}

%------------------------------------------------------------------------------
% Optional refs in format of (see <ref>)
%------------------------------------------------------------------------------

\newif\ifnotpaper
\newcommand{\seeSection}[1]{%
    \ifnotpaper
        \seeSectionAlways{#1}%
    \fi
}
\newcommand{\seeSectionPar}[2][]{%
    \ifnotpaper
        \seeSectionParAlways{#2}{#1}%
    \fi
}
\newcommand{\seeSectionFoot}[1]{%
    \ifnotpaper
        \seeSectionFootAlways{#1}%
    \fi
}
\NewDocumentCommand{\seeSectionAlways}{ m }{\seeAlways{#1}}
\NewDocumentCommand{\seeSectionParAlways}{ m O{} }{\seeParAlways{#1}[#2]}
\NewDocumentCommand{\seeSectionFootAlways}{ m }{\seeFootAlways{#1}}

\NewDocumentCommand{\seeThesisIssue}{ m }{\ifnotpaper\seeAlways[\thesisissueref]{#1}\fi}
\NewDocumentCommand{\seeThesisIssuePar}{ m o }
{\ifnotpaper\seeParAlways[\thesisissueref]{#1}[#2]\fi}
\NewDocumentCommand{\seeThesisIssueFoot}{ m }
{\ifnotpaper\seeFootAlways[\thesisissueref]{#1}\fi}

\NewDocumentCommand{\seeAlways}{ O{\Cref} m }{{ see #1{#2}}}
\NewDocumentCommand{\seeParAlways}{ O{\Cref} m O{} }{{ (see #1{#2}#3)}}
\NewDocumentCommand{\seeFootAlways}{ O{\Cref} m }{\footnote{See #1{#2}.}}

\newcommand{\formatPaper}[2]{%
    \ifnotpaper
        #1{#2}%
    \else
        \underline{#2}%
    \fi
}

\newcommand\ifblind[2]{\IfEndWith*{\jobname}{_blind}{#1}{#2}}

%------------------------------------------------------------------------------
% Generic "chunks" that get reused
%------------------------------------------------------------------------------

\newcommand{\accelTolTest}{astronauts \citep[p.~11]{MorgunEtAl1999}, aviators
    \citep[pp.~27,~42]{HoweAndJohnson1995}, or catalysts
    \citep[p.~1463]{LiuEtAl2023}}

\newcommand{\procLevel}[1]{``Test level'' can also refer to the scope
of a test process; for example, ``across the whole organization'' or only
``to specific projects'' #1[p.~24]{IEEE2022}}
\newcommand{\phaseDef}{can also refer to the ``period of time in the software
    life cycle'' when testing occurs \citeyearpar[p.~470]{IEEE2017}, usually
    after the implementation phase
    \ifnotpaper
        (\citeyear[pp.~420,~509]{IEEE2017}; \citealp[p.~56]{Perry2006}).
    \else
        \cite[pp.~420,~509]{IEEE2017}, \cite[p.~56]{Perry2006}.
    \fi}

% Define common footnotes about IEEE testing terms for reuse
\newcommand{\distinctIEEE}[1]{distinct from the notion of ``test #1'' described
    in \nameref{tab:ieeeTestTerms}.}
\newcommand{\notDefDistinctIEEE}[1]{\footnote{Not formally defined, but
        \distinctIEEE{#1}}}
\newcommand{\gerrardDistinctIEEE}[1]{\footnote{``Each type of test addresses a
        different risk area'' \citep[p.~12]{Gerrard2000a}, which is
        \distinctIEEE{#1}}}

% Capitalization can change
\newcommand{\ftrnote}[1]{#1ault tolerance testing may also be a sub-approach of
    reliability testing \ifnotpaper
        \citetext{\citealp[p.~375]{IEEE2017}; \citealp[p.~7-10]{SWEBOK2024}}%
    \else \cite[p.~375]{IEEE2017}, \cite[p.~7-10]{SWEBOK2024}%
    \fi, which is distinct from robustness testing \citep[p.~53]{Firesmith2015}.}

\def\graphGenDesc{
    We \ifnotpaper\else then \fi developed a tool to automatically
    generate graphs of the relations between test approaches.
    \ifnotpaper Since child-parent and synonym relations between approaches are
        tracked in \ifblind{our glossary}{\href{
                https://github.com/samm82/TestGen-Thesis/blob/main/ApproachGlossary.csv
            }{our glossary}} in a consistent format, they can be parsed
        systematically. For example, if the entries in \Cref{tab:exampleGlossary}
        appear in the glossary, then they are displayed as \Cref{fig:exampleGraph}
        in the generated graph.

        \begin{table}[hbtp!]
            \centering
            \begin{tabular}{ll} \hline
                Name                        & Parent(s)                        \\ \hline
                A                           & B (Author, 0000; 0001), C (0000) \\
                B                           & C (implied by Author, 0000)      \\
                C                           & D (Author, 0002)                 \\
                D (implied by Author, 0002) &                                  \\ \hline
            \end{tabular}
            \caption{Sample example glossary entries with only relevant columns shown.}
            \label{tab:exampleGlossary}
        \end{table}

        \ExampleGraph{}

    \fi
    All child-parent relations are graphed, as well as synonym relations where either:
    \begin{enumerate}
        \item both terms are present in the glossary, or
        \item a term is a synonym to more than one term present in the
              glossary.
    \end{enumerate}
    \ifnotpaper These conditions are also deduced from the information parsed
        from the glossary. For example, if the entries in \Cref{tab:synExampleGlossary}
        appear in the glossary, then they are displayed as \Cref{fig:synExampleGraph}
        in the generated graph (note that X does not appear since it does not
        meet the criteria given above).

        \begin{table}[hbtp!]
            \centering
            \begin{tabular}{ll} \hline
                Name & Synonym(s)                        \\ \hline
                E    & F (Author, 0000; implied by 0001) \\
                G    & H (Author, 0000), F (0002)        \\
                H    & X                                 \\ \hline
            \end{tabular}
            \caption{Sample example glossary entries with only relevant columns shown.}
            \label{tab:synExampleGlossary}
        \end{table}

        This allows for automatic detection of some classes of discrepancies. The
        most trivial to automate is ``multi-synonym'' relations, which are already
        found to generate the graph as desired. The list found in \Cref{multiSyns}
        is automatically generated based on glossary entries such as those found
        in \Cref{tab:synExampleGlossary}. The self-referential definitions in
        \Cref{par-rels} were also trivial, found by simply looking for lines
        the generated .tex files starting with \texttt{I -> I} which would
        result in the graph in \Cref{fig:selfExampleGraph}. A similar process
        is used to detect instances where two approaches have a synonym
        \emph{and} a child-parent relation. A dictionary of each term's
        synonyms is built to evaluate which synonym relations are notable
        enough to include in the graph, and these mappings are then checked to
        see if one appears as a parent of the other. For example, if J and K
        are synonyms, a generated .tex file with a parent line starting with
        \texttt{J -> K} would result in these approaches being graphed as shown
        in \Cref{fig:parSynExampleGraph}.

        The visual nature of these graphs make it possible to represent both
        explicit and implicit relations, since this doesn't affect any counts.
        If a relation is both explicit \emph{and} implicit, the implicit relation
        is only shown in the graph if it is from a more ``trusted'' source
        category. Implicit approaches and relations are denoted by dashed lines,
        as shown in \Cref{fig:exampleGraph}; explicit approaches are \emph{always}
        denoted by solid lines, even if they are also implicit. ``Rigid''
        versions of graphs that only include explicit information can also be
        generated.

    \fi
    \Cref{fig:recovery-graph-current,fig:scal-graph-current} are modified versions
    of these graphs generated based on the existing literature, focused on specific
    subsets of testing terminology. This tool was also expanded to be able to make
    changes to these generated graphs based on our \nameref{recs}.
    \Cref{fig:recovery-graph-proposed,fig:scal-graph-proposed,fig:perf-graph} are
    modified versions of these proposed graphs.}

%------------------------------------------------------------------------------
% For populating values from files
%------------------------------------------------------------------------------

\ExplSyntaxOn
\ior_new:N \g_hringriin_file_stream

\NewDocumentCommand{\ReadFile}{mm}
{
    \hringriin_read_file:nn { #1 } { #2 }
    \cs_new:Npn #1 ##1
    {
        \str_if_eq:nnTF { ##1 } { * }
        { \seq_count:c { g_hringriin_file_ \cs_to_str:N #1 _seq } }
        { \seq_item:cn { g_hringriin_file_ \cs_to_str:N #1 _seq } { ##1 } }
    }
}

\cs_new_protected:Nn \hringriin_read_file:nn
{
    \ior_open:Nn \g_hringriin_file_stream { #2 }
    \seq_gclear_new:c { g_hringriin_file_ \cs_to_str:N #1 _seq }
    \ior_map_inline:Nn \g_hringriin_file_stream
    {
        \seq_gput_right:cx
        { g_hringriin_file_ \cs_to_str:N #1 _seq }
        { \tl_trim_spaces:n { ##1 } }
    }
    \ior_close:N \g_hringriin_file_stream
}

\ExplSyntaxOff

% Define/read values for Undefined Terms methodology for reuse and calculation!
\ReadFile{\undefTermCounts}{\miscAssets/undefTermCounts}

\newcount\TotalBefore
\newcount\UndefBefore
\newcount\TotalAfter
\newcount\UndefAfter

\TotalBefore=\undefTermCounts{1}
\UndefBefore=\undefTermCounts{2}
\TotalAfter=\undefTermCounts{3}
\UndefAfter=\undefTermCounts{4}

\def\approachCount{\undefTermCounts{3}}

\ReadFile{\parSynCounts}{build/parSynCounts}

\def\parSynCount{\parSynCounts{1}}
% \def\parSynOne{\parSynCounts{2}}
% \def\parSynBoth{\parSynCounts{3}}
\def\selfCycleCount{\parSynCounts{2}}

\ReadFile{\stdDiscBrkdwn}{build/stdDiscBrkdwn}
\ReadFile{\metaDiscBrkdwn}{build/metaDiscBrkdwn}
\ReadFile{\textDiscBrkdwn}{build/textDiscBrkdwn}
\ReadFile{\otherDiscBrkdwn}{build/otherDiscBrkdwn}

\ReadFile{\otherDiscrepCounts}{build/otherDiscrepCounts}

% From https://tex.stackexchange.com/a/248135/192195
\newcounter{laterdef} % just a dummy counter
\newcommand{\laterdef}[2]{%
    \renewcommand\thelaterdef{#2}%
    \refstepcounter{laterdef}\label{#1}%
}

%------------------------------------------------------------------------------
% TODOs
%------------------------------------------------------------------------------

% Generic Inlined TODOs
\newcommand{\intodo}[1]{\todo[inline]{#1}}

% Unimportant TODOs for "later" (i.e., finishing touches or changes immediately before submission)
\newcommand{\latertodo}[1]{\todo[backgroundcolor=Cyan]{\textit{Later}: #1}}

% "Important" TODOs
\newcommand{\imptodo}[1]{\todo[inline,backgroundcolor=Red]{\textbf{Important}: #1}}

% "Easy" TODOs
\newcommand{\easytodo}[1]{\todo[inline,backgroundcolor=SeaGreen]{\textit{Easy}: #1}}
\newcommand{\eztodo}[1]{\easytodo{#1}}

% "Tedious" TODOs
\newcommand{\tedioustodo}[1]{\todo[inline,backgroundcolor=PineGreen]{\textit{Needs time}: #1}}

% "Question" TODO Notes
\newcounter{todonoteQuestionsCtr}
\newcommand{\questiontodo}[1]{\stepcounter{todonoteQuestionsCtr}\todo[backgroundcolor=Lavender]{\textbf{Q \#\thetodonoteQuestionsCtr{}}: #1}}
\newcommand{\qtodo}[1]{\questiontodo{#1}}

%------------------------------------------------------------------------------
% Citations
%------------------------------------------------------------------------------

\newcommand{\exhInfCite}{(\citealp[p.~5-5]{SWEBOK2024}; \citealp[p.~4]{IEEE2022};
    \citealp[p.~421]{vanVliet2000}; \citealp[pp.~439, 461]{PetersAndPedrycz2000})}

%------------------------------------------------------------------------------
% Link to Drasil issue
%------------------------------------------------------------------------------

\newcommand{\issueref}[1]{\href{https://github.com/JacquesCarette/Drasil/issues/#1}{\##1}}
\newcommand{\pullref}[1]{\href{https://github.com/JacquesCarette/Drasil/pull/#1}{\##1}}
\newcommand{\thesisissueref}[1]{\href{https://github.com/samm82/TestGen-Thesis/issues/#1}{\##1}}

