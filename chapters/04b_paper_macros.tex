\phantomsection{}\label{paper-macros}
In addition to this thesis, we also prepare a conference paper based on our
research. While we can reuse most content without modifying it, we need to
address the formatting differences between the two document types.
In general, we use the conditional \texttt{\textbackslash ifnotpaper} to allow
for manual distinctions between the two documents' formats using this basic
template:
\begin{displayquote}
    \texttt{\textbackslash ifnotpaper <thesis code> \textbackslash else <paper code> \textbackslash fi}
\end{displayquote}
We define the macros given in \Cref{tab:macrosPaper} for recurring edge cases
between formatting requirements for these document types, presented along with
example usages and renderings. Since this document is the thesis, we have to
hardcode some of the paper renderings, so they may not be perfectly accurate.

\def\refHelperEx{\refHelper \ifnotpaper \citet{IEEE2022}
    \else \defcitealias{IEEE2022}{16}[\citetalias{IEEE2022}]
    \fi \multiAuthHelper{form} the basis of this \docType{}.}

\begin{longtblr}[
    note{a} = {Section omitted for brevity.},
    caption = {\LaTeX{} macros for handling formatting differences between thesis and paper.},
    label = {tab:macrosPaper}
    ]{
    colspec={|Q[c,m]|X[l,m]|},
    column{1} = {font=\bfseries},
    width = \textwidth, rowhead = 1
    }
    \hline
    \thead{Context} & \thead{Displayed as}                                                \\
    \hline
    Code            & {\texttt{\macro{refHelper} \macro[IEEE2022]{citet}\
    \macro[form]{multiAuthHelper}} \displayNL\texttt{the basis of this \macro{docType}.}} \\*
    Thesis          & \refHelperEx{}                                                      \\*
    Paper           & {\notpaperfalse \refHelperEx{}}                                     \\
    \hline
    Code            & \macro[cat-acro]{flawref}                                           \\*
    Thesis          & \flawref{cat-acro}                                                  \\*
    Paper           & Section \hyperref[cat-acro]{III-B2}                                 \\
    \hline
    Code            & \macro{redun}                                                       \\*
    Thesis          & \redun{}                                                            \\*
    Paper           & Redundancies\TblrNote{a}                                            \\
    \hline
\end{longtblr}


The most common difference between the two document styles is how citations are
displayed. We use the \texttt{natbib} package for our thesis, but the IEEE
guidelines for paper submissions suggest the use of \texttt{cite}
\citep[p.~8]{Shell2015a}. For simplicity, we define aliases so we can reuse
text that includes a single citation
\seeSrcCode{0a2dcdf}{paper_preamble.tex}{19}{60}. However, text that
cites multiple sources requires more work to be reused and has to be done so
manually, since the \texttt{natbib} package groups multiple citations within a
single set of parentheses while the \texttt{cite} package keeps them separate
inside their own set of square brackets.

For example, in \Cref{nonIEEE-sources}, we provide a list of non-IEEE sources
that support a claim made by the IEEE. Since we sort sources based on
credibility (defined in \Cref{cred}) and then by publication year, the
relevant thesis code is given below. (We use ``\texttt{\textbackslash =/}'',
etc. from the \texttt{extdash} package for non-breaking dashes.)
\begin{displayquote}
    \texttt{(\textbackslash citealp[pp.\textasciitilde 5\textbackslash =/6 to 5\textbackslash =/7]\{SWEBOK2024\};
        \displayNL \textbackslash citealpISTQB\{\};
        \textbackslash citealp[pp.\textasciitilde 807\textbackslash ==808]\{Perry2006\};
        \displayNL \textbackslash citealp[pp.\textasciitilde 443\textbackslash ==445]\{PetersAndPedrycz2000\};
        \displayNL \textbackslash citealp[p.\textasciitilde 218]\{KuļešovsEtAl2013\}; %\textbackslash todo\{OG Black, 2009\};
        \displayNL \textbackslash citealp[pp.\textasciitilde 9, 13]\{Gerrard2000a\})}
\end{displayquote}
Meanwhile, IEEE guidelines recommend that references ``appear in the order in
which they are cited'' \citep[p.~1]{Shell2015b}. Therefore, the relevant paper
code for this list of sources is:
\begin{displayquote}
    \texttt{\textbackslash cite[pp.\textasciitilde 443\textbackslash ==445]\{PetersAndPedrycz2000\},
        \displayNL \textbackslash cite[pp.\textasciitilde 5\textbackslash =/6 to 5\textbackslash =/7]\{SWEBOK2024\},
        \textbackslash cite\{ISTQB\},
        \displayNL \textbackslash cite[pp.\textasciitilde 807\textbackslash ==808]\{Perry2006\},
        \displayNL \textbackslash cite[pp.\textasciitilde 9, 13]\{Gerrard2000a\},
        \displayNL \textbackslash cite[p.\textasciitilde 218]\{KuļešovsEtAl2013\}}
\end{displayquote}\utd{}In particular, note the usage of the \macro{cite}
command, the \emph{lack} of use of the custom alias for citing the \acs{istqb}
glossary, and the different ordering and punctuation.
% , and the lack of the \macro{todo}, since these are only rendered for
% reference in the thesis.
