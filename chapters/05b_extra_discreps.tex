\begin{enumerate}
    \item % Discrep count (SYNS, OVER): {IEEE2017} {IEEE2013} | {IEEE2022} {IEEE2017} {Perry2006}
          \citeauthor*{IEEE2017} say that ``test level'' and ``test phase''
          are synonyms, both meaning a ``specific instantiation of [a] test
          sub-process'' (\citeyear[pp.~469,~470]{IEEE2017}; \citeyear[p.~9]{IEEE2013}),
          but there are also alternative definitions for them.
          \procLevel{\citeyearpar}, while ``test phase'' \phaseDef{}
    \item % Discrep count (DEFS, CONTRA): {IEEE2021} | {IEEE2017}
          % TODO: SUPP DEFS?
          \citeauthor{IEEE2021} define an ``extended entry (decision) table''
          both as a decision table where the ``conditions consist of multiple
          values rather than simple Booleans'' \citeyearpar[p.~18]{IEEE2021}
          and one where ``the conditions and actions are generally described
          but are incomplete'' \citeyearpar[p.~175]{IEEE2017}\todo{OG ISO1984}.
    \item % Discrep count (TERMS, WRONG): {IEEE2021}
          A typo in \citep[Fig.~2]{IEEE2021} means that ``specification-based
          techniques'' is listed twice, when the latter should be
          ``structure-based techniques''.
    \item % Discrep count (PARS, OVER): {IEEE2017}
          \citeauthor*{IEEE2017} provide a definition for ``inspections and
          audits'' \citeyearpar[p.~228]{IEEE2017}, despite also giving
          definitions for ``inspection'' \citetext{p.~227} and ``audit''
          \citetext{p.~36}; while the first term \emph{could} be considered a
          superset of the latter two, this distinction doesn't seem useful.
    \item % Discrep count (DEFS, WRONG): {IEEE2017}
          % TODO: SUPP DEFS?
          \citeauthor{IEEE2017} use the same definition for ``partial correctness''
          \citeyearpar[p.~314]{IEEE2017} and ``total correctness'' (p.~480).

          % STD | META
    \item % Discrep count (DEFS, AMBI): ISTQB
          While ergonomics testing is out of scope (as it tests hardware, not
          software), its definition of ``testing to determine whether a
          component or system and its input devices are being used properly
          with correct posture'' \citepISTQB{} seems to focus on how the
          system is \emph{used} as opposed to the system \emph{itself}.
    \item % Discrep count (DEFS, AMBI): ISTQB
          % TODO: SRC? 
          % {SPICE2022} not included as part of this discrepancy since it
          % is used as the ground truth
          \citetISTQB{} describe the term ``software in the loop'' as a
          kind of testing, while the source they reference
          % \ifnotpaper they reference \else it references \fi
          seems to describe
          ``Software-in-the-Loop-Simulation'' as a ``simulation environment''
          that may support software integration testing
          \citep[p.~153]{SPICE2022}; is this a testing approach or a tool
          that supports testing?
    \item % Discrep count (DEFS, WRONG): ISTQB
          \phantomsection{} \label{specificISTQB}
          The definition of ``math testing'' given by \citetISTQB{} is
          too specific to be useful, likely taken from an example instead of
          a general definition: ``testing to determine the correctness of the
          pay table implementation, the random number generator results, and
          the return to player computations''.
    \item % Discrep count (DEFS, WRONG): ISTQB
          A similar issue exists with multiplayer testing, where its
          definition specifies ``the casino game world'' \citepISTQB{}.
    \item % Discrep count (DEFS, REDUN): ISTQB
          While correct, ISTQB's definition of ``specification-based testing''
          is not helpful: ``testing based on an analysis of the specification
          of the component or system'' \citepISTQB{}.
    \item % Discrep count (TERMS, WRONG): ISTQB
          % {Bluejay2024} not included as part of this discrepancy since it
          % is used as the ground truth
          ``Par sheet testing'' from \citepISTQB{} seems to refer to the
          specific example brought up in \hyperref[specificISTQB]
          {this definition discrepancy} and does not seem more widely
          applicable, since a ``PAR sheet'' is ``a list of all the symbols
          on each reel of a slot machine'' \citep{Bluejay2024}.%
          \todo{Does this belong here?}
    \item % Discrep count (SRCS, WRONG): ISTQB
          The source that \citetISTQB{} cite for the definition of ``test
          type'' does not seem to actually provide a definition.
    \item % Discrep count (SRCS, WRONG): ISTQB
          The same is true for ``visual testing'' \citepISTQB{}.
    \item % Discrep count (SRCS, WRONG): ISTQB
          The same is true for ``security attack'' \citepISTQB{}.
    \item % Discrep count (DEFS, AMBI): {Firesmith2015}
          While model testing is said to test the object under test,
          it seems to describe testing the models themselves
          \citep[p.~20]{Firesmith2015}; using the models to test the object
          under test seems to be called ``driver-based testing''
          \citetext{p.~33}.
    \item % Discrep count (DEFS, AMBI): {Firesmith2015}
          Similarly, it is ambiguous whether ``tool/environment testing'' refers
          to testing the tools/environment \emph{themselves} or \emph{using}
          them to test the object under test; the latter is implied, but the
          wording of its subtypes \citep[p.~25]{Firesmith2015} seems to imply
          the former.
    \item % Discrep count (DEFS, MISS): {Firesmith2015}
          The acronym ``SoS'' is used but not defined by
          \citet[p.~23]{Firesmith2015}.
    \item % Discrep count (TERMS, OVER): {Firesmith2015}
          ``Customer acceptance testing'' and ``contract(ual) acceptance
          testing'' have the same acronym (``CAT'') \citep[p.~30]{Firesmith2015}.
    \item % Discrep count (TERMS, OVER): {Firesmith2015} | {Firesmith2015} {PreußeEtAl2012}
          The same is true for ``hardware-in-the-loop testing'' and
          ``human-in-the-loop testing'' (``HIL'') \citep[p.~23]{Firesmith2015},
          although \citet[p.~2] {PreußeEtAl2012} \multAuthHelper{use} ``HiL''
          for the former.
    \item % Discrep count (SRCS, WRONG): {DoğanEtAl2014}
          \citet[p.~184]{DoğanEtAl2014} \multAuthHelper{claim} that
          \citet{SakamotoEtAl2013} \multAuthHelper{define} ``prime path
          coverage'', but they do not.

          % META | TEXT

          % TEXT | OTHER
    \item % Discrep count (TERMS, REDUN): {Gerrard2000a} | {IEEE2022} ISTQB
          The phrase ``continuous automated testing'' \citep[p.~11]{Gerrard2000a}
          is redundant since continuous testing is a sub-category of automated
          testing (\citealp[p.~35]{IEEE2022}; \citealpISTQB{}).
    \item % Discrep count (CATS, AMBI): {IEEE2022} | {BarbosaEtAl2006}
          Retesting and regression testing seem to be separated from the rest
          of the testing approaches \citep[p.~23]{IEEE2022}, but it is not
          clearly detailed why; \citet[p.~3]{BarbosaEtAl2006}
          \multAuthHelper{consider} regression testing to be a test level.
    \item % Discrep count (CATS, CONTRA): {SWEBOK2024} | {Kam2008}
          Although ad hoc testing is sometimes classified as a ``technique''
          \citep[p.~5-14]{SWEBOK2024}, it is one in which ``no recognized test
          design technique is used'' \citep[p.~42]{Kam2008}.
    \item % Discrep count (DEFS, MISS): {Bas2024}
          \refHelper \citet[p.~16]{Bas2024} lists ``three [backup] location
          categories: local, offsite and cloud based [sic]'' but does not
          define or discuss ``offsite backups'' \citetext{pp.~16-17}.
    \item % Discrep count (TERMS, WRONG): {Kam2008}
          \refHelper \citet{Kam2008} misspells ``state-based'' as ``state-base''
          \citetext{pp.~13,~15} and ``stated-base'' \citetext{Tab.~1}.
    \item % Discrep count (CATS, OVER): {Gerrard2000a}
          ``Visual browser validation'' is described as both static \emph{and}
          dynamic in the same table \citep[Tab.~2]{Gerrard2000a}, even though
          they are implied to be orthogonal classifications: ``test types can
          be static \emph{or} dynamic'' \citetext{p.~12,~emphasis added}.
    \item % Discrep count (DEFS, MISS): {Gerrard2000a}
          \refHelper \citet[Tab.~2]{Gerrard2000a} makes a distinction between
          ``transaction verification'' and ``transaction testing'' and
          uses the phrase ``transaction flows'' \citetext{Fig.~5} but doesn't
          explain them.
    \item % Discrep count (DEFS, MISS): {Gerrard2000a}
          Availability testing isn't assigned to a test priority
          \citep[Tab.~2]{Gerrard2000a}, despite the claim that ``the test
          types\gerrardDistinctIEEE{type} have been allocated a slot against
          the four test priorities'' \citetext{p.~13}; I think usability and/or
          performance would have made sense.
    \item % Discrep count (SYNS, WRONG): {SneedAndGöschl2000}
          \refHelper \citet[p.~18]{SneedAndGöschl2000}\todo{OG Hetzel88}
          \multAuthHelper{give} ``white-box testing'', ``grey-box testing'',
          and ``black-box testing'' as synonyms for ``module testing'',
          ``integration testing'', and ``system testing'', respectively, but
          this mapping is incorrect; black-box testing can be performed on a
          module, for example\todo{find source}.
    \item % Discrep count (SYNS, WRONG): {SneedAndGöschl2000}
          The previous discrepancy makes the claim that
          ``red-box testing'' is a synonym for ``acceptance testing''
          \citep[p.~18]{SneedAndGöschl2000} lose credibility.
    \item % Discrep count (SYNS, WRONG): implied by {Kam2008}
          \refHelper \citet[p.~46]{Kam2008} seems to imply that ``mutation
          testing'' is a synonym of ``back-to-back testing''
          but these are two quite distinct techniques.
    \item % Discrep count (SYNS, AMBI): implied by {Kam2008}
          ``Conformance testing'' is implied to be a synonym of ``compliance
          testing'' by \citet[p.~43]{Kam2008} which only makes sense because
          of the vague definition of the latter: ``testing to
          determine the compliance of the component or system''.
\end{enumerate}