\ifnotpaper\section{Tools}
\else\subsection{Tools}
\fi\label{tools}

\ifnotpaper
    To better understand our findings, we build tools to more intuitively
    visualize relations between test approaches (\Cref{app-rel-vis}) and
    automatically track their flaws (\Cref{flaw-analysis}). Doing
    this manually would be error-prone due to the amount of data involved (for
    example, we identify \approachCount{} test approaches) and the number of
    situations where the underlying data would change, including more detailed
    analysis, error corrections, and the addition of data. These all require
    tedious updates to the corresponding visualizations that may be overlooked
    or done incorrectly, so automating these processes allows for our results
    to be reproduced (by us or others) and account for new data. Besides being
    more systematic, this also allows us to observe the impacts of smaller
    changes, such as unexpected flaws that arise from a new relation between
    two approaches. It also helps verify the tools themselves; for example,
    tracking a flaw manually should affect relevant flaw counts, which we can
    double-check. We also define \LaTeX{} macros to help achieve our goals of
    maintainability, traceability, and reproducibility (\Cref{macros}).

    \subsection{Approach Relation Visualization}\label{app-rel-vis}
\fi

To better understand the relations between test approaches, we develop a tool
to generate visualizations of these relations automatically.
\ifnotpaper
    % We can describe these graphs formally as ordered triplets
    %     $G = (A, S, P)$, where:  % Format based on https://en.wikipedia.org/wiki/Graph_theory
    %     \begin{itemize}
    %         \item $T$ is the set of terms assigned to test approaches by the
    %               literature,
    %         \item $A \subseteq T$ is a subset of the \approachCount{} test
    %               approaches we record as rows in \ourApproachGlossary{} that
    %               function as the vertices of the graph,
    %         \item $S \subseteq \left\{ \{x, y\} \mid x, y \in T \,\textrm{ and }\,
    %                   (x \in A \,\textrm{ or }\, y \in A) \,\textrm{ and }\, x \neq y \right\}$
    %               is a subset of our identified synonym relations (defined
    %               in \Cref{syn-rels}) that function as edges, and
    %         \item $P \subseteq \left\{(x,y) \mid (x, y) \in A^2 \right\}$ is a
    %               subset of our identified parent-child relations (defined in
    %               \Cref{par-chd-rels}) that also function as edges.
    %     \end{itemize}
    Since we use a consistent format to track synonym and parent-child
    relations (defined in \Cref{syn-rels,par-chd-rels}, respectively)
    between approaches in \ourApproachGlossary{}, we can parse them
    systematically. For example, if the entries in \Cref{tab:exampleGlossary}
    appear, then their parent-child relations are visualized as
    shown in \Cref{fig:exampleGraph}. We also capture relevant
    citation information in our glossary in the author-year citation format,
    ``reusing'' information from previous citations when applicable.
    For example, the first row of \Cref{tab:exampleGlossary}
    contains the citation ``(Author, 2022; 2021)'', which means that this
    information was present in two documents by Author: one written in
    2022 and one in 2021. The following citation, ``(2022)'',
    contains no author, which means it was written by the same one as the
    previous citation (Author). We process these citations according to this
    logic \seeSrcCode{231bd5e}{scripts/csvToGraph.py}{54}{95} so we can
    consistently track them throughout our analysis.

    % makecell with new lines so VS Code doesn't freak out
\def\app{\makecell{Approach\\Category}}

\begin{table}[hbtp!]
    \centering
    \caption{Selected entries from glossary of test approaches.}
    \label{tab:exampleGlossary}
    \begin{tabularx}{\linewidth}{|m{1.7cm}|m{3cm}|X|m{3.2cm}|m{2.4cm}|X|}
        \hline
        \thead{Name}             & \thead{\app}                                                                                                 & \thead{Definition}                                                                                                                                   & \thead{Parent(s)}                                                                                                                                  & \thead{Synonym(s)}                            & \thead{Notes}                                                                                                                                                                                                             \\
        \hline
        A/B Testing              & Practice \citep[p.~22]{IEEE2022}, Type (implied by \citealp[p.~58]{Firesmith2015})                           & Testing ``that allows testers to determine which of two systems or components performs better'' \citep[p.~1]{IEEE2022}"                              & Statistical Testing \citep[pp.~1,~35]{IEEE2022}, Usability Testing \citep[p.~58]{Firesmith2015}                                                    & Split-Run Testing \citep[pp.~1,~35]{IEEE2022} & ``Not a test case generation technique as test inputs are not generated''; ``us[es] the existing system as a partial oracle'' \citep[p.~36]{IEEE2022}                                                                     \\
        All Combinations Testing & Technique \citetext{\citealp[p.~22]{IEEE2022}; \citeyear[p.~2,~16]{IEEE2021}; \citealp[p.~5-11]{SWEBOK2024}} & Testing that covers ``all unique combinations of P-V pairs'' \citep[p.~16]{IEEE2021}                                                                 & Combinatorial Testing \citetext{\citealp[p.~22]{IEEE2022}; \citeyear[p.~2,~16,~Fig.~2]{IEEE2021}; \citealp[p.~5-11]{SWEBOK2024}}                   &                                               & ``The minimum number of test cases required to achieve 100\% \dots{} coverage corresponds to the product of the number of P-V pairs for each test item parameter'' \citep[p.~16]{IEEE2021}. See also Grindal et al., 2005 \\
        Big-Bang Testing         & Level (inferred from integration testing)                                                                    & ``Testing in which \dots{} [components of a system] are combined all at once into an overall system, rather than in stages'' \citep[p.~45]{IEEE2017} & Integration Testing \citetext{\citealp[p.~45]{IEEE2017}; \citealp[p.~5-7]{SWEBOK2024}; \citealp[p.~603]{SharmaEtAl2021}; \citealp[p.~42]{Kam2008}} &                                               &                                                                                                                                                                                                                           \\
        \hline
    \end{tabularx}
\end{table}
    \ExampleParChdGraphs{}

    \clearpage\fi
\phantomsection{}\label{relevantSyns}
Overall, the parent-child relations between test approaches \emph{should}
result in a hierarchy (or multiple discrete hierarchies), but due to flaws in
the literature such as self-loops (see \Cref{selfParDef}), this is not the
case. Therefore, we visualize all parent-child relations as they are guaranteed
to be meaningful.
% , but only graph some synonym relations.
% For a given synonym pair to
% be captured by our methodology, at least one term will have its own row in its
% relevant glossary.
However, since each term is trivially a synonym of itself and there are many
non-problematic synonyms that do not imply flaws (see \Cref{syn-rels}),
we only visualize the following synonym relations, which may indicate flaws:

% We then decide whether to include or exclude the synonym
% pair from our generated graphs based on the following possible cases:
\begin{enumerate}
    % \item[1. (Excluded)] \phantomsection{}\label{syn-case-one}
    %       \hfill \ifnotpaper
    %           $\left\{ \{x, y\} \in S \mid x \in A \,\textrm{ xor }\, y \in A \right\}$
    %       \fi \break
    %       \textbf{Only one synonym has its own row.}
    %       This is a ``typical'' synonym relation (see \Cref{syn-rels}) where
    %       the terms are interchangeable. We \emph{could} include the synonym
    %       as an alternate name inside the node of its partner, but we do not
    %       want to clutter our graphs unnecessarily.

    \item%[2. (Included)] \phantomsection{}\label{syn-case-two}
          %   \hfill \ifnotpaper
          %       $\left\{ \{x, y\} \in S \mid x, y \in A \right\}$
          %   \fi \break
          \textbf{Synonyms between approaches defined independently.}\hfill\break
          If two separate approaches have their own definitions, nuances,
          etc.~but are also labelled as synonyms, this may indicate that the
          two terms are interchangeable and could be merged \emph{or} that
          either their definitions or this synonym relation is incorrect.
          % TODO: pretty hacky
          % into one row, which would result in \hyperref[syn-case-one]{Case 1} above.

    \item%[3. (Included)] \phantomsection{}\label{syn-case-three}
          %   \hfill \ifnotpaper
          %       $\left\{ \{x, z\}, \{y, z\} \in S \mid x, y \in A \,\textrm{ and }\,
          %           z \notin A \right\}$
          %   \fi \break
          \textbf{Synonyms that violate transitivity.}\hfill\break
          If two distinct approaches share a synonym, that implies that they
          are synonyms themselves. If they are \emph{not}, one or more
          relations may be incorrect or missing.
\end{enumerate}
\ifnotpaper
    We deduce these conditions from the information we parse from our glossary.
    For example, if the entries in \Cref{tab:synExampleGlossary} appear, then
    they are visualized as shown in \Cref{fig:exampleSynGraph} (note that X
    does not appear since it is not defined independently and does not violate
    transitivity).

    \input{build/synExampleGlossary.tex}
    \ExampleSynGraph{}

    \phantomsection{}\label{visExplicit}
    Since we also track the ``explicitness'' of information (defined in
    \Cref{explicitness}), we can represent explicit \emph{and} implicit
    relations without double counting them during the analysis in
    \Cref{flaw-analysis}. If a relation is both explicit \emph{and} implicit,
    we only display the implicit relation if its source is from a more
    credible source tier (see \Cref{cred,source-tiers}). For example, if
    ``StdAuthor'' from \Cref{tab:synExampleGlossary} is an author of an
    established standard, then we display the implicit relation from their
    document alongside the explicit one from ``Author'' as shown in
    \Cref{fig:exampleSynGraph}.
    % For example, only the explicit synonym relation between E and F
    % from \Cref{tab:exampleGlossary} appears in \Cref{fig:synExampleGraph}.
    Implicit approaches and relations are denoted by dashed lines, as shown in
    \Cref{fig:exampleGraph,fig:exampleSynGraph}, while explicit approaches are
    \emph{always} denoted by solid lines, even if they are also implicit. We
    can also generate ``explicit'' versions of these visualizations that exclude
    implicit approaches and relations; for example, \Cref{fig:expExampleGraph}
    is the explicit version of \Cref{fig:exampleGraph}%, and
    % \Cref{fig:expSynGraph} likewise only contains explicit approaches and
    % synonym relations
    .

    We also colour each relation according to its source tier, including
    inferences (see \Cref{infers}) and proposals (see \Cref{recs}). For
    clarity, we only display the relation with the most credible source tier
    (except if there is a more credible implicit relation as we previously
    describe). Each source tier gets its own colour that we include in the
    legend for each visualization (such as \Cref{fig:exampleSynGraph}),
    although we omit this colouring from \Cref{fig:exampleParChdGraphs,%
        fig:exampleFlawGraphs} for clarity.

\fi
These visualizations tend to be large, so it is often useful to focus on
specific subsets of them. \ifnotpaper For each approach category (defined in
    \Cref{cats-def}), we generate a visualization restricted to its approaches
    and the relations between them. We also generate a visualization of all static
    approaches along with the relations between them \emph{and} between a
    static approach and a dynamic approach. This static-focused visualization is
    notable because static testing is sometimes considered to be a separate
    approach category (see \flawref{static-test-flaw}). Since dynamic
    approaches are our primary focus (see \Cref{static-test}), we include them
    in this static visualization, colouring their nodes grey to distinguish them.
    %, as in \Cref{fig:staticExampleGraph}, 

    We can also \else We can \fi generate more focused visualizations from a
given subset of approaches, such as \ifnotpaper\else those in a selected
    approach category (defined in \Cref{cats-def}) or \fi those pertaining to
recovery or scalability\ifnotpaper.
% These areas are of particular note as we discuss their
% flaws in their own sections (\Cref{recov-flaw,scal-flaw}, respectively).
We use these visualizations to better understand the relations within these
subsets of approaches, but we can also update them based on
our recommendations by specifying sets of approaches and relations to add or
remove. % given \else; the latter are shown \fi
% in \Cref{fig:rec-graph-current,fig:scal-graph-current}, respectively.
% applying those given in \Cref{rec-test-rec,,scal-test-rec,,\ifnotpaper\else%
%         perf-test-rec\fi} results in the updated graphs in
% \Cref{fig:rec-graph-proposed,,fig:scal-graph-proposed,,\ifnotpaper\else%
%         fig:perf-graph\fi}, respectively.
When doing so\ifnotpaper\ in \Cref{recs}\fi, we colour any added approaches or
relations orange to distinguish them.
% \ifnotpaper
% Recommendations can also be inherited; for example, we generate
% \Cref{fig:perf-graph} based on the modifications we apply to
% \Cref{fig:rec-graph-proposed,fig:scal-graph-proposed} and
% other changes from \Cref{perf-test-rec}. \fi

\subsection{Flaw Analysis}\label{flaw-analysis}

In addition to manually recording flaws (described in \Cref{record-flaws}), we
also automatically detect certain classes of flaws (\Cref{auto-flaw-detect}).
We can then analyze all of these flaws using automated tools
(\Cref{flaw-comment-analysis}), giving us an overview of:
\begin{itemize}
    \item how many flaws (defined in \Cref{flaw-def}) there are,
    \item how these flaws present themselves (see \Cref{mnfst-def}),
    \item in which knowledge domains these flaws occur (see \Cref{dmn-def}),
    \item how explicit (see \Cref{explicitness}) these flaws are, and
    \item how responsible each source tier (defined in \Cref{source-tiers}) is
          for these flaws.
\end{itemize}

\phantomsection{}\label{flaw-cred-compare}
To understand where flaws exist in the literature, we group them based on the
source tier(s) responsible for them. We then count each flaw \emph{once} per
source tier if it appears within it \emph{and/or} between it and a more
credible tier\footnote{If an inconsistency occurs between two source tiers
    and the more credible one is \emph{incorrect}, we instead count it as an
    inconsistency between it and the asserted truth from the less credible
    source, as described in \Cref{less-cred-assert}.} (see \Cref{cred,%
    source-tiers}). This avoids counting the same flaw
more than once for a given source tier\thesisissueref{83}, which would give the
number of \emph{occurrences} of all flaws instead of the more useful number of
flaws \emph{themselves}. When taking a more detailed look at the \emph{sources}
of flaws (as opposed to just the responsible source \emph{tiers}) as we do in
\Cref{fig:flawBars}, we also count the following sources of flaws separately:
\begin{enumerate}
    \item self-contained flaws (defined in \Cref{one-src-flaws}),
    \item internal flaws (defined in \Cref{one-src-flaws}),
    \item those between documents with the same set of authors, which includes
          \begin{enumerate}
              \item the various combinations of authors of established
                    standards (defined in \Cref{stds})---ISO, the \acf{iec},
                    and IEEE---as shown in \Cref{fig:ieeeSourceSets} and
              \item the different versions of the \acfp{swebok}, which have
                    different editors \citep{SWEBOK2024,SWEBOK2014} but are
                    written by the same organization: the IEEE Computer Society
                    (\citealp{AboutSWEBOK}; see \Cref{metas}), and
          \end{enumerate}
    \item those within a single source tier.
\end{enumerate}
As before, we do not double count these sources of flaws, meaning that the
maximum number of counted flaws possible within a \emph{single} source tier in
this more detailed view is four (one for each type). This only occurs if there
is an example of each flaw source that is \emph{not} ignored to avoid double
counting; for example, while a single flaw within a single document would
technically and trivially fulfill all four criteria, we would only count it
once.

\begin{figure}[bt!]
    \centering
    \begin{tikzpicture}
        \def\radius{4cm}
        \def\spread{1.7}
        \def\offset{\spread*1.8}

        \draw (-\spread, 0) circle (\radius);
        \draw ( \spread, 0) circle (\radius);
        \draw (0, -\offset) circle (\radius);

        \node[above] at (-\offset,       \radius) {ISO};
        \node[above] at ( \offset,       \radius) {IEC};
        \node[above] at (0, -\offset-1.9*\radius) {IEEE};

        % ALL
        \node[above] at (0, -\spread-0.75) {\parbox{3.6cm}{\centering
                \citealp{IEEE2022,IEEE2021a,IEEE2021b,IEEE2021c,IEEE2019a,
                    IEEE2019b,IEEE2017,IEEE2016,IEEE2015,IEEE2013,IEEE2010}}};
        % ISO/IEC
        \node[above] at (0, 0.25*\radius) {\parbox{3.25cm}{\centering
                \citealp{ISO_IEC2023a,ISO_IEC2023b,ISO_IEC2018,ISO_IEC2015,
                    ISO_IEC2014,ISO_IEC2011,ISO_IEC2005}}};
        % ISO/IEEE
        \node[above] at ( \offset, -\offset) {---};
        % IEC/IEEE
        \node[above] at (-\offset, -\offset) {---};

        % ISO
        \node[above] at (-1.4*\offset, 0.25) {\parbox{2cm}{\centering
                \citealp{ISO2022,ISO2015}}};
        % IEC
        \node[above] at ( 1.4*\offset, 0.25) {---};
        % IEEE
        \node[above] at (0, -\spread-1.4*\radius) {\citealp{IEEE2012}};

    \end{tikzpicture}
    \caption{The sets of authors of established standards.}\label{fig:ieeeSourceSets}
\end{figure}

% \phantomsection{}\label{flaw-analysis-example}
% As an example of this process, consider \flawref{level-phase-syns}, where an
% IEEE standard has an internal flaw, an inconsistency with two other IEEE
% standards, and an inconsistency with a textbook. This adds one to the following
% rows of \Cref{tab:flawMnfsts,tab:flawDmns} in the relevant column for a total
% of two counted flaws:

% \begin{itemize}
%     \item \textbf{\stds{}}: this flaw occurs:
%           \begin{enumerate}
%               \item within one standard and
%               \item between three standards (with the same set of authors).
%           \end{enumerate}
%           This increments the count by just one to avoid double counting and
%           would do so even if only one of the above conditions was true. A more
%           nuanced breakdown of flaws that identifies those within a
%           singular document and those between documents by the same author is
%           given in \Cref{fig:flawBars} and explained in more detail in
%           \Cref{flaw-comment-analysis}; this view counts three total flaws here.
%     \item \textbf{\texts{}}: this flaw occurs between a source in this tier and
%           a ``more credible'' one (the IEEE standards; see \Cref{cred}).
%           % \item \textbf{\papers{}}: this flaw occurs between a source in this tier
%           %       and a ``more credible'' one. Even though there are two sources in this
%           %       tier \emph{and} two ``more credible'' tier involved, this increments
%           %       the count by just one to avoid double counting.
% \end{itemize}

\subsubsection{Automated Flaw Detection}\label{auto-flaw-detect}

As outlined in \Cref{relevantSyns}, we automatically detect synonym relations
from \ourApproachGlossary{} that violate transitivity to generate our
visualizations. These relations are significant because they indicate potential
flaws. We automatically detect and format these flaws to present them when
discussing synonym relation flaws in \Cref{syns}. % \Cref{multiSyns,infMultiSyns}
For these and other kinds of flaws, we also generate the corresponding comments
described in \Cref{flaw-comment-syntax} and include them in the
corresponding \LaTeX{} files to ensure that we analyze and count these flaws in
addition to those we record manually. Since we already automatically detect
one kind of flaw, the next logical step is then to detect more.

\ExampleFlawGraphs{}

\phantomsection{}\label{autoSelfPar}
Parent-child relations that violate irreflexivity as outlined in
\Cref{par-chd-rels} (i.e., cases where a child is given as a parent of itself)
are also trivial to automate by looking for lines in the generated \LaTeX{}
files that start with \texttt{I~->~I}, where \texttt{I} is the label used for a
test approach node in these visualizations. This process results in output
similar to \Cref{fig:selfExampleGraph}.% \Cref{selfPars}
\phantomsection{}\label{autoParSyn} We use a similar process to detect pairs of
approaches with a synonym relation \emph{and} a parent-child relation as
described in \Cref{parSynDef}. To find these pairs, we build a dictionary of
each term's synonyms to evaluate which synonym relations are notable enough
to include in the visualization, and then check these mappings to see if one
appears as a parent of the other. For example, if \texttt{J} and \texttt{K} are
synonyms, a generated \LaTeX{} file with a parent line starting with
\texttt{J~->~K} \emph{or} \texttt{K~->~J} would be visualized as shown in
\Cref{fig:parSynExampleGraph}. We present these two
classes of flaws when discussing parent-child relation flaws in \Cref{pars}.
% \Cref{tab:parSyns,infParSyns}

While just counting the total number of flaws (found automatically \emph{or}
manually) is trivial, tracking
the source(s) of these flaws is more useful, albeit more involved. Since
we consistently track the appropriate citations for each piece of information
we record (see \Cref{tab:exampleGlossary,tab:synExampleGlossary} for examples
of our citation format), we can use them to
identify the offending source tier(s). This comes with the added benefit that
we can format these citations to use with \LaTeX{}'s citation commands in this
\docType{}, including generating the comments described in
\Cref{flaw-comment-syntax}.

\phantomsection{}\label{auto-flaw-detect-explicitness}
Alongside this citation information, we include keywords so we can assess how
``explicit'' a piece of information is (see \Cref{explicitness}). This is
useful when counting flaws, since they can be both objective and subjective but
should not be double counted as both\thesisissueref{83,176}! When presenting
the numbers of flaws sorted by various criteria in \Cref{flaws},
%\Cref{tab:flawMnfsts,tab:flawDmns}
we only count each flaw for its most ``explicit'' occurrence%
% (i.e., it will only increment a value in the ``(Sub)jective'' column if it is
% \emph{not} also ``(Obj)ective'')
, similarly to how we visualize the relations
between approaches as described in \Cref{visExplicit}.

\subsubsection{Flaw Comment Analysis}\label{flaw-comment-analysis}

To perform more detailed analysis on the flaws we uncover, we use \LaTeX{}
comments to capture information about the flaws themselves as outlined in
\Cref{flaw-comment-syntax}. We include these flaw comments when manually
recording flaws from the literature (\Cref{record-info}) and generate them when
automatically detecting flaws from \ourApproachGlossary{}
(\Cref{auto-flaw-detect}).

The main way we use these comments is to determine where each flaw originates.
We compare the authors and years of each source involved with a given flaw
to determine if it manifests within a single document and/or between documents
with the same set of authors. Then, we group these sources into their tiers
\seeSrcCode{a13e9d4}{scripts/flawCounter.py}{63}{83}.
% done by the function in \Cref{lst:getSrcCat}, since each source tier
% outlined in \Cref{source-tiers} is comprised of a small number of authors (with
% the exception of papers and other documents; see \Cref{papers}).
We then distill these lists of sources down to sets of tiers and compare them
against each other to determine how many times a given flaw manifests between
source tiers, which we use when counting flaws in \Cref{flaws}.
% This determines which row(s) of \Cref{tab:flawMnfsts,tab:flawDmns}
% and which bar(s) of \Cref{fig:flawBars} that the given flaw should count toward.

% To give a more complete example, we track \flawref{level-phase-syns} with the
% following comment line:\utd{}
% \begin{displayquote}
%     \texttt{\% Flaw count (OVER, SYNS): \{IEEE2017\} \{IEEE2013\} | \{IEEE2022\}
%         \displayNL{} \{IEEE2017\} \{Perry2006\}}
% \end{displayquote}%
% % We parse this as the example given in \Cref{flaw-analysis-example}.
% Since \texttt{IEEE2022}, \texttt{IEEE2017}, and \texttt{IEEE2013} are all
% written by the same standards
% organizations (\begin{NoHyper}\citeauthor{IEEE2022}\end{NoHyper}), we count
% this as an inconsistency between documents with the same set of authors in
% \Cref{fig:flawBars}, but only once to avoid double counting.

We parse implicit information following
the same rules given in \Cref{auto-flaw-detect-explicitness} for
automatically detecting flaws. Note that we only count subjective flaws if there
is not an equivalent objective flaw, as we do when visualizing relations
(as described in \Cref{visExplicit}). The following comment line\utd{} from
\flawref{c-use-def} is an example of a flaw that is both objective and subjective:
\begin{displayquote}
    \texttt{\% Flaw count (CONTRA, DEFS): \{IEEE2021c\} \{IEEE2017\} |
        \displayNL \{vanVliet2000\} implied by \{IEEE2021c\}}
\end{displayquote}
This indicates that the following flaws are present:
\begin{itemize}
    \item an objective inconsistency between a textbook and a standard,
    \item a subjective flaw within a single document, and
    \item a subjective inconsistency between documents with the same set of
          authors (\begin{NoHyper}\citeauthor{IEEE2022}\end{NoHyper}; see
          \Cref{fig:ieeeSourceSets}).
\end{itemize}
This third flaw only affects our more nuanced breakdown of the sources of flaws
in \Cref{fig:flawBars}. % The rest increment their corresponding
% count in \Cref{fig:flawBars,tab:flawMnfsts,tab:flawDmns} by only one.
Note that we do not double count the first flaw. We likewise do not double
count flaws that reappear when comparing between pairs of groups; for example,
we would only count the inconsistency between \texttt{X} and \texttt{Z}
\emph{once} in the following flaw comment:
\begin{displayquote}
    \texttt{\% Flaw count: \{X\} | \{X\} \{Y\} | \{Z\}}
\end{displayquote}

In cases where a flaw can be viewed in multiple ways (see \Cref{multi-view-flaws}),
we only represent the flaw once in the subsections of \Cref{flaws} and
the full lists in \Cref{flaws-full} based on the pair of keys that appears
first. This allows us to decide which view is most central to understanding the
flaw without affecting the results of our research. The choice of which flaw is
more ``meaningful'' only affects its presentation, since we also count its other
views as flaws (for example, in \Cref{tab:flawMnfsts,tab:flawDmns}), with the
benefit of not introducing clutter by displaying it in full multiple times.

\subsection{Helper Commands}\label{macros}
To improve maintainability, traceability, and reproducibility, we define
helper commands (also called ``macros'') for content that is prone to change
or used in multiple places. For example, we use scripts to calculate
values based on our glossaries and save them to files to be assigned to
corresponding macros. We use these throughout our documents instead of
manually updating these constantly changing values, which is prone to error.
\Cref{tab:macrosCalc} lists these macros and descriptions of what they
represent. Our scripts convert numbers to their textual equivalents when
necessary to follow IEEE guidelines.

% `macro` command assisted by GitHub Copilot
\newcommand\macro[1]{\texttt{\textbackslash#1\{\}}}

\begin{longtblr}[
    note{a} = {These macros are defined as counters to allow them to be used in
            calculations within \LaTeX{} (such as in \Cref{undef-terms,fig:undefPies}).},
    caption = {\LaTeX{} macros for calculated values.},
    label = {tab:macrosCalc}
    ]{
    colspec={|X[0.3,l,m]X[0.5,c,m]X[0.2,c,m]|},
    width = \linewidth, rowhead = 1
    }
    \hline
    \thead{Macro}                                   & \thead{What it Counts}        & \thead{Value}    \\
    \hline
    \macro{approachCount}                           & Test approaches identified    & \approachCount{} \\
    \macro{qualityCount}                            & Software qualities identified & \qualityCount{}  \\
    \hline
    \macro{TotalBefore}\TblrNote{a}                 & Test approaches identified
    before process in \Cref{undef-terms}            & \the\TotalBefore{}                               \\
    \macro{UndefBefore}\TblrNote{a}                 & Undefined test approaches
    identified before process in \Cref{undef-terms} & \the\UndefBefore{}                               \\
    \macro{TotalAfter}\TblrNote{a}                  & Test approaches identified
    after process in \Cref{undef-terms}             & \the\TotalAfter{}                                \\
    \macro{UndefAfter}\TblrNote{a}                  & Undefined test approaches
    identified after process in \Cref{undef-terms}  & \the\UndefAfter{}                                \\
    \hline
    \macro{parSynCount}                             & Pairs of test approaches
    with a child-parent \emph{and} synonym
    relation                                        & \parSynCount{}                                   \\
    \macro{selfCycleCount}                          & Test approaches that are
    a parent of themselves                          & \selfCycleCount{}                                \\
    \hline
\end{longtblr}

\phantomsection{}\label{flawCounts}
Additionally, we count flaws based on their manifestation and domain,
explicitness, and source tier (defined in \Cref{flaw-def,explicitness,%
    source-tiers}, respectively).
For each source tier, we create two files that each include both levels of
explicitness: one for manifestations and one for domains. For example, flaws in
standards are saved to \texttt{build/stdFlawMnfstBrkdwn.tex} by manifestation%
% and \texttt{build/stdFlawDmnBrkdwn.tex} for domains
. We then assign these data to macros (such as \macro{stdFlawMnfstBrkdwn}) to
populate \Cref{tab:flawMnfsts,tab:flawDmns}. For example, we access the number
of objective and subjective mistakes in standards by using
\macro[1]{stdFlawMnfstBrkdwn} and \macro[2]{stdFlawMnfstBrkdwn}, respectively.
We follow a similar process for tracking the total numbers of flaws; this
includes \macro[13]{totalFlawMnfstBrkdwn} and \macro[15]{totalFlawDmnBrkdwn}
which are identical and track the total number of identified flaws.

\phantomsection{}\label{text-macros}
Just as with calculated values, it is important that repeated text is updated
consistently, which we accomplish by defining more macros. Similarly to
calculated values, we use scripts generate macros for flaw manifestations,
flaw domains, and source tiers as shown in
\Cref{tab:macrosSections}. The latter are built by extracting all
sources cited in our three glossaries, categorizing, sorting, and formatting
them (including handling edge cases), and saving them to a file. These are then
assigned to \macro{stdSources}, \macro{metaSources}, \macro{textSources}, and
\macro{paperSources} and include:
\begin{enumerate}
    \item the source tier's name,
    \item the list of sources in the tier, and
    \item the number of sources in the tier.
\end{enumerate}
These are accessed by passing in the corresponding number in the above
enumeration (e.g., \macro[2]{paperSources}). We use the first value for the
subheadings in \Cref{source-tiers}, the first two for \Cref{app-src-tiers} and
the third to build \Cref{fig:sourceSummary} and calculate \macro{srcCount}
(see \Cref{tab:macrosCalc}).
% We also define macros for well-defined sections in \Cref{tab:macrosSections}.

% However, we create most of these macros for reused text manually when we first
% notice the reuse, including the source tier macros in \Cref{tab:macrosSections}.
% Some of these macros account for context-specific formatting depending on how
% they are used, such as capitalization. These tend to be less well-defined,
% since they arise naturally from the writing process, so we omit these details
% from the manually created text macros in \Cref{tab:macrosText}, which are
% grouped based on the type of information they contain.

\begin{landscape}
    \begin{longtblr}[
    note{a} = {Defined in \Cref{mnfst-def}; we also define starred versions,
            such as \macro{wrong*} (\wrong*{}), that use the singular noun for
            use in \Cref{tab:flawMnfstDefs}.},
    note{b} = {Defined in \Cref{dmn-def}; we only include domains with their
            own section.},
    note{c} = {Defined in \Cref{source-tiers}.},
    note{d} = {We overwrite the primitive \TeX{} command \macro{over}
            % Source: https://tex.stackexchange.com/a/73825/192195
            since we do not otherwise use it.},
    note{e} = {Used in \Cref{tab:flawMnfsts,tab:flawDmns} to manage line length.},
    caption = {Macros for referencing well-defined sections.},
    label = {tab:macrosSections}
    ]{
    colspec={|Q[1.75cm,c,m]|Q[l,m]Q[r,m]|Q[l,m]Q[r,m]|Q[l,m]Q[r,m]|},
    row{1} = {halign=c},
    width = \textwidth, rowhead = 1
    }
    \hline
                     & \SetCell[c=2]{c} \thead{Flaw Manifestations\TblrNote{a}}  &             & \SetCell[c=2]{c} \thead{Flaw Domains\TblrNote{b}} &           & \SetCell[c=2]{c} \thead{Source Tiers\TblrNote{c}} &              \\
    \hline
    \SetCell[r=6]{c} \textbf{Macros                                                                                                                                                                                               \\
    (Values)}        & \macro{wrong}                                             & (\wrong{})  & \macro{cats}                                      & (\cats{}) & \macro{stds}                                      & (\stds{})    \\
                     & \macro{miss}                                              & (\miss{})   & \macro{syns}                                      & (\syns{}) & \macro{metas}                                     & (\metas{})   \\
                     & \macro{contra}                                            & (\contra{}) & \macro{pars}                                      & (\pars{}) & \macro{texts}                                     & (\texts{})   \\
                     & \macro{ambi}                                              & (\ambi{})   &                                                   &           & \macro{papers}                                    & (\papers{})  \\
                     & \macro{over}\TblrNote{d}                                  & (\over{})   &                                                   &           & \macro{papers*}\TblrNote{e}                       & (\papers*{}) \\
                     & \macro{redun}                                             & (\redun{})  &                                                   &           &                                                                  \\
    \hline
    \textbf{Used In} & \SetCell[c=2]{c} {\Cref{tab:flawMnfstDefs,tab:flawMnfsts}
        % \\ (in both thesis and paper)
    }                &
                     & \SetCell[c=2]{c} {\Cref{tab:flawDmnDefs,tab:flawDmns}
        % \\ (in both thesis and paper)
    }                &                                                           &
    \SetCell[c=2]{c} {
    \Cref{fig:sourceSummary,fig:flawBars,fig:flawBarsSummary,fig:normFlawBarsSummary}                                                                                                                                             \\
        \Cref{tab:flawMnfsts,tab:flawDmns}
        % \\ \Cref{flaw-analysis-example} (only \macro{stds} and \macro{texts})
    }                &                                                                                                                                                                                                            \\
    \hline
\end{longtblr}

\end{landscape}
% \newcommand\macroType[2]{\multirow[c]{#1}{*}{\rotatebox[origin=c]{90}{#2}}}

\begin{longtblr}[
    note{a} = {See \Cref{sources}.},
    caption = {\LaTeX{} macros for reused text.},
    label = {tab:macrosText}
    ]{
    colspec={|Q[c,m]Q[l,m]X[l,m]|},
    row{1} = {halign=c},
    width = \textwidth, rowhead = 1
    }
    \hline
    \thead{Type}                 & \thead{Macro}               & \thead{Used in}                                           \\*
    \hline
    \macroType{7}{Discrepancies} & \macro{bugPattonDiscrep}    & \Cref{intro} and \discrepref{bug-patton}                  \\*
                                 & \macro{alphaDiscrep}        & \Cref{intro} and \discrepref{alpha-def}                   \\*
                                 & \macro{loadDiscrep}         & \Cref{intro} and \discrepref{load-def}                    \\*
                                 & \macro{expBasedCatMain}     & \Cref{intro,multiCats}                                    \\*
                                 & \macro{tourDiscrep}         & \Cref{intro,discreps} and \discrepref{tour-def}           \\*
                                 & \macro{perfAsFamily}        & \Cref{method-family,classFamilyDiscrep}                   \\*
                                 & \macro{tolTestingDiscrep}   & \Cref{aug-discrep-analysis} and \discrepref{ground-truth} \\
    \hline
    \macroType{6}{Footnotes}     & \macro{ftrnote}             & \SetCell[r=3]{l} Thesis (automated) and paper (manual)
    versions of \Cref{tab:parSyns}                                                                                         \\*
                                 & \macro{specfn}              &                                                           \\*
                                 & \macro{ucstn}               &                                                           \\*
    \cline{2-3}                  & \macro{notDefDistinctIEEE}  & \discrepref{static-test-discrep} and \Cref{exist-tax}     \\*
                                 & \macro{gerrardDistinctIEEE} & \Cref{tab:otherCategorizations} and
    \discrepref{gerrard-distinct}                                                                                          \\*
                                 & \macro{distinctIEEE}        & \macro{gerrardDistinctIEEE}, \macro{notDefDistinctIEEE},
    and \Cref{method-family,classFamilyDiscrep}                                                                            \\
    \hline
    \macroType{4}{Links}         & \macro{ourApproachGlossary} & \Cref{orth-test,procedure,graph-gen}                      \\*
                                 & \macro{seeSrcCode}          & \Cref{rigidity,graph-gen,auto-discrep-analysis}           \\*
                                 & \macro{suggSrcs}            & \Cref{sources,exist-tax}                                  \\*
                                 & \macro{recFigs}             & \Cref{infers,recs}                                        \\
    \hline
    \macroType{5}{Source Tiers\TblrNote{a}}
                                 & \macro{stds}                &
    \Cref{fig:stdDiscrepSources,fig:sourceSummary,discrep-analysis-example,tab:sntxDiscreps,tab:smntcDiscreps}             \\*
                                 & \macro{metas}               &
    \Cref{fig:metaDiscrepSources,fig:sourceSummary,discrep-analysis-example,tab:sntxDiscreps,tab:smntcDiscreps}            \\*
                                 & \macro{texts}               &
    \Cref{fig:textDiscrepSources,fig:sourceSummary,tab:sntxDiscreps,tab:smntcDiscreps}                                     \\*
                                 & \macro{papers}              &
    \Cref{fig:paperDiscrepSources,fig:sourceSummary,discrep-analysis-example}                                              \\*
                                 & \macro{papersTbl}           & \Cref{tab:sntxDiscreps,tab:smntcDiscreps}                 \\
    \hline
    \macroType{3}{RQs}           & \macro{rqatext}             & \SetCell[r=3]{l} \Cref{intro} and seminar slides          \\*
                                 & \macro{rqbtext}             &                                                           \\*
                                 & \macro{rqctext}             &                                                           \\
    \hline
    \macroType{3}{Misc.}         & \macro{accelTolTest}        & \macro{tolTestingDiscrep} and \Cref{hard-test}            \\*
                                 & \macro{supersAck}           & \hyperref[acknowledgements]{Acknowledgements}
    and seminar slides                                                                                                     \\*
                                 & \macro{displayNL}           & \Cref{graph-gen,aug-discrep-analysis}                     \\*
    \hline
\end{longtblr}


\else
% Moved here to display nicely in paper
\flawMnfstsTable{}
\flawDmnsTable{}
\fi
