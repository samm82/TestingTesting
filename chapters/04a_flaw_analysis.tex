\subsection{Flaw Analysis}
\label{flaw-analysis}

In addition to analyzing specific flaws, it is also useful to examine them at
a higher level. We automate subsets of this task where applicable
(\Cref{auto-flaw-analysis}) and augment the remaining manual portion with
automated tools (\Cref{aug-flaw-analysis}). This gives us an overview of:
\begin{itemize}
    \item how many flaws there are,
    \item how responsible each source tier (see \Cref{sources}) is for these flaws,
    \item how obvious (or ``rigid''; see \Cref{rigidity}) these flaws are,
    \item how these flaws present themselves (see \Cref{flawMnfsts}), and
    \item in which knowledge domains these flaws occur (see \Cref{flawDmns}).
\end{itemize}

To understand where flaws exist in the literature, we group them based on the
source tier(s) responsible for them. Each flaw is then counted \emph{once} per
source tier if it appears within it \emph{and/or} between it and a more
``trusted'' tier (see \Cref{trust,sources}). This avoids counting the same flaw
more than once for a given
source tier\thesisissueref{83}, which would give the number of \emph{occurrences}
of all flaws instead of the more useful number of flaws \emph{themselves}. The
exception to this is \Cref{fig:flawSources}, which counts the following sources
of flaws separately:
\begin{enumerate}
    \item those that appear once in (or consistently throughout) a document
          (i.e., are ``self-contained'')\thesisissueref{137,138},
    \item those between two parts of a single document
          (i.e., internal conflicts)\thesisissueref{137,138},
    \item those between documents with the same set of authors, which includes
          \begin{enumerate}
              \item the different versions of the \acfp{swebok}, which have
                    different editors \citep{SWEBOK2024,SWEBOK2014} but are
                    written by the same organization: the IEEE Computer Society
                    (\citealp{AboutSWEBOK}; see \Cref{metas}) and
              \item the various combinations of ISO, the \acf{iec}, and IEEE as
                    shown in\todo{add Venn diagram}, and
          \end{enumerate}
    \item those within a single source tier.
\end{enumerate}
As before, these are not double counted, meaning that the maximum number of
counted flaws possible within a \emph{single} source tier in
\Cref{fig:flawSources} is four (one for each type). This only occurs if
there is an example of each flaw source that is \emph{not} ignored to
avoid double counting; for example, while a single flaw within a single
document would technically fulfill all four criteria, it would only be counted
once.

\phantomsection{}
\label{flaw-analysis-example}
As an example of this process, consider \flawref{level-phase-syns}, where an
IEEE standard has an internal flaw, and an inconsistency with two other IEEE
standards, and an inconsistency with a textbook. This adds one to the following
rows of \Cref{tab:flawMnfsts,tab:flawDmns} in the relevant column for a total
of two counted flaws:

\begin{itemize}
    \item \textbf{\stds{}}: this flaw occurs:
          \begin{enumerate}
              \item within one standard and
              \item between three standards (with the same set of authors).
          \end{enumerate}
          This increments the count by just one to avoid double counting and
          would do so even if only one of the above conditions was true. A more
          nuanced breakdown of flaws that identifies those within a
          singular document and those between documents by the same author is
          given in \Cref{fig:flawSources} and explained in more detail in
          \Cref{aug-flaw-analysis}; this view counts three total flaws here.
    \item \textbf{\texts{}}: this flaw occurs between a source in this tier and
          a ``more trusted'' one (the IEEE standards; see \Cref{trust}).
          % \item \textbf{\papers{}}: this flaw occurs between a source in this tier
          %       and a ``more trusted'' one. Even though there are two sources in this
          %       tier \emph{and} two ``more trusted'' tier involved, this increments
          %       the count by just one to avoid double counting.
\end{itemize}

\subsubsection{Automated Flaw Analysis}
\label{auto-flaw-analysis}

As outlined in \Cref{graph-gen}, we can detect some classes of flaws
automatically. While just counting the total number of flaws is trivial,
tracking the source(s) of these flaws is more useful yet more involved. Since
we consistently track the appropriate citations for each piece of information
we record (see \Cref{tab:exampleGlossary,tab:synExampleGlossary} for examples
of how these citations are formatted in the glossaries), we can use them to
identify the offending source tier(s). This comes with the added benefit that
we can format these citations to use with \LaTeX{}'s citation commands in this
\docType{}.

We compare the authors and years of each source involved with a given flaw
to determine if it manifests within a single document and/or between documents
with the same set of authors. Then, we group these sources into their tiers
\seeSrcCode{82167b7}{scripts/flawCounter.py}{63}{80}
% done by the function in \Cref{lst:getSrcCat}, since each source tier
% outlined in \Cref{sources} is comprised of a small number of authors (with the
% exception of papers and other documents; see \Cref{papers}).
to determine the row(s) of \Cref{tab:flawMnfsts,tab:flawDmns} and the graph(s)
and slice(s) in \Cref{fig:flawSources} that the given flaw should count toward.
We then distill these lists of sources down to sets of tiers and compare them
against each other to determine how many times a given flaw manifests between
source tiers. Examples of this process are described in more detail in
\Cref{aug-flaw-analysis}.

\phantomsection{}
\label{auto-flaw-analysis-rigidity}
Alongside this citation information, we include keywords so we can assess how
``rigid'' a piece of information is (see \Cref{rigidity}). This is useful when
counting flaws, since they can be both explicit and implicit but should not be
double counted as both\thesisissueref{83}! When counting flaws in
\Cref{tab:flawMnfsts,tab:flawDmns}, each one is
counted only for its most ``rigid'' manifestation (i.e., it will only increment
a value in the ``Implicit'' column if it is \emph{not} also explicit),
similarly to how we generate graphs (see \Cref{graphRigid}).

\subsubsection{Augmented Flaw Analysis}
\label{aug-flaw-analysis}
While we can detect some subsets of flaws automatically by analyzing
\ourApproachGlossary, most are too complex and need to be tracked manually. We
record these more detailed flaws as \LaTeX{} enumeration items along with
comments that we can parse automatically, allowing us to analyze them more
broadly. We also add these comments to flaws we detect automatically before
generating the corresponding \LaTeX{} file to ensure these flaws also get
analyzed. These comments have the following format:
\begin{displayquote}
    \texttt{\% Flaw count (MNFST, DMN): \{A1\} \{A2\} \dots{} | \{B1\} % \{B2\}
        \dots{} | \{C1\} \dots}
\end{displayquote}
\texttt{MNFST} and \texttt{DMN} are placeholders for the ``keys'' given in
\Cref{tab:flawMnfstDefs,tab:flawDmnDefs}, respectively, that we use to track a
flaw's manifestation(s) and domain(s) (defined in \Cref{flaw-def}). We omit
these keys from constructed examples of these comments without associated flaws
throughout this chapter for brevity. Finally, \texttt{A1}, \texttt{A2}, % \texttt{B2},
\texttt{B1}, and \texttt{C1} are each placeholders for a source involved in
this example flaw; in general, there can be arbitrarily many. We represent each
source by its \BibTeX{} key, and wrap each one in curly braces (with the
exception of the \acs{istqb} glossary due to its use of custom commands via
\macro{citealias}) to mimic \LaTeX{}'s citation commands for ease of parsing.
We then separate each ``group'' of sources with a pipe symbol (\texttt{|}) so
we can compare each pair of groups; in general, a flaw can have any number of
groups of sources.

We make a distinction between ``self-contained'' flaws and ``internal'' flaws.
Self-contained flaws are those that manifest by comparing a document to a
source of ground truth. Sometimes, these do not require an explicit comparison;
for example, omissions (listed in \Cref{miss}) often fall into this category,
since the lack of information is contained within a single source and does not
need to be cross-checked against a source of ground truth. If only one group of
sources is present in a flaw's comment, such as the first line below, we
consider it to be a self-contained flaw. On the other hand, internal flaws
arise when a document disagrees with itself by containing two conflicting
pieces of information; this includes many contradictions and overlaps (listed
in \Cref{contra,over}, respectively). These can even occur on the same page,
such as when a source gives an acronym to two distinct terms
(see \flawref{cat-acro,hil-acro})! If a
source appears in multiple groups in a flaw's comment, we consider it to
be an internal flaw. The second line is a standard example of this, while the
third is more complex; in this case, source Y agrees with only one of the
conflicting sources of information in X.
\begin{displayquote}
    \texttt{\% Flaw count: \{X\}\\\% Flaw count: \{X\} | \{X\}\\
        \% Flaw count: \{X\} | \{X\} \{Y\}}
\end{displayquote}
We do not double count flaws that reappear when comparing between pairs of
groups; this means the following line adds an inconsistency between X and Z
\emph{and} between Y and Z \emph{without} double counting the former.
\begin{displayquote}
    \texttt{\% Flaw count: \{X\} | \{X\} \{Y\} | \{Z\}}
\end{displayquote}
To give a more complete example, we track \flawref{level-phase-syns} with the
following comment line:\utd{}
\begin{displayquote}
    \texttt{\% Flaw count (OVER, SYNS): \{IEEE2017\} \{IEEE2013\} | \{IEEE2022\}
        \displayNL{} \{IEEE2017\} \{Perry2006\}}
\end{displayquote}%
We parse this as the example given in \Cref{flaw-analysis-example}. Since
\texttt{IEEE2022}, \texttt{IEEE2017}, and \texttt{IEEE2013} are all written by
the same standards
organizations (\begin{NoHyper}\citeauthor{IEEE2022}\end{NoHyper}), we count
this as an inconsistency between documents with the same set of authors in
\Cref{fig:flawSources}, but only once to avoid double counting.

We can also specify the rigidity (see \Cref{rigidity}) of a flaw by inserting
the phrase ``implied by'' after the sources of explicit information and before
those of implicit information. This information is parsed following the same
rules described in \Cref{auto-flaw-analysis-rigidity} for automatically
detected flaws. Note that we only count implicit flaws if there is not an
equivalent explicit flaw, as we do when generating graphs (\Cref{graphRigid}).
\begin{displayquote}
    \texttt{\% Flaw count (CONTRA, DEFS): \{IEEE2021\} \{IEEE2017\} |
        \displayNL \{vanVliet2000\} implied by \{IEEE2021\}}
\end{displayquote}
For example, the above comment line\utd{} from \flawref{c-use-def} indicates that the
flaws given below are present. The third flaw only affects \Cref{fig:flawSources}
due to its more nuanced breakdown of the sources of flaws. The rest increment
their corresponding count in \Cref{fig:flawSources,tab:flawMnfsts,tab:flawDmns}
by only one:
\begin{itemize}
    \item an explicit inconsistency between a textbook and a standard,
    \item an implicit flaw within a single document, and
    \item an implicit inconsistency between documents with the same set of
          authors (\begin{NoHyper}\citeauthor{IEEE2022}\end{NoHyper}).
\end{itemize}

Occasionally, we use a source from a lower tier as the ``ground truth'' for a
flaw. For example, \tolTestFlaw*{} This flaw is supported
by additional papers found via a miniature literature review (described in
\Cref{undef-terms}) from a lower source tier than \citep{Firesmith2015}
(which is a terminology collection; see \Cref{sources}). However, this flaw
is really based in \citep{Firesmith2015} and not in these
additional papers, but this would be counted as a flaw in these papers if they
were included as detailed above. Therefore, we document these ``ground
truth'' sources separately to track them for traceability without incorrectly
counting flaws, such as the following for this specific example
(\flawref{ground-truth}):
\begin{displayquote}
    \texttt{\% Flaw count (WRONG, LABELS): \{Firesmith2015\}\\
        \% Ground truth: \{LiuEtAl2023\} \{MorgunEtAl1999\} \{HolleyEtAl1996\}
        \displayNL \{HoweAndJohnson1995\}}
\end{displayquote}

\subsection[LaTeX Commands]{\LaTeX{} Commands}\label{macros}
To improve maintainability, traceability, and reproducibility, we define
helper commands (also called ``macros'') for content that is prone to change
or used in multiple places. For example, many values are calculated by a Python
script and saved to a file. We then assign these values to a corresponding
\LaTeX{} macro, which we can use instead of manually replacing the value
throughout our documents every time it changes. \Cref{tab:macrosCalc} lists
these macros, along with descriptions of what they define and their current
values.

\begin{longtblr}[
    note{a} = {Calculated in \LaTeX{} from source tier lists; see \Cref{text-macros}.},
    note{b} = {Alias for \texttt{\textbackslash totalFlawDmnBrkdwn\{13\}}; see \Cref{flawCounts}.},
    note{c} = {These macros are defined as counters to allow them to be used in
            calculations within \LaTeX{} (such as in \Cref{undef-terms,fig:undefPies}).},
    caption = {\LaTeX{} macros for calculated values.},
    label = {tab:macrosCalc}
    ]{
    colspec={|X[0.3,l,m]X[0.5,c,m]X[0.2,c,m]|},
    width = \linewidth, rowhead = 1
    }
    \hline
    \thead{Macro}                                   & \thead{What it Counts}        & \thead{Value}    \\
    \hline
    \macro{approachCount}                           & Identified test approaches    & \approachCount{} \\
    \macro{qualityCount}                            & Identified software qualities & \qualityCount{}  \\
    \macro{srcCount}\TblrNote{a}                    & Sources used in glossaries    & \srcCount{}      \\
    \macro{flawCount}\TblrNote{b}                   & Identified flaws              & \flawCount{}     \\
    \hline
    \macro{TotalBefore}\TblrNote{c}                 & Test approaches identified
    before process in \Cref{undef-terms}            & \the\TotalBefore{}                               \\
    \macro{UndefBefore}\TblrNote{c}                 & Undefined test approaches
    identified before process in \Cref{undef-terms} & \the\UndefBefore{}                               \\
    \macro{TotalAfter}\TblrNote{c}                  & Test approaches identified
    after process in \Cref{undef-terms}             & \the\TotalAfter{}                                \\
    \macro{UndefAfter}\TblrNote{c}                  & Undefined test approaches
    identified after process in \Cref{undef-terms}  & \the\UndefAfter{}                                \\
    \hline
    \macro{multiSynCount}                           & Terms given as synonyms for
    multiple discrete terms                         & \multiSynCount{}                                 \\
    \macro{parSynCount}                             & Pairs of test approaches
    with a child-parent \emph{and} synonym relation & \parSynCount{}                                   \\
    \macro{selfCycleCount}                          & Test approaches that are
    a parent of themselves                          & \selfCycleCount{}                                \\
    \hline
\end{longtblr}


\phantomsection{}\label{flawCounts}
Additionally, we count flaws based on their rigidity, source tier,
and whether they are syntactic or semantic (see
\Cref{rigidity,sources,auto-flaw-analysis,aug-flaw-analysis,flaws}).
We save these counts to files, a syntactic and semantic version for each
source tier%
% ; for example, syntactic flaws in standards are saved to
% \texttt{build/stdFlawMnfstBrkdwn.tex} and semantic flaws in standards to
% \texttt{build/stdFlawDmnBrkdwn.tex}. These data
, then read them in to macros
% (such as \macro{stdFlawDmnBrkdwn})
to populate \Cref{tab:flawMnfsts,tab:flawDmns}. For example,
\macro[1]{stdFlawMnfstBrkdwn} corresponds to the number of explicit mistakes in
standards documents, and \macro[2]{stdFlawMnfstBrkdwn} to the number of implicit
ones. These macros also include \macro[13]{totalFlawMnfstBrkdwn} and \newline
\macro[13]{totalFlawMnfstBrkdwn}, which are identical and
track the total number of identified flaws.

\phantomsection{}\label{text-macros}
Just as with calculated values, it is important that repeated text is updated
consistently, which we accomplish by defining more macros. Some of these are
generated by Python scripts in a similar fashion to calculated values, such as
the lists of sources in \Cref{sources}. These are built by extracting all
sources cited in our three glossaries, categorizing, sorting, and formatting
them (including handling edge cases), and saving them to a file. These are then
defined as \macro{stdSources}, \macro{metaSources}, \macro{textSources}, and
\macro{paperSources}, which include:
\begin{enumerate}
    \item the source tier's name,
    \item the list of sources in the tier, and
    \item the number of sources in the tier.
\end{enumerate}
These are accessed by passing in the corresponding number in the above
enumeration. We use the first one for the subheadings in \Cref{sources}, the
first two for \Cref{app-src-tiers} and the third to build \Cref{fig:sourceSummary}
and calcuate \macro{srcCount} (see \Cref{tab:macrosCalc}). We also define
macros for well-defined sections in \Cref{tab:macrosSections}.

\begin{landscape}
    \begin{longtblr}[
    note{a} = {Defined in \Cref{mnfst-def}; we also define starred versions,
            such as \macro{wrong*} (\wrong*{}), that use the singular noun for
            use in \Cref{tab:flawMnfstDefs}.},
    note{b} = {Defined in \Cref{dmn-def}; we only include domains with their
            own section.},
    % we also define starred versions,
    % such as \macro{cats*} (\cats*{}), that use the singular noun.
    note{c} = {Defined in \Cref{source-tiers}.},
    note{d} = {We overwrite the primitive \TeX{} command \macro{over}
            % Source: https://tex.stackexchange.com/a/73825/192195
            since we do not otherwise use it.},
    % note{e} = {We define a starred version, \macro{papers*} (\papers*{}), to shorten the
    %         display name for use in tables.},
    caption = {\LaTeX{} macros for referencing well-defined sections.},
    label = {tab:macrosSections}
    ]{
    colspec={|Q[1.75cm,c,m]|Q[l,m]Q[r,m]|Q[l,m]Q[r,m]|Q[l,m]Q[r,m]|},
    row{1} = {halign=c},
    width = \textwidth, rowhead = 1
    }
    \hline
                     & \SetCell[c=2]{c} \thead{Flaw Manifestations\TblrNote{a}}  &             & \SetCell[c=2]{c} \thead{Flaw Domains\TblrNote{b}} &           & \SetCell[c=2]{c} \thead{Source Tiers\TblrNote{c}} &              \\
    \hline
    \SetCell[r=6]{c} \textbf{Macros                                                                                                                                                                                               \\
    (Values)}        & \macro{wrong}                                             & (\wrong{})  & \macro{cats}                                      & (\cats{}) & \macro{stds}                                      & (\stds{})    \\
                     & \macro{miss}                                              & (\miss{})   & \macro{syns}                                      & (\syns{}) & \macro{metas}                                     & (\metas{})   \\
                     & \macro{contra}                                            & (\contra{}) & \macro{pars}                                      & (\pars{}) & \macro{texts}                                     & (\texts{})   \\
                     & \macro{ambi}                                              & (\ambi{})   &                                                   &           & \macro{papers}                                    & (\papers{})  \\ %\TblrNote{e}                                              \\
                     & \macro{over}\TblrNote{d}                                  & (\over{})   &                                                   &           & \macro{papers*}                                   & (\papers*{}) \\
                     & \macro{redun}                                             & (\redun{})  &                                                   &           &                                                                  \\
    \hline
    \textbf{Used In} & \SetCell[c=2]{c} {\Cref{tab:flawMnfstDefs,tab:flawMnfsts}
        % \\ (in both thesis and paper)
    }                &
                     & \SetCell[c=2]{c} {\Cref{tab:flawDmnDefs,tab:flawDmns}
        % \\ (in both thesis and paper)
    }                &                                                           &
    \SetCell[c=2]{c} {
    \Cref{fig:sourceSummary,fig:flawBars,fig:flawBarsSummary,fig:normFlawBarsSummary}                                                                                                                                             \\
        \Cref{tab:flawMnfsts,tab:flawDmns}
        % \\ \Cref{flaw-analysis-example} (only \macro{stds} and \macro{texts})
    }                &                                                                                                                                                                                                            \\
    \hline
\end{longtblr}

\end{landscape}

However, we create most of these macros for reused text manually when we first
notice the reuse. Some of these macros account for context-specific formatting,
such as capitalization, depending on how they are used; we omit these details
here for brevity. We create macros to reuse many types of information
throughout our documents as shown in \Cref{tab:macrosText}.

% With help from https://tex.stackexchange.com/a/40468/192195, https://tex.stackexchange.com/a/245663/192195, and Copliot
\newcommand\macroType[2]{\multirow[b]{#1}{*}{\adjustbox{minipage=\the\dimexpr 0.3cm * #1 \relax,rotate=90}{#2}}}
% \newcommand\macroType[2]{\multirow[c]{#1}{*}{\rotatebox[origin=c]{90}{#2}}}

\begin{longtblr}[
    note{a} = {Defined in \Cref{flaw-def}.},
    note{b} = {See \Cref{tab:macrosSections} for more details on how we use
            \macro{redunNote} alongside \macro{redun}.},
    caption = {\LaTeX{} macros for reused text.},
    label = {tab:macrosText}
    ]{
    colspec={|Q[c,m]Q[l,m]X[l,m]|},
    row{1} = {halign=c},
    width = \textwidth, rowhead = 1
    }
    \hline
    \thead{Type}             & \thead{Macro}               & \thead{Used in}                                             \\*
    \hline
    \macroType{12}{Flaws\TblrNote{a}}
                             & \macro{bugPattonFlaw}       & \Cref{intro} and \flawref{bug-patton}                       \\*
                             & \macro{alphaFlaw}           & \Cref{intro} and \flawref{alpha-def}                        \\*
                             & \macro{loadFlaw}            & \Cref{intro} and \flawref{load-def}                         \\*
                             & \macro{expBasedCatMain}     & \Cref{intro,multiCats}                                      \\*
                             & \macro{tourFlaw}            & \Cref{intro,flaws} and \flawref{tour-def}                   \\*
                             & \macro{redBoxFlaw}          & \Cref{explicitness,wrong} and \flawref{dubious-red-box-syn} \\*
                             & \macro{perfAsFamily}        & \Cref{method-family,classFamilyFlaw}                        \\*
                             & \macro{tolTestFlaw}         & \Cref{less-cred-assert,wrong} and \flawref{assert-truth}    \\*
                             & \macro{accelTolTest}        & \macro{tolTestFlaw} and \Cref{hard-test}                    \\*
                             & \macro{errorGuessFlaw}      & \Cref{wrong} and \flawref{error-guess}                      \\*
                             & \macro{parSheetTestFlaw}    & \Cref{wrong} and \flawref{par-sheet-test}                   \\*
                             & \macro{perfSecParFlaw}      & \flawref{perf-sec-par} and paper version of \Cref{pars}     \\
    \hline
    \macroType{7}{Footnotes} & \macro{ftrnote}             & \SetCell[r=3]{l} Thesis (automated) and paper (manual)
    versions of \Cref{tab:parSyns}                                                                                       \\*
                             & \macro{specfn}              &                                                             \\*
                             & \macro{ucstn}               &                                                             \\*
    \cline{2-3}              & \macro{redunNote}           & \macro{redun} and \Cref{tab:flawMnfstDefs}\TblrNote{b}      \\*
    \cline{2-3}              & \macro{notDefDistinctIEEE}  & \flawref{static-test-flaw} and \Cref{exist-tax}             \\*
                             & \macro{gerrardDistinctIEEE} & \Cref{tab:otherCategorizations} and
    \flawref{gerrard-distinct}                                                                                           \\*
                             & \macro{distinctIEEE}        & \macro{gerrardDistinctIEEE}, \macro{notDefDistinctIEEE},
    and \Cref{method-family,classFamilyFlaw}                                                                             \\
    \hline
    \macroType{3}{Links}     & \macro{ourApproachGlossary} &
    \Cref{tab:approachGlossaryExcerpt,explicitness,record-terms,imp-info,app-rel-vis,%
    auto-flaw-analysis,aug-flaw-analysis,oat-test-rec}                                                                   \\*
                             & \macro{seeSrcCode}          & \Cref{app-rel-vis,auto-flaw-analysis}                       \\*
    % paper-macros
                             & \macro{recFigs}             & \Cref{flaws,recs}                                           \\
    \hline
    \macroType{3}{RQs}       & \macro{rqatext}             & \SetCell[r=3]{l} \Cref{intro} and seminar slides            \\*
                             & \macro{rqbtext}             &                                                             \\*
                             & \macro{rqctext}             &                                                             \\
    \hline
    \macroType{12}{Misc.}    & \macro{supersAck}           & \nameref{acknowledgements} and seminar slides               \\*
                             & \macro{supers}              & \nameref{decl_aca_ach} and \macro{supersAck}                \\*
                             & \macro{highLvlScope}        & \Cref{scope-overview,stds}                                  \\*
                             & \macro{defRel}              & \Cref{syn-rels,par-chd-rels}                                \\*
                             & \macro{defLabelDistinct}    & \Cref{label-flaw-def,labels}                                \\*
                             & \macro{oneSrcDistinct}      & \Cref{one-src-flaws,aug-flaw-analysis}                      \\*
                             & \macro{approachFields}      & \Cref{explicitness,record-terms}                            \\*
                             & \macro{impKeywords}         & \Cref{explicitness,imp-info}                                \\*
                             & \macro{orthTestIntro}       & \Cref{infers,orth-test}                                     \\*
                             & \macro{listAllSrcs}         & \Cref{source-tiers,ident-sources}                           \\*
                             & \macro{addTextEx}           & \Cref{texts,ident-sources}                                  \\*
                             & \macro{displayNL}           &
    \Cref{app-rel-vis,aug-flaw-analysis}                                                                                 \\
    % paper-macros,tab:macrosPaper
    \hline
\end{longtblr}


\phantomsection{}\label{paper-macros}
In addition to this thesis, we also prepare a conference paper based on our
research. While we can reuse most content without modifying it, there are some
formatting differences between the two document types. For example, our thesis
uses the \texttt{natbib} package for citations while the IEEE guidelines for
paper submissions suggest the use of \texttt{cite} \citep[p.~8]{Shell2015};
we define aliases so that we can reuse text that includes citations
\seeSrcCode{82167b7}{paper_preamble.tex}{19}{60}.

In general, we use the command\todo{Is this a ``command''?}
\texttt{\textbackslash ifnotpaper} to allow for manual distinctions between the
two documents' formats, such as how they handle citations, using this basic format:
\begin{displayquote}
    \texttt{\textbackslash ifnotpaper <thesis code> \textbackslash else <paper code> \textbackslash fi}
\end{displayquote}
For example, in \Cref{nonIEEE-sources}, we provide a list of non-IEEE sources
that support a claim made by the IEEE. Since we sort sources based on
trustworthiness (defined in \Cref{trust}), publication year, and number of
authors, the relevant thesis code is:
\begin{displayquote}
    \texttt{(\textbackslash citealp[pp.\textasciitilde 5\textbackslash =/6 to 5\textbackslash =/7]\{SWEBOK2024\};
        \displayNL \textbackslash citealpISTQB\{\};
        \textbackslash citealp[pp.\textasciitilde 807\textbackslash ==808]\{Perry2006\};
        \displayNL \textbackslash citealp[pp.\textasciitilde 443\textbackslash ==445]\{PetersAndPedrycz2000\};
        \displayNL \textbackslash citealp[p.\textasciitilde 218]\{KuļešovsEtAl2013\}\textbackslash todo\{OG Black, 2009\};
        \displayNL \textbackslash citealp[pp.\textasciitilde 9, 13]\{Gerrard2000a\})}
\end{displayquote}
Meanwhile, IEEE guidelines prefer that sources are kept in separate brackets,
sorted in order of their first appearance in the document. Therefore, paper
code for this list of sources is:
\begin{displayquote}
    \texttt{\textbackslash cite[pp.\textasciitilde 443\textbackslash ==445]\{PetersAndPedrycz2000\},
        \displayNL \textbackslash cite[pp.\textasciitilde 5\textbackslash =/6 to 5\textbackslash =/7]\{SWEBOK2024\},
        \textbackslash cite\{ISTQB\},
        \displayNL \textbackslash cite[pp.\textasciitilde 807\textbackslash ==808]\{Perry2006\},
        \displayNL \textbackslash cite[pp.\textasciitilde 9, 13]\{Gerrard2000a\},
        \displayNL \textbackslash cite[p.\textasciitilde 218]\{KuļešovsEtAl2013\}}
\end{displayquote}\utd{}
In particular, note the usage of the \macro{cite} command, the \emph{lack} of
use of the custom alias for citing the \acs{istqb} glossary, the different
order, and the lack of the \macro{todo}, since these are only rendered for
reference in the thesis. For other edge cases with different formatting styles
for these document types, we define the macros given in \Cref{tab:macrosPaper}
via an example usage of the macro(s) and how the code is rendered in our thesis
\emph{and} paper\footnote{Since this document is the thesis version, some of
    the paper renderings are hardcoded.}\todo{Does this footnote make sense?}:

\def\refHelperEx{\refHelper \ifnotpaper \citet{IEEE2022}
    \else \defcitealias{IEEE2022}{16}[\citetalias{IEEE2022}]
    \fi \multiAuthHelper{form} the basis of this \docType{}.}

\begin{longtblr}[
    note{a} = {Section omitted for brevity.},
    caption = {\LaTeX{} macros for handling formatting differences between thesis and paper.},
    label = {tab:macrosPaper}
    ]{
    colspec={|Q[c,m]|X[l,m]|},
    column{1} = {font=\bfseries},
    width = \textwidth, rowhead = 1
    }
    \hline
    \thead{Context} & \thead{Displayed as}                                                \\
    \hline
    Code            & {\texttt{\macro{refHelper} \macro[IEEE2022]{citet}\
    \macro[form]{multiAuthHelper}} \displayNL\texttt{the basis of this \macro{docType}.}} \\*
    Thesis          & \refHelperEx{}                                                      \\*
    Paper           & {\notpaperfalse \refHelperEx{}}                                     \\
    \hline
    Code            & \macro[cat-acro]{flawref}                                           \\*
    Thesis          & \flawref{cat-acro}                                                  \\*
    Paper           & Section \hyperref[cat-acro]{III-B2}                                 \\
    \hline
    Code            & \macro{redun}                                                       \\*
    Thesis          & \redun{}                                                            \\*
    Paper           & Redundancies\TblrNote{a}                                            \\
    \hline
\end{longtblr}

