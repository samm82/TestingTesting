\section{Future Work}\label{future-work}

Ideally, our work would have resulted in a complete taxonomy of software
testing terminology based on the literature. Unfortunately, we were not able to
fully accomplish this due to time constraints. With more time, we would
continue iterating over undefined terms (\Cref{future-undef-terms}) and
investigate terms we \emph{expected} to find but never did
(\Cref{future-miss-terms}). Additionally, there is much more work we could do
to analyze our data on the software testing literature, such as detecting
more classes of flaws (and identifying them!) (\Cref{future-detect-flaws}).

\subsection{Iterating Over Undefined Terms}\label{future-undef-terms}

As we explain in \Cref{undef-terms}, our methodology includes performing
miniature literature reviews on undefined terms to record their
missing definitions (and any relations). We were able to do this for the
following approaches, although some are out of scope, such as \acf{emsec}
testing, aspects of \acf{orthat} and loop testing (see \Cref{hard-test}),
and HTML testing (see \Cref{lang-test}). We investigate the following terms
(and their respective related terms) in the sources given:
\input{build/undefTerms}

Applying our procedure shown in \Cref{fig:recAppFlowchart} to these sources
uncovers \the\numexpr \TotalAfter - \TotalBefore\relax\ new approaches and
\the\numexpr \TotalAfter - \UndefAfter - \TotalBefore + \UndefBefore\relax\ new
definitions. These definitions are either for existing undefined approaches or
new uncovered approaches; while not every new approach is presented alongside
a definition, if we assume that each of these definitions is for a new approach,
we can deduce that about \the\numexpr 100 - 100 * (\UndefAfter - \UndefBefore) /
(\TotalAfter - \TotalBefore)\relax\% of added test approaches are defined. This
indicates that this procedure leads to a higher proportion of defined terms
(\the\numexpr 100 - 100 * \UndefBefore / \TotalBefore\relax\% vs.~%
\the\numexpr 100 - 100 * \UndefAfter / \TotalAfter\relax\%), as shown in
\Cref{fig:undefPies}, which helps verify that our procedure constructively
uncovers \emph{and} defines new terminology. With repeated iterations, this
ratio would approach 100\%, resulting in a (plausibly) complete taxonomy.

\begin{figure*}[hbtp!]
    \begin{subfigure}[c]{0.35\linewidth}
        \beforeIterGraph{0.7}
        \caption{The \the\TotalBefore{} approaches before investigating undefined terms.}
        \label{fig:undefPiesBefore}
    \end{subfigure}
    \hfill
    \begin{subfigure}[c]{0.35\linewidth}
        \afterIterGraph{0.7}
        \caption{The \the\TotalAfter{} approaches after investigating undefined terms.}
        \label{fig:undefPiesAfter}
    \end{subfigure}
    \hfill
    \begin{subfigure}[c]{0.2\linewidth}
        \iterLegend{1}
    \end{subfigure}
    \caption{Breakdown of how many test approaches are undefined.}
    \label{fig:undefPies}
\end{figure*}

\clearpage
\subsection{Missing Terms}\label{future-miss-terms}
In addition to these undefined terms, some terms do not appear in
the literature at all! While most test approaches arise as a result of our
snowballing approach, we each have preexisting knowledge of what test
approaches exist (a form of experience-based testing, if you will).
Test approaches that arise independently of snowballing may
serve as starting points for continued research if we do not find them in
the literature using our iterative approach. The following terms come from
previous knowledge, conversations with colleagues, research for other
projects, or ad hoc cursory research to see what other test approaches exist:
\newline

\begin{minipage}{0.92\textwidth}
    \centering
    \begin{multicols}{2}
        \begin{enumerate}
            \item Chaos engineering
            \item Chosen-ciphertext attacks
            \item Concolic testing
            \item Concurrent testing\footnote{This seems to be distinct from
                      ``concurrency testing''.}
            \item Context-driven testing
            \item Destructive testing
            \item Dogfooding
            \item Implementation-based testing\footnote{This may or may not be
                      distinct from ``implementation-oriented testing.''}
            \item Interaction-based testing
            \item Lunchtime attacks\footnote{In previous meetings, Dr.~Smith
                      mentioned that with the number of test approaches that
                      suggest that people just like to label everything as
                      ``testing'', he would not be surprised if something
                      like ``Monday morning testing'' existed. While
                      independently researching chosen-ciphertext attacks
                      out of curiosity, this prediction of a time-based
                      test approach came true with ``lunchtime attacks''.}
            \item Parallel testing
            \item Property-based testing
            \item Pseudo-random bit testing
            \item Rubber duck testing
            \item Sanity testing
            \item Scream testing
            \item Shadow testing
            \item Situational testing
        \end{enumerate}
    \end{multicols}
\end{minipage}

\subsection{Detecting More Flaws}\label{future-detect-flaws}
In addition to the classes of flaws we \emph{do} detect automatically, we could
detect many more if time permitted. We currently detect parent-child relations
that violate irreflexivity (see \Cref{autoSelfPar}) which can be thought of as
cycles with length $n=1$. Since parent-child relations should also be
transitive (see \Cref{par-chd-rels}), cycles of \emph{any} size are flaws.
Given the current way we generate visualizations of these relations (see
\Cref{app-rel-vis}), detecting cycles where $n=2$ would be straightforward: if
a parent-child relation \emph{and} its inverse (i.e., \texttt{A~->~B} and
\texttt{B~->~A} for test approaches with labels \texttt{A} and \texttt{B}) both
exist in the generated \LaTeX{} file for a visualization, (at least) one of
these parent-child flaws is incorrect since they contradict each other and we
have found a cycle. The main reason this would be time-consuming would be
deciding on how to format these findings, writing code to do so automatically
for use in this \docType{}, and resolving any issues that arise during this
process. Detecting larger cycles would also be possible but would likely
require the use of an additional tool to analyze the graph of parent-child
relations.
