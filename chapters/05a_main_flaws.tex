\begin{enumerate}
    \item % Flaw count (CONTRA, PARS): {IEEE2022} {Firesmith2015} | {IEEE2017} {SWEBOK2024} {BarbosaEtAl2006} | {vanVliet2000}
          Regression testing and retesting are sometimes given as two distinct
          approaches \ifnotpaper
              (\citealp[p.~8]{IEEE2022}; \citealp[p.~34]{Firesmith2015})%
          \else
              \cite[p.~8]{IEEE2022}, \cite[p.~34]{Firesmith2015}%
          \fi, but sometimes regression testing is defined as a form of
          ``selective retesting'' \ifnotpaper (\citealp[p.~372]{IEEE2017};
              \citealp[pp.~5-8, 6-5, 7-5 to 7-6]{SWEBOK2024};
              \citealp[p.~3]{BarbosaEtAl2006})%
          \else \cite[p.~372]{IEEE2017},
              \cite[pp.~5-8, 6-5, 7-5 to 7-6]{SWEBOK2024},
              \cite[p.~3]{BarbosaEtAl2006}%
          \fi. Moreover, the two possible variations
          of regression testing given by \citet[p.~411]{vanVliet2000} are
          ``retest-all'' and ``selective retest''\todo{Are these separate
              approaches?}, which is possibly the source
          of the above misconception. This creates a cyclic
          relation between regression testing and selective retesting.
    \item % Flaw count (OVER, DEFS): {IEEE2022} | {IEEE2022}
          \refHelper \citet[p.~34]{IEEE2022} \multiAuthHelper{give} the
          ``landmark tour'' as an example of ``a tour used for exploratory
          testing'', but they also use the analogy of ``a tour guide lead[ing]
          a tourist through the landmarks of a big city'' to describe tours in
          general. Is the distinction between them the fact that landmark tours
          are pre-planned and follow a decided-upon sequence \citetext{p.~34}?
    \item % Flaw count (MISS, DEFS): {IEEE2022} {IEEE2021}
          Integration testing, system testing, and system integration testing
          are all listed as ``common test levels'' \ifnotpaper
              \citetext{\citealp[p.~12]{IEEE2022}; \citeyear[p.~6]{IEEE2021}}%
          \else
              \cite[p.~12]{IEEE2022}, \cite[p.~6]{IEEE2021}%
          \fi, but no
          definitions are given for the latter two, making it unclear what
          ``system integration testing'' is; it is a combination of the two?
          somewhere on the spectrum between them? It is listed as a child
          % Flaw count (CONTRA, PARS): ISTQB | {Firesmith2015}
          of integration testing by \citetISTQB{}
          and of system testing by \citet[p.~23]{Firesmith2015}.
    \item % Flaw count (MISS, DEFS): {IEEE2017}
          Similarly, component testing, integration testing, and component
          integration testing are all listed in \citep{IEEE2017}, but ``component
          integration testing'' is only defined as ``testing of groups of
          related components'' \citep[p.~82]{IEEE2017}; it is a combination of
          the two? somewhere on the spectrum between them? As above, it is
          listed as a child of integration testing by \citetISTQB{}.
    \item % Flaw count (WRONG, LABELS): {IEEE2021} {SWEBOK2024} {SWEBOK2014}
          % Flaw count (CONTRA, DEFS): {IEEE2021} | {SWEBOK2024} {SWEBOK2014}
          % Ground truth: {IEEE2010} {SWEBOK2024} {vanVliet2000}
          Since errors are distinct from defects/faults \ifnotpaper
              (\citealp[pp.~128, 140]{IEEE2010}; \citealp[p.~12\=/3]{SWEBOK2024};
              \citealp[pp.~399--400]{vanVliet2000})\else
              \cite[p.~12\=/3]{SWEBOK2024}, \cite[pp.~128, 140]{IEEE2010},
              \cite[pp.~399--400]{vanVliet2000}\fi, error guessing should
          instead be called ``defect guessing'' if it is based on a ``checklist
          of potential defects'' \citep[p.~29]{IEEE2021} or ``fault guessing''
          if it is a ``fault-based technique'' \citep[p.~4\=/9]{SWEBOK2014}
          that ``anticipate[s] the most plausible faults in each \acs{sut}''
          \citep[p.~5\=/13]{SWEBOK2024}. One (or both) of these proposed terms
          may be useful in tandem with ``error guessing'', which would focus on
          errors as traditionally defined; this would be a subapproach of
          error-based testing (implied by \citealp[p.~399]{vanVliet2000}).
    \item % Flaw count (WRONG, LABELS): {IEEE2017} {Firesmith2015} {vanVliet2000}
          % Flaw count (WRONG, SYNS): {IEEE2017} {vanVliet2000}
          % Ground truth: {IEEE2010} {SWEBOK2024} {vanVliet2000}
          Similarly, ``fault seeding'' is not a synonym of ``error seeding''
          as claimed by \citet[p.~165]{IEEE2017} and
          % TODO: may need to be reordered via \ifnotpaper
          \citet[p.~427]{vanVliet2000}. The term ``error seeding'', also
          used by \citet[p.~34]{Firesmith2015},
          should be abandoned in favour of ``fault seeding'', as it is defined
          as the ``process of intentionally adding known faults to those
          already in a computer program \dots{} [to] estimat[e] the number of
          faults remaining'' \citep[p.~165]{IEEE2017} based on the ratio
          between the number of new faults and the number of introduced faults
          that were discovered \citep[p.~427]{vanVliet2000}.
    \item % Flaw count (CONTRA, DEFS): {IEEE2017} {SWEBOK2024} {Firesmith2015} {AmmannAndOffutt2017} {PetersAndPedrycz2000} | {IEEE2022} {Gerrard2000a}
          % Label DEFS static-test-flaw 
          ``Software testing'' is often defined to exclude static testing
          \ifnotpaper
              (\citealp[p.~13]{Firesmith2015}; \citealp[p.~222]{AmmannAndOffutt2017};
              \citealp[p.~439]{PetersAndPedrycz2000})%
          \else
              \cite[p.~439]{PetersAndPedrycz2000}, \cite[p.~13]{Firesmith2015},
              \cite[p.~222]{AmmannAndOffutt2017}%
          \fi, restricting ``testing'' to mean dynamic validation
          \citep[p.~5\=/1]{SWEBOK2024} or verification ``in which a system or
          component is executed'' \citep[p.~427]{IEEE2017}. However,
          ``terminology is not uniform among different communities, and some
          use the term `testing' to refer to static techniques%
          \notDefDistinctIEEE{technique} as well'' \citep[p.~5\=/2]{SWEBOK2024}.
          This is done by \citet[pp.~16\==17]{IEEE2022} and
          \citet[pp.~8--9]{Gerrard2000a}; the \ifnotpaper \else authors of the
          \fi former even explicitly \emph{exclude} static testing in another
          document \citeyearpar[p.~440]{IEEE2017}!
    \item % Flaw count (CONTRA, SYNS): {IEEE2017} ISTQB | {IEEE2017} {BaresiAndPezzè2006}
          A component is an ``entity with discrete structure \dots\ within a
          system considered at a particular level of analysis''
          \citep{ISO_IEC2023b} and ``the terms module, component, and unit
              [sic] are often used interchangeably or defined to be subelements
          of one another in different ways depending upon the context'' with
          no standardized relationship \citep[p.~82]{IEEE2017}. For example,
          \citetISTQB{} \multiAuthHelper{define} them as synonyms while
          \citet[p.~107]{BaresiAndPezzè2006} \multiAuthHelper{say} ``components
          differ from classical modules for being re-used in different contexts
          independently of their development''. Additionally, since components
          are structurally, functionally, or logically discrete
          \citep[p.~419]{IEEE2017} and ``can be tested in isolation''
          \citepISTQB{}, ``unit/component/module testing'' could refer to the
          testing of both a module \emph{and} a specific function in a module%
          \thesisissueref{14}, introducing a further level of ambiguity.
          % Flaw count (AMBI, SYNS): {IEEE2017} ISTQB
    \item % Flaw count (OVER, SYNS): {IEEE2017} {IEEE2013} | {IEEE2022} {IEEE2017} {Perry2006}
          % Label SYNS level-phase-syns
          \citeauthor{IEEE2017} say that ``test level'' and ``test phase''
          are synonyms, both meaning a ``specific instantiation of [a] test
          sub-process'' \ifnotpaper
              (\citeyear[pp.~469, 470]{IEEE2017}; \citeyear[p.~9]{IEEE2013})%
          \else
              \cite[pp.~469, 470]{IEEE2017}, \cite[p.~9]{IEEE2013}%
          \fi, but they have other definitions as well. ``Test level'' can also
          refer to the scope of a test process; for example, ``across the whole
          organization'' or only ``to specific projects''
          \citeyearpar[p.~24]{IEEE2022} and ``test phase'' can also refer to
          the ``period of time in the software life cycle'' when testing occurs
          \citeyearpar[p.~470]{IEEE2017}, usually after the implementation phase
          \ifnotpaper
              (\citeyear[pp.~420, 509]{IEEE2017}; \citealp[p.~56]{Perry2006})%
          \else
              % \cite[pp.~420, 509]{IEEE2017}
              \citetext{pp.~420, 509}, \cite[p.~56]{Perry2006}%
          \fi.
    \item % Flaw count (OVER, DEFS): {IEEE2010} | {IEEE2010}
          \citeauthor{IEEE2010} define ``error'' as ``a human action that
          produces an incorrect result'', but also as ``an incorrect result''
          itself \citeyearpar[p.~128]{IEEE2010}. Since faults are inserted when
          a developer makes an error \ifnotpaper
              (\citeyear[pp.~128, 140]{IEEE2010};
              \citealp[p.~12\=/3]{SWEBOK2024};
              \citealp[pp.~399--400]{vanVliet2000})\else
              \cite[p.~12\=/3]{SWEBOK2024}, \cite[pp.~128, 140]{IEEE2010},
              \cite[pp.~399--400]{vanVliet2000}\fi, this means that they are
          ``incorrect results'', making ``error'' and ``fault'' synonyms and
          the distinction between them less useful.
    \item % Flaw count (OVER, DEFS): {IEEE2010} {SWEBOK2024}
          Additionally, ``error'' can also be defined as ``the
          difference between a computed, observed, or measured value or
          condition and the true, specified, or theoretically correct value
          or condition'' \ifnotpaper (\citealp[p.~128]{IEEE2010}; similar in
              \citealp[pp.~17\=/18 to 17\=/19, 18\=/7 to 18\=/8]{SWEBOK2024})%
          \else \cite[p.~128]{IEEE2010} (similar in
              \cite[pp.~17\=/18 to 17\=/19, 18\=/7 to 18\=/8]{SWEBOK2024})\fi.
          While this is a widely used definition, particularly in mathematics,
          it makes some test approaches ambiguous; for example, back-to-back
          testing is ``testing in which two or more variants of a program are
          executed with the same inputs, the outputs are compared, and errors
          are analyzed in case of discrepancies'' \ifnotpaper
              (\citealp[p.~30]{IEEE2010}; similar in \citealpISTQB{})\else
              \cite[p.~30]{IEEE2010} (similar in \cite{ISTQB})\fi,
          % Flaw count (AMBI, DEFS): {IEEE2010} {SWEBOK2024} | {IEEE2010} ISTQB
          which seems to refer to this definition of ``error''.

          % STD | META
    \item % Flaw count (WRONG, CATS): {IEEE2022} {IEEE2021} {IEEE2017} | ISTQB
          \refHelper \citetISTQB{} \ifnotpaper classify \else classifies \fi
          \acs{ml} model testing as a test level, which \ifnotpaper they \else
              it \fi \multiAuthHelper{define} as ``a specific instantiation of a
          test process'': a vague definition that does not match the one in
          \Cref{tab:ieeeCats}.
    \item % Flaw count (CONTRA, PARS): {ISO_IEC2023a} | {Firesmith2015}
          % Label PARS perf-sec-par
          \perfSecParFlaw{}
    \item % Flaw count (CONTRA, PARS): {IEEE2022} {IEEE2021} {SWEBOK2024} ISTQB | {Firesmith2015}
          Similarly, random testing is a subtechnique of specification-based
          testing \ifnotpaper
              \citetext{\citealp[pp.~7, 22]{IEEE2022};
                  \citeyear[pp.~5, 20, Fig.~2]{IEEE2021};
                  \citealp[p.~5\=/12]{SWEBOK2024}; \citealpISTQB{}} \else
              \cite[pp.~7, 22]{IEEE2022}, \cite{ISTQB},
              \cite[p.~5\=/12]{SWEBOK2024}, \cite[pp.~5, 20, Fig.~2]{IEEE2021}
          \fi but is listed
          separately by \citet[p.~46]{Firesmith2015}.
    \item % Flaw count (OVER, DEFS): {SWEBOK2024} | {IEEE2022}
          The \acs{swebok} V4 defines ``privacy testing'' as testing that
          ``assess[es] the security and privacy of users' personal data to
          prevent local attacks'' \citep[p.~5-10]{SWEBOK2024}; this seems to
          overlap (both in scope and name) with the definition of ``security
          testing'' in \citep[p.~7]{IEEE2022}: testing
          ``conducted to evaluate the degree to which a test item, and
          associated data and information, [sic] are protected so that'' only
          ``authorized persons or systems'' can use them as intended.
    \item % Flaw count (CONTRA, DEFS): {SWEBOK2024} {Patton2006} | {IEEE2017}
          % Label DEFS path-test
          Path testing ``aims to execute all entry-to-exit control flow paths
          in a \acs{sut}'s control flow graph'' \ifnotpaper
              (\citealp[p.~5-13]{SWEBOK2024}; similar in
              \citealp[p.~119]{Patton2006})\else \cite[p.~5-13]{SWEBOK2024}
              (similar in \cite[p.~119]{Patton2006})\fi, but
          \citet[p.~316]{IEEE2017} \multiAuthHelper{add} that it can also be
          ``designed to execute \dots{} selected paths.''
    \item % Flaw count (CONTRA, DEFS): {IEEE2022} | ISTQB
          % Label DEFS tour-def
          % Label CONTRA tour-def-contra
          \tourFlaw{}
    \item % Flaw count (CONTRA, DEFS): {IEEE2017} | {SWEBOK2024} | ISTQB
          % Label DEFS alpha-def
          \alphaFlaw{}
    \item % Flaw count (CONTRA, SYNS): ISTQB | {IEEE2022}
          % Label SYNS use-case-scenario
          % IEEE2021 omitted to avoid double-counting in parSyns table
          ``Use case testing'' is given as a synonym of ``scenario testing''
          by \citetISTQB{}
          %   Old; based on inconsistent citations from Kam2008/ISTQB
          %   and \citet[pp.~47--49]{Kam2008} \ifnotpaper (see
          %       \Cref{fig:threeWaySyns}) \fi
          but listed separately by
          \citet[Fig.~2]{IEEE2022} and described as a ``common form of scenario
          testing'' in \citeyearpar[p.~20]{IEEE2021}\todo{OG Hass, 2008}.
          This implies that use case testing may instead be a child of
          user scenario testing (see \Cref{tab:parSyns}).
    \item % Flaw count (AMBI, LABELS): {IEEE2017} | {Firesmith2015} | {Firesmith2015} {Gerrard2000a}
          The distinctions between development testing \citep[p.~136]{IEEE2017},
          developmental testing \citep[p.~30]{Firesmith2015}, and developer
          testing
          \ifnotpaper
              (\citealp[p.~39]{Firesmith2015}; \citealp[p.~11]{Gerrard2000a})
          \else
              \cite[p.~39]{Firesmith2015}, \cite[p.~11]{Gerrard2000a}
          \fi are unclear and seem miniscule.\todo{Is this a def flaw?}
    \item % Flaw count (AMBI, DEFS): ISTQB
          % TODO: SUPP DEFS?
          \refHelper \citetISTQB{} \multiAuthHelper{define}
          ``\acf{ml} model testing'' and ``\acs{ml} functional performance''
          in terms of ``\acs{ml} functional performance criteria'',
          which is defined in terms of ``\acs{ml} functional performance
          metrics'', which is defined as ``a set of measures that relate to the
          functional correctness of an \acs{ml} system''. The use
          of ``performance'' (or ``correctness'') in these definitions is at
          best ambiguous and at worst incorrect.
    \item % Flaw count (CONTRA, SYNS): ISTQB implied by {Gerrard2000a} | {SWEBOK2024}
          % Label SYNS stage-level-syns
          The terms ``test level'' and ``test stage'' are given as synonyms
          (\citealpISTQB{}; implied by \citealp[p.~9]{Gerrard2000a}), but
          \citet[p.~5\=/6]{SWEBOK2024} says ``[test] levels can be distinguished
          based on the object of testing, the \emph{target}, or on the purpose
          or \emph{objective}'' and calls the former ``test stages'', giving
          the term a child relation (see \Cref{par-chd-rels}) to ``test level''
          instead. However, the examples listed---unit testing, integration
          testing, system testing, and acceptance testing
          \citep[pp.~5\=/6 to 5\=/7]{SWEBOK2024}---are commonly categorized as
          ``test levels'' (see \Cref{cats-def}).
    \item % Flaw count (WRONG, LABELS): {Firesmith2015}
          % Ground truth: {LiuEtAl2023} {MorgunEtAl1999} {HolleyEtAl1996} {HoweAndJohnson1995}
          % Label LABELS ground-truth
          \tolTestFlaw{}
    \item % Flaw count (OVER, LABELS): {SWEBOK2024} implied by {Valcheva2013} {YuEtAl2011} | {Firesmith2015}
          ``Orthogonal array testing'' \ifnotpaper \citetext{%
                  \citealp[pp.~5\=/1, 5\=/11]{SWEBOK2024};
                  implied by \citealp[pp.~467, 473]{Valcheva2013};
                  \citealp[pp.~1573\==1577, 1580]{YuEtAl2011}} \else
              \cite[pp.~5\=/1, 5\=/11]{SWEBOK2024} \fi and ``operational
          acceptance testing'' \citep[p.~30]{Firesmith2015} have the same
          acronym (``OAT'').
    \item % Flaw count (CONTRA, SYNS): ISTQB | {Firesmith2015}
          % Label SYNS oat-pat-syns
          ``Operational acceptance testing'' and ``production acceptance
          testing'' are given as synonyms by \citetISTQB{} but listed
          separately by \citet[p.~30]{Firesmith2015}.

          % META | TEXT
    \item % Flaw count (AMBI, LABELS): {IEEE2022} {IEEE2021} {IEEE2017} {SWEBOK2024} ISTQB {KuļešovsEtAl2013} {Perry2006} {PetersAndPedrycz2000} {Gerrard2000a} | {vanVliet2000} implied by {SWEBOK2024}
          ``Installability testing'' is given as a test type
          \ifnotpaper
              (\citealp[p.~22]{IEEE2022}; \citeyear[p.~38]{IEEE2021};
              \citeyear[p.~228]{IEEE2017})%
          \else
              \cite[p.~22]{IEEE2022}, \cite[p.~38]{IEEE2021}, \cite[p.~228]{IEEE2017}%
          \fi, while ``installation testing'' is given as a test level
          \ifnotpaper
              (\citealp[p.~439]{vanVliet2000}; implied by
              \citealp[p.~5-8]{SWEBOK2024})%
          \else
              \cite[p.~439]{vanVliet2000}%
          \fi. Since ``installation testing'' is not given as an example of a
          test level throughout the sources that describe them (see
          \Cref{cats-def}), it is likely that the term ``installability
          testing'' with all its related information should be used instead.
          % Previous: Flaw count (CONTRA, CATS): {IEEE2022} {IEEE2021} | {vanVliet2000} implied by {SWEBOK2024}
    \item % Flaw count (CONTRA, PARS): implied by {Patton2006} | {vanVliet2000}
          While \citet[p.~120]{Patton2006} implies that condition testing is a
          subtechnique of path testing, \citet[Fig.~13.17]{vanVliet2000} says
          that multiple condition coverage (which seems to be a synonym of
          condition coverage \citetext{p.~422}) does not subsume and is not
          subsumed by path coverage.
    \item % Flaw count (WRONG, SYNS): {IEEE2010} {SWEBOK2024} {vanVliet2000} | {Patton2006}
          % Label SYNS bug-patton
          The differences between the terms ``error'', ``failure'',
          ``fault'', ``defect'' are significant and meaningful \ifnotpaper
              (\citealp[pp.~128, 139--140]{IEEE2010};
              \citealp[p.~12\=/3]{SWEBOK2024};
              \citealp[pp.~399--400]{vanVliet2000})\else
              \cite[pp.~128, 139--140]{IEEE2010},
              \cite[pp.~399--400]{vanVliet2000},
              \cite[p.~12\=/3]{SWEBOK2024}\fi, but \bugPattonFlaw{}
    \item % Flaw count (CONTRA, DEFS): {IEEE2022} | {Patton2006}
          % Label DEFS load-def
          \loadFlaw{}
    \item % Flaw count (CONTRA, DEFS): {IEEE2021} | {Patton2006}
          State testing requires that ``all states in the state model
          \dots\ [are] `visited'\,'' in \citep[p.~19]{IEEE2021} which
          is only one of its possible criteria in \citep[pp.~82-83]{Patton2006}.
    \item % Flaw count (CONTRA, DEFS): {IEEE2017} {PetersAndPedrycz2000} {vanVliet2000} | {Patton2006}
          \refHelper \citet[p.~456]{IEEE2017} \multiAuthHelper{say} system
          testing is ``conducted on a complete, integrated system'' (which
          \citet[Tab.~12.3]{PetersAndPedrycz2000} and
          \citet[p.~439]{vanVliet2000} agree with), while
          \citet[p.~109]{Patton2006} says it can also be done on ``at least a
          major portion'' of the product.
    \item % Flaw count (AMBI, SYNS): ISTQB | {Patton2006}
          % TODO: NEAR SYNS?
          \refHelper \citetISTQB{} \multiAuthHelper{claim} that code inspections
          are related to peer reviews but \citet[pp.~94--95]{Patton2006} makes
          them quite distinct.
    \item % Flaw count (CONTRA, SYNS): ISTQB | {PetersAndPedrycz2000}
          % Label SYNS walkthrough-syns
          ``Walkthroughs'' and ``structured walkthroughs'' are given
          as synonyms by \citetISTQB{} but \citet[p.~484]{PetersAndPedrycz2000}
          \ifnotpaper imply \else implies \fi that they are different, saying a
          more structured walkthrough may have specific roles.
    \item % Flaw count (WRONG, SYNS): {PetersAndPedrycz2000} | {PetersAndPedrycz2000}
          \refHelper \citet[p.~447]{PetersAndPedrycz2000}
          \multiAuthHelper{claim} that ``structural testing subsumes white box
          testing'' but the two terms seem to describe the same thing:
          \ifnotpaper they say \else it says \fi ``structure tests are aimed at
          exercising the internal logic of a software system'' and ``in white box
          testing \dots, using detailed knowledge of code, one creates a battery of
          tests in such a way that they exercise all components of the code
          (say, statements, branches, paths)'' on the same page!
    \item % Flaw count (CONTRA, DEFS): {vanVliet2000} | implied by {Patton2006}
          \refHelper \citet[p.~92\ifnotpaper, emphasis added\fi]{Patton2006}
          says that reviews are ``\emph{the} process[es] under which static
          white-box testing is performed'' but correctness proofs are given
          as another example by \citet[pp.~418--419]{vanVliet2000}.

          % TEXT | PAPER
    \item % Flaw count (WRONG, DEFS): ISTQB | {Kam2008}
          \refHelper \citet[p.~46]{Kam2008}\todo{OG Beizer} says that the goal
          of negative testing is ``showing that a component or system does not
          work'' which is not true; if robustness is an important quality for
          the system, then testing the system ``in a way for which it was not
          intended to be used'' \citepISTQB{} (i.e., negative testing) is one
          way to help test this!
    \item % Flaw count (WRONG, TRACE): {Kam2008}
          % Flaw count (MISS, DEFS): {Kam2008}
          % Label TRACE see-ref-missing
          \refHelper \citet[p.~42]{Kam2008} says ``See \emph{boundary value
              analysis},'' for the glossary entry of ``boundary value testing''
          but does not include ``boundary value analysis'' in the glossary.
    \item % Flaw count (CONTRA, CATS): {Kam2008} | implied by {IEEE2021}
          \refHelper \citet[p.~46]{Kam2008}\todo{OG Beizer} says ``negative
          testing is related to the testers' attitude rather than a specific
          test approach or test design technique''; while \citet{IEEE2021}
          \multiAuthHelper{seem} to support this idea of negative testing being
          at a ``higher'' level than other approaches, \ifnotpaper they also
              imply \else it also implies \fi that it is a test technique
          \citetext{pp.~10, 14}.
\end{enumerate}