\section{Conclusion}\label{conclusion}

While a good starting point, the software testing literature contains many
flaws that create unnecessary barriers to testing software. While there is
merit in allowing the state-of-the-practice terminology to guide software
testing communication descriptively, standardizing these conventions gives
common ground for software testers to use or adapt for their needs. This
\docType{} exposes how unstandardized software testing terminology is by
outlining the flaws we find in the literature.

\input{build/methodOverviewConc} In addition to these manually recorded flaws,
we identify classes of flaws we can detect automatically as described in
\Cref{dmn-def}. These include:
\begin{itemize}
    \item \multiCatCount{} test approaches with more than one category,
    \item \multiSynCount{} terms used as synonyms to two (or more) disjoint
          test approaches,
    \item \selfParCount{} test approaches that are parents of themselves, and
    \item \parSynCount{} pairs of test approaches that may either be synonyms
          or have a parent-child relationship.
\end{itemize}
In total, we find \flawCount{} flaws in the software testing literature, of
which \contraSummary*{} Additionally, \catsSummary*{}

Our work provides a solid baseline for what more thorough standardization of
software testing literature could look like: a centralized location for
recording data along with tools for analyzing it that can be updated by the
community as standards change or as new test approaches are developed. However,
there is more work to be done in this area, such as addressing the threats to
validity from \Cref{threats}.%