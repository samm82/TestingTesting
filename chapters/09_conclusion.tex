\section{Conclusion}\label{conclusion}

While a good starting point, the software testing literature contains many
flaws that create unnecessary barriers to testing software;
% While there is merit in allowing the state-of-the-practice terminology to
% guide software testing communication descriptively,
standardized conventions give software testers common ground to use or
adapt for their needs. This \docType{} exposes how unstandardized software
testing terminology is by outlining the flaws we find in the literature.

\input{build/methodOverviewConc} In addition to these manually recorded flaws,
we identify classes of flaws we can detect automatically in \Cref{dmn-def}.
These include:
\begin{itemize}
    \item \multiCatCount{} test approaches with more than one category,
    \item \multiSynCount{} terms used as synonyms to two (or more) disjoint
          test approaches,
    \item \selfParCount{} test approaches that are parents of themselves, and
    \item \parSynCount{} pairs of test approaches that may either be synonyms
          or have a parent-child relationship.
\end{itemize}
We also identify synonyms between independently defined approaches
(see \Cref{fig:expSynGraph} for a visualization of those that are explicit)
but only count these as flaws when appropriate, since this type of synonym
relation may just mean that the terms are interchangeable and the relation
is trivially reflexive. In total, we found \flawCount{} flaws in the software
testing literature, from which we observe the following:
\begin{enumerate}
    \item \contraSummary{}
    \item \catsSummary{}
    \item \semFlaws{}
\end{enumerate}

Our work provides a solid baseline for addressing flaws such as these in the
software testing literature: a centralized location for recording data that
the software community can update as standards change or as new test approaches
emerge, along with tools for analyzing these data and checking for flaws.
However, there is more work to be done in this area, such as addressing the
threats to validity described in \Cref{threats} and continuing our research as
described in \Cref{future-work}.
