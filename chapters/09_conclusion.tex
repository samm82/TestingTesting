\section{Conclusion}\label{conclusion}

While a good starting point, the software testing literature contains many
flaws that create unnecessary barriers to testing software. While there is
merit in allowing the state-of-the-practice terminology to guide software
testing communication descriptively, standardizing these conventions gives
common ground for software testers to use or adapt for their needs. This
\docType{} exposes how unstandardized software testing terminology is by
outlining the flaws we find in the literature.

\input{build/methodOverviewConc} In addition to these manually recorded flaws,
we identify classes of flaws we can detect automatically as described in
\Cref{dmn-def}. These include:
\begin{itemize}
    \item \multiCatCount{} test approaches with more than one category,
    \item \multiSynCount{} terms used as synonyms to two (or more) disjoint
          test approaches,
    \item \selfParCount{} test approaches that are parents of themselves, and
    \item \parSynCount{} pairs of test approaches that may either be synonyms
          or have a parent-child relationship.
\end{itemize}
In total, we find \flawCount{} flaws in the software testing literature, of
which \contraSummary*{} Additionally, \catsSummary*{}

% Temporary, borrowed from paper
Future work in this
area will continue to investigate the current use of terminology, in
particular \nameref{undef-terms}, determine if IEEE's current
\nameref{cats-def} are sufficient, and rationalize the definitions of
and relations between terms.