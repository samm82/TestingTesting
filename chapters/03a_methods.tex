\section{Methods}
\label{methods}

\newcommand{\autoDiscreps}[1][,]{discrepancies in \Cref{syns,par-rels},
    including \refDiscrepsTable{}#1}

To better understand our findings, we define classifications of discrepancies,
including ``rigidity'' (\Cref{rigidity}) and build tools to visualize the
relations between test approaches more intuitively (\Cref{graph-gen}) and track
discrepancies surrounding them automatically (\Cref{discrep-analysis}).

\subsection{Rigidity}
\label{rigidity}

Since there is a considerable degree of nuance introduced by the use of natural
language, not all discrepancies are equal! To capture this nuance and provide a
more complete picture, we make a distinction between explicit and implicit
discrepancies, such as in \Cref{tab:discreps}. A piece of information is
``implicit'' if it is not directly given by a source but seems to be implied
or if it is sometimes, but not always, true. Discrepancies based on implicit
information are themselves implicit. These are automatically detected when
\hyperref[graph-gen]{generating graphs} and
\hyperref[discrep-analysis]{analyzing discrepancies} by
looking for keywords used in the glossaries that indicate that a piece of
infomation is implicit: ``implied'', ``inferred'', ``can be'', ``ideally'',
``usually'', ``most'', ``often'', ``if'', and ``although''\todo{Question marks?}
(see the \href
{https://github.com/samm82/TestGen-Thesis/blob/fdf031b732f0ded0b5d970069561e25b10755a5c/scripts/csvToGraph.py#L115-L116}
{relevant source code}).

\subsection{Approach Graph Generation}
\label{graph-gen}

\graphGenDesc{}

\subsection{Discrepancy Analysis}
\label{discrep-analysis}

In addition to analyzing specific discrepancies (or classes of discrepancies),
an overview of the amount, severity, and source of these discrepancies is also
useful. Subsets of this task can be automated (\Cref{auto-discrep-analysis}),
and even parts that need to be done manually (such as finding and categorizing
\nameref{other-discrep}) can be augmented with automated
tools (\Cref{aug-discrep-analysis}).

As the main portion of this analysis, discrepancies are categorized
based on the source categories involved (see \Cref{sources}) to identify how
many discrepancies a source category is responsible for. These
are then counted \emph{once} if this discrepancy exists in a source category at
least as ``trusted'' as the current source category under consideration, which
avoids counting the same discrepancy twice for a given category (see
\thesisissueref{83}). Not doing this would result in the number of
\emph{occurrences} of all discrepancies, instead of the number of discrepancies
\emph{themselves}, which is more useful.
As an example of this process, consider a discrepancy \emph{within} an IEEE
document (e.g., two different definitions are given for a term within the same
IEEE document) \emph{and} between the \acs{swebok} \emph{and} two papers%
\todo{Is this clear?}. This would add one to the following rows of
\refDiscrepsTable{} in the relevant column:

\begin{itemize}
    \item \textbf{\stdDiscBrkdwn{1}}, since this discrepancy occurs between
          sources within this category, even though these ``sources'' are the
          same document\footnote{A more nuanced breakdown that identifies
              discrepancies within a singular document is given in
              \Cref{fig:discrepSources}.},
    \item \textbf{\metaDiscBrkdwn{1}}, since this discrepancy occurs between a
          source in this category and a ``more trusted'' one
          (the IEEE standard), and
    \item \textbf{\otherDiscBrkdwn{1}}, since this discrepancy occurs between a
          source in this category and a ``more trusted'' one; even though there
          are two sources in this category \emph{and} two ``more trusted''
          categories involved, this discrepancy is only counted once to avoid
          any double counting.
\end{itemize}

\subsubsection{Automated Discrepancy Analysis}
\label{auto-discrep-analysis}

As outlined in \Cref{graph-gen}, some types of discrepancies can be detected
automatically. While just counting the total number of these types of
discrepancies is trivial, tracking the source(s) of these discrepancies is more
involved. Since the appropriate citations for each piece of information is
tracked (see \Cref{tab:exampleGlossary,tab:synExampleGlossary} for examples of
how these citations are formatted in the glossaries), they can be used to find
the offending source categories. This comes with the added benefit of being
available to format these citations to use \LaTeX{}'s citation commands for use
in the lists of \autoDiscreps[.]{}

Comparing the authors and years of each source related to a given discrepancy
can determine if it manifests within a single document and/or between documents
by the same author(s) when creating \Cref{fig:discrepSources}. Then, the
relevant sources can be sorted into their categories based on their citations,
done by the function in \Cref{lst:getSrcCat}, since each source category
outlined in \Cref{sources} is comprised of a small number of authors (with the
exception of \otherDiscBrkdwn{1}\todo{Update with \texttt{sources} branch!}).
This determines the appropriate row of \refDiscrepsTable{} and the appropriate
graph and slice in \Cref{fig:discrepSources}. In the latter case, lists of
sources can be distilled down to sets of categories which are compared against
each other to determine how times a given discrepancy manifests between source
categories. This process is described in more detail, with examples, in
\Cref{aug-discrep-analysis}.

Alongside this citation information are the keywords relevant for assessing a
piece of information's \nameref{rigidity}. This is useful when counting
discrepancies, since a discrepancy can be both explicit and implicit, but
should not be double counted as both (see \thesisissueref{83})! When counting
discrepancies in \Cref{tab:discreps}, a given discrepancy is counted only for
its most ``rigid'' manifest (i.e., it will only increment a value in the
``Implicit'' column if it is \emph{not} also explicit).

% TODO: how are these source categories determined by analyzing citation info?

\subsubsection{Augmented Discrepancy Analysis}
\label{aug-discrep-analysis}
While the \autoDiscreps{} could be deduced automatically from analyzing the
testing approach glossary, other types of discrepancies (namely
\nameref{other-discrep}) needed to be tracked manually.
