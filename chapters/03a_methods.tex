\section{Methods}
\label{methods}

\newcommand{\autoDiscreps}[1][,]{discrepancies in \Cref{syns,par-rels},
    including \refDiscrepsTable{}#1}

To better understand our findings, we define classifications of discrepancies,
including ``rigidity'' (\Cref{rigidity}) and build tools to visualize the
relations between test approaches more intuitively (\Cref{graph-gen}) and track
discrepancies surrounding them automatically (\Cref{discrep-analysis}).

\subsection{Rigidity}
\label{rigidity}

Since there is a considerable degree of nuance introduced by the use of natural
language, not all discrepancies are equal! To capture this nuance and provide a
more complete picture, we make a distinction between explicit and implicit
discrepancies, such as in \Cref{tab:discreps}. A piece of information is
``implicit'' if it is not directly given by a source but seems to be implied
or if it is sometimes, but not always, true. Discrepancies based on implicit
information are themselves implicit. These are automatically detected when
\hyperref[graph-gen]{generating graphs} and
\hyperref[discrep-analysis]{analyzing discrepancies} by looking for indicators
of uncertainty, which include question marks, ``~(Testing)'' (which indicates
that a test approach isn't explicitly denoted as such), and the keywords
``implied'', ``inferred'', ``can be'', ``ideally'', ``usually'', ``most'',
``likely'', ``often'', ``if'', and ``although''(see the \href
{https://github.com/samm82/TestGen-Thesis/blob/50380a3/scripts/csvToGraph.py#L124-L140}
{relevant source code}).

\subsection{Approach Graph Generation}
\label{graph-gen}

\graphGenDesc{}

\subsection{Discrepancy Analysis}
\label{discrep-analysis}

In addition to analyzing specific discrepancies (or classes of discrepancies),
an overview of the amount, severity, and source of these discrepancies is also
useful. Subsets of this task can be automated (\Cref{auto-discrep-analysis}),
and even parts that need to be done manually (such as finding and categorizing
\nameref{other-discrep}) can be augmented with automated
tools (\Cref{aug-discrep-analysis}).

As the main portion of this analysis, discrepancies are categorized
based on the source categories involved (see \Cref{sources}) to identify how
many discrepancies a source category is responsible for. These
are then counted \emph{once} if this discrepancy exists in a source category at
least as ``trusted'' as the current source category under consideration, which
avoids counting the same discrepancy twice for a given category (see
\thesisissueref{83}). Not doing this would result in the number of
\emph{occurrences} of all discrepancies, instead of the number of discrepancies
\emph{themselves}, which is more useful.

\phantomsection{}
\label{discrep-analysis-example}
As an example of this process, consider a discrepancy \emph{within} an IEEE
document (e.g., two different definitions are given for a term within the same
IEEE document) \emph{and} between another IEEE document, the \acs{istqb}
glossary \emph{and} two papers. This would add one to the
following rows of \refDiscrepsTable{} in the relevant column:

\begin{itemize}
    \item \textbf{\stdDiscBrkdwn{1}}: this discrepancy occurs:
          \begin{enumerate}
              \item within one standard and
              \item between two standards.
          \end{enumerate}
          This increments the count by just one to avoid double counting and
          would do so even if only one of the above conditions was true. A more
          nuanced breakdown of discrepancies that identifies those within a
          singular document and those between documents by the same author is
          given in \Cref{fig:discrepSources} and explained in more detail in
          \Cref{aug-discrep-analysis}.
    \item \textbf{\metaDiscBrkdwn{1}}: this discrepancy occurs between a
          source in this category and a ``more trusted'' one
          (the IEEE standards).
    \item \textbf{\otherDiscBrkdwn{1}}: this discrepancy occurs between a
          source in this category and a ``more trusted'' one. Even though there
          are two sources in this category \emph{and} two ``more trusted''
          categories involved, this increments the count by just one to avoid
          double counting.
\end{itemize}

\subsubsection{Automated Discrepancy Analysis}
\label{auto-discrep-analysis}

As outlined in \Cref{graph-gen}, some types of discrepancies can be detected
automatically. While just counting the total number of these types of
discrepancies is trivial, tracking the source(s) of these discrepancies is more
involved. Since the appropriate citations for each piece of information is
tracked (see \Cref{tab:exampleGlossary,tab:synExampleGlossary} for examples of
how these citations are formatted in the glossaries), they can be used to find
the offending source categories. This comes with the added benefit of being
available to format these citations to use \LaTeX{}'s citation commands for use
in the lists of \autoDiscreps[.]{}

Comparing the authors and years of each source related to a given discrepancy
can determine if it manifests within a single document and/or between documents
by the same author(s) when creating \Cref{fig:discrepSources}. Then, the
relevant sources can be sorted into their categories based on their citations,
done by the function in \Cref{lst:getSrcCat}, since each source category
outlined in \Cref{sources} is comprised of a small number of authors (with the
exception of \otherDiscBrkdwn{1}\todo{Update with \texttt{sources} branch!}).
This determines the appropriate row of \refDiscrepsTable{} and the appropriate
graph and slice in \Cref{fig:discrepSources}. In the latter case, lists of
sources can be distilled down to sets of categories which are compared against
each other to determine how times a given discrepancy manifests between source
categories. An example of this process is described in more detail in
\Cref{aug-discrep-analysis}.

Alongside this citation information are the keywords relevant for assessing a
piece of information's \nameref{rigidity}. This is useful when counting
discrepancies, since a discrepancy can be both explicit and implicit, but
should not be double counted as both (see \thesisissueref{83})! When counting
discrepancies in \Cref{tab:discreps}, a given discrepancy is counted only for
its most ``rigid'' manifest (i.e., it will only increment a value in the
``Implicit'' column if it is \emph{not} also explicit).

\subsubsection{Augmented Discrepancy Analysis}
\label{aug-discrep-analysis}
While the \autoDiscreps{} could be deduced automatically from analyzing the
testing approach glossary, other types of discrepancies (namely
\nameref{other-discrep}) needed to be tracked manually. This is done by adding
comments to the relevant \LaTeX{} files (generated or not) of the form
\begin{displayquote}
    \texttt{\% Discrep count: \{A1\} \{A2\} \dots{} | \{B1\} \{B2\} \dots{} |
        \{C1\} \{C2\} \dots}
\end{displayquote}
which can then be parsed to detemine where discrepancies occur. Each group of
sources is separated with a pipe symbol to be compared with the others, so any
number of groups are permitted. If only one group is present, it is compared
with itself. For example, the first line below means that source X has a
discrepancy with itself, while the second line adds a discrepancy between X and Y.
\begin{displayquote}
    \texttt{\% Discrep count: \{X\}\\\% Discrep count: \{X\} | \{X\} \{Y\}}
\end{displayquote}
Discrepancies between groups are not double counted; this means the following
line adds discrepancies between X and Z \emph{and} between Y and Z, without
counting the discrepancy between X and Z twice.
\begin{displayquote}
    \texttt{\% Discrep count: \{X\} | \{X\} \{Y\} | \{Z\}}
\end{displayquote}
Each source is given using its BibTeX key wrapped in curly braces to mimic
\LaTeX{}'s citation commands for ease of parsing, with the exception of the
\acs{istqb} glossary, due to its use of custom commands via
\texttt{\textbackslash citealias}. For example, the line
\begin{displayquote}
    \texttt{\% Discrep count: \{IEEE2022\} | \{IEEE2022\} \{IEEE2017\}\\
        $\,\hookrightarrow\,$\quad ISTQB \{Kam2008\} \{Bas2024\}}
\end{displayquote}
would be parsed as the example given in \Cref{discrep-analysis-example}. Since
the IEEE documents are written by the same standards organizations
(\begin{NoHyper}\citeauthor{IEEE2022}\end{NoHyper}), they are counted as a
discrepancy between documents by the same author(s) in \Cref{fig:discrepSources}.
Note that discrepancies within a single document, between documents by the same
author(s), and between documents from the same source category are not double
counted. The maximum number of discrepancies possible within a source category
in \Cref{fig:discrepSources} is three (once for each type), but only if there
is an example of each type where a ``stricter'' type does not apply.%
\todo{Does this make sense? Is this sentence necessary?}
