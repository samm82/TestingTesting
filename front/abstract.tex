\label{abstract}%
Despite the prevalence and importance of software testing, it lacks
a standardized and consistent taxonomy, instead relying on a large body of
literature with many flaws---even within individual documents! This hinders
precise communication, contributing to misunderstandings when planning and
performing testing. In this \docType{}, we %systematically
explore the current state of software testing terminology by:
\begin{enumerate}
    \item identifying established standards and prominent testing resources,
    \item capturing relevant testing terms from these sources, along with their
          definitions and relationships (both explicit and implicit), and
    \item constructing graphs to visualize and analyze these data.
\end{enumerate}
This process uncovers \approachCount{} test approaches and
\ifnotpaper four in-scope methods for deriving test approaches, such as those
    related to \fi \qualityCount{} software qualities\ifnotpaper\else\ that may
    imply additional related test approaches\fi. We also build
a tool for generating graphs that illustrate relations between test
approaches and track flaws captured by this tool and manually through
the research process. This reveals \flawCount{} flaws, including
\multiSynCount{} terms used as synonyms to two (or more) disjoint test approaches
and \parSynCount{} pairs of test approaches that may either be synonyms or have
a parent-child relationship. This also highlights notable confusion surrounding
functional, operational acceptance, recovery, and scalability testing. Our
findings make clear the urgent need for improved testing terminology so that
the discussion, analysis and implementation of various test approaches can be
more coherent. We provide some preliminary advice on how to achieve this
standardization.