\documentclass[22pt]{beamer}
\usepackage[orientation=portrait, size=custom, width=91.44, height=91.44,scale=1.2]{beamerposter} % 36in*2.5 = 90cm
\usepackage[absolute,overlay]{textpos}
\usepackage{bookmark} %pdflatex says to use this to avoid errors...
\usepackage{graphicx} %for including images
\graphicspath{{assets/images/}} %location of images
\usepackage{wrapfig} %wrap text around the images
\usepackage{listingsutf8}    %package for code environment; use this instead of verbatim to get automatic line break; use this instead of listings to get (•)
\usepackage{amsmath}
\usepackage{gensymb}
\usepackage[export]{adjustbox}
\usepackage[skins,theorems]{tcolorbox}
\usepackage{tikz}
\newcommand*\circled[1]{\tikz[baseline=(char.base)]{
            \node[shape=circle,draw,inner sep=2pt] (char) {#1};}}
\usepackage{array}
\usepackage{booktabs,adjustbox}
\usepackage[center]{caption}
\usepackage{pgfplots}
%plot options
\pgfplotsset{width=7cm,compat=1.8}
\PassOptionsToPackage{grey}{xcolor}
\usepackage{cite}
\usepackage{fancyvrb}

\usetikzlibrary{shapes,shapes.geometric,arrows,fit,calc,positioning,automata,}
\usepackage{pgf-pie}

\usepackage{wrapfig}

\input{assets/misc/posterHelpers}

% Acronyms
\usepackage{acro}
% Alphabetically sorted list of acronyms
\DeclareAcronym{aop}{short=AOP,long=Aspect-Oriented Programming}
\DeclareAcronym{api}{short=API,long=Application Programming Interface}
\DeclareAcronym{ast}{short=AST,long=Abstract Syntax Tree}
\DeclareAcronym{c-use}{short=c-use,long=Computation data Use,alt=C-use}
\DeclareAcronym{cgi}{short=CGI,long=Common Gateway Interface}
\DeclareAcronym{cli}{short=CLI,long=Command-Line Interface}
\DeclareAcronym{cms}{short=CMS,long=Content Management System}
\DeclareAcronym{cpu}{short=CPU,long=Central Processing Unit}
\DeclareAcronym{cs}{short=CS,long=Constituent System}
\DeclareAcronym{csp}{short=CSP,long=Cross-Stage Persistence}
\DeclareAcronym{csv}{short=CSV,long=Comma-Separated Values}
\DeclareAcronym{ct}{short=CT,long=Continuous Testing}
\DeclareAcronym{ctmp}{short=CTMP,long=Compile-Time MetaProgramming}
\DeclareAcronym{dac}{short=DAC,long=Differential Assertion Checking}
\DeclareAcronym{dom}{short=DOM,long=Document Object Model}
\DeclareAcronym{dsl}{short=DSL,long=Domain-Specific Language}
\DeclareAcronym{du-path}{short=du-path,long=Definition-Use path,alt=DU-path}
\DeclareAcronym{emsec}{short=EMSEC,long=EManations SECurity}
\DeclareAcronym{ffi}{short=FFI,long=Foreign Function Interface}
\DeclareAcronym{fist}{short=FIST,long=Fault Injection Security Tool}
\DeclareAcronym{gool}{short=GOOL,long=Generic Object-Oriented Language}
\DeclareAcronym{gui}{short=GUI,long=Graphical User Interface}
\DeclareAcronym{html}{short=HTML,long=HyperText Markup Language}
\DeclareAcronym{href}{short=HREF,long=Hypertext REFerence}
\DeclareAcronym{ide}{short=IDE,long=Integrated Development Environment}
\DeclareAcronym{iec}{short=IEC,long=International Electrotechnical Commission}
\DeclareAcronym{istqb}{short=ISTQB,long=International Software Testing Qualifications Board}
\DeclareAcronym{json}{short=JSON,long=JavaScript Object Notation}
\DeclareAcronym{jvm}{short=JVM,long=Java Virtual Machine}
\DeclareAcronym{lcsaj}{short=LCSAJ,long=Linear Code Sequence and Jump}
\DeclareAcronym{mbt}{short=MBT,long=Model-Based Testing}
\DeclareAcronym{mdd}{short=MDD,long=Model-Driven Development}
\DeclareAcronym{ml}{short=ML,long=Machine Learning}
\DeclareAcronym{mop}{short=MOP,long=MetaObject Protocol}
\DeclareAcronym{mr}{short=MR,long=Metamorphic Relation}
\DeclareAcronym{msl}{short=MSL,long=MultiStage Language}
\DeclareAcronym{msp}{short=MSP,long=MultiStage Programming}
\DeclareAcronym{mt}{short=MT,long=Metamorphic Testing}
% From https://tex.stackexchange.com/a/583838/192195
\DeclareAcronym{operat}{short=OAT,long=Operational Acceptance Testing,list=Operational Acceptance/Orthogonal Array Testing}
\DeclareAcronym{orthat}{short=OAT,long=Orthogonal Array Testing,tag=dupe}
\DeclareAcronym{ot}{short=OT,long=Operational Testing}
\DeclareAcronym{nasa}{short=NASA,long=National Aeronautics and Space Administration}
\DeclareAcronym{p-use}{short=p-use,long=Predicate data Use,alt=P-use}
\DeclareAcronym{par}{short=PAR,long=Product Anomaly Report}
\DeclareAcronym{pdf}{short=PDF,long=Portable Document Format}
\DeclareAcronym{pir}{short=PIR,long=Product Incident Report}
\DeclareAcronym{pptmp}{short=PPTMP,long=PreProcessing-Time MetaProgramming}
\DeclareAcronym{qai}{short=QAI,long=Quality Assurance Institute}
\DeclareAcronym{rac}{short=RAC,long=Runtime Assertion Checking}
\DeclareAcronym{rq}{short=RQ,long=Research Question}
\DeclareAcronym{rtmp}{short=RTMP,long=RunTime MetaProgramming}
\DeclareAcronym{sos}{short=SoS,long=System of Systems,long-plural-form=Systems of Systems}
\DeclareAcronym{srs}{short=SRS,long=Software Requirements Specification}
\DeclareAcronym{sst}{short=SST,long=Skeleton Syntax Tree}
\DeclareAcronym{sut}{short=SUT,long=System Under Test}
\DeclareAcronym{sv}{short=SV,long=Software Verification}
\DeclareAcronym{swebok}{short=SWEBOK Guide,long=Guide to the SoftWare Engineering Body Of Knowledge,
    long-plural-form=Guides to the SoftWare Engineering Body Of Knowledge}
\DeclareAcronym{toat}{short=TOAT,long=Taguchi's Orthogonal Array Testing}
\DeclareAcronym{uml}{short=UML,long=Unified Modeling Language}
\DeclareAcronym{vnv}{short=V\&V,long=Verification and Validation}
\DeclareAcronym{wysiwyg}{short=WYSIWYG,long=What You See Is What You Get}

% Case Studies 
\DeclareAcronym{glassbr}{short=GlassBR,long=Glass BReaking}
% \DeclareAcronym{projectile}{short=Projectile,long=Projectile}
\DeclareAcronym{sglpend}{short=SglPend,long=Single Pendulum}
\DeclareAcronym{dblpend}{short=DblPend,long=Double Pendulum}
% \DeclareAcronym{gamephysics}{short=GamePhysics,long=Game Physics}
\DeclareAcronym{hghc}{short=HGHC,long=Heat Transfer Coefficients between Fuel and Cladding in Fuel Rods}
\DeclareAcronym{swhs}{short=SWHS,long=Solar Water Heating System}
\DeclareAcronym{swhsnopcm}{short=SWHSNoPCM,long=Solar Water Heating System with No Phase Change Material}
\DeclareAcronym{ssp}{short=SSP,long=Slope Stability analysis Program}


%------------------------------------------------------------------------------
%- Extra commands for more functionality -- in particular, capitalizing the
%- long form of acronyms.
%------------------------------------------------------------------------------

% Defining \ACL - to capitalize all words in an acronym
% Credits to: https://tex.stackexchange.com/a/257896
\NewDocumentCommand\ACF{sm}{%
    \begingroup
    \acsetup{uppercase/cmd=\ecapitalisewords}%
    \IfBooleanTF{#1}{\Acf*{#2}}{\Acf{#2}}%
    \endgroup
}

\NewDocumentCommand\ACFP{sm}{%
    \begingroup
    \acsetup{uppercase/cmd=\ecapitalisewords}%
    \IfBooleanTF{#1}{\Acfp*{#2}}{\Acfp{#2}}%
    \endgroup
}

\NewDocumentCommand\ACL{sm}{%
    \begingroup
    \acsetup{uppercase/cmd=\ecapitalisewords}%
    \IfBooleanTF{#1}{\Acl*{#2}}{\Acl{#2}}%
    \endgroup
}

\NewDocumentCommand\ACLP{sm}{%
    \begingroup
    \acsetup{uppercase/cmd=\ecapitalisewords}%
    \IfBooleanTF{#1}{\Aclp*{#2}}{\Aclp{#2}}%
    \endgroup
}


%\mode<presentation>
%this doesn't seem to make any difference; leave for now for trying out
\usetheme{Berlin}
\definecolor{MacBlue}{rgb}{0.10196,0.22353,0.53725}
\definecolor{MacMaroon}{rgb}{0.47843, 0, 0.23137}
\definecolor{MacMaroon2}{rgb}{0.47451, 0, 0}
\definecolor{MacGrey}{rgb}{0.54118,0.59216,0.62353}
\definecolor{MacMaroon3}{rgb}{00.47,0.2,0.31}
\definecolor{MacGold}{rgb}{1, 0.75,0.35}
\usecolortheme[named=MacMaroon2]{structure}

\definecolor{IEEEGrey}{rgb}{0.92549,0.92549,0.92941}
\setbeamertemplate{caption}[numbered]
\setbeamertemplate{navigation symbols}{}
\setbeamertemplate{footline}{}

\setbeamerfont{title}{size=\fontsize{60}{30}}
% From https://tex.stackexchange.com/questions/201216/beamer-nested-lists-do-not-decrease-the-font-size
\setbeamerfont{itemize/enumerate subbody}{size=\normalsize} %to set sub-itemize size

\setbeamercolor{block title example}{bg=MacGrey, fg=black}
\setbeamercolor{block body example}{bg=IEEEGrey, fg=black}

\title{A New Taxonomy of Software Testing Approaches}
\subtitle{Seeking More Standardized Standards}
  \author[Crawford, Carette, \& Smith]{Samuel Joseph Crawford, Jacques Carette,
    \& Spencer Smith\newline \small \{crawfs1, carette, smiths\}@mcmaster.ca}
  % institute not necessary, since this info is captured in the images
  % Date placed within \institute: HACK
    \institute{\fontsize{26}{12}\selectfont \today} % $^\dagger$
    \date{}
    % \institute[McMaster University]{Department of Computing and Software, McMaster University} % $^\dagger$
%   \date{\vspace{-5mm}April 22, 2023}

\begin{document}
%compile with pdflatex

%there is only one frame, because there is only one page; yeah, it's a poster
%textblock and block seem to work nicely to organize layout
\begin{frame}[fragile]

    \begin{textblock}{2}(0.7,1)
        \includegraphics[height=8.5cm]{eng_logo.png}
    \end{textblock}

    \begin{textblock}{2}(13,0.55)
        \includegraphics[height=12.5cm]{cas_logo.png}
    \end{textblock}

    \begin{textblock}{8}(4,1)
        \titlepage
    \end{textblock}

    \begin{textblock}{7.25}(0.5,2.75)
        \begin{block}{\fontsize{37}{20}\selectfont Goal}
            Taxonomy of software testing approaches
            \begin{itemize}
                \item Should be \textbf{systematic, rigorous, and ``complete''}
                \item Application: \textbf{automatically generating test cases}
                      in Drasil % \cite{Drasil}
                \item The \textbf{underlying domain} should drive the scope and
                      prerequisites for generated test cases
            \end{itemize}

            % The first step to any formal process is \textbf{understanding the
            %     underlying domain}. Therefore, a systematic and rigorous
            % understanding of software testing approaches is needed to develop formal
            % tools to test software. In our specific case, our motivation was seeing
            % \textbf{which kinds of testing can be generated automatically by Drasil},
            % ``a framework for generating all of the software artifacts for
            % (well understood) research software'' \cite{Drasil}.
            \vspace{5mm}
        \end{block}

        \begin{block}{\fontsize{37}{20}\selectfont Problem}
            Existing software testing taxonomies are inadequate
            \begin{itemize}
                \item Tebes et al.~(2020): focuses on parts of the
                      testing process (e.g., test goal, testable entity)
                \item Souza et al.~(2017): prioritizes organizing testing
                      approaches over defining them
                \item Unterkalmsteiner et al.~(2014): focuses on knowledge
                      transfer between development phases
            \end{itemize}
            \vspace{5mm}
        \end{block}

        \begin{block}{\fontsize{37}{20}\selectfont Methodology}
            Since a taxonomy doesn't already exist, we should create one!
            \begin{itemize}
                \item Start from \textbf{``standard'' resources}
                      (e.g., IEEE \cite{IEEE2022}, \cite{IEEE2021b},
                      \cite{IEEE2017}, \cite{IEEE2013};
                      the \acs{swebok} \cite{SWEBOK2024})
                \item \textbf{Collect} relevant information (over 500 testing
                      approaches and 70 software qualities, along with their
                      definitions) and \textbf{organize} it into spreadsheets
                \item \emph{Note: static testing approaches are included, since
                          they are sometimes included in ``software testing''
                          \cite[p.~17]{IEEE2022}, \cite[p.~440]{IEEE2017},
                          \cite[p.~5-2]{SWEBOK2024}}
                \item Iterate this process until there are
                      diminishing returns, implying that something approaching
                      a \textbf{complete taxonomy} has emerged!
                \item Since there are many standardized documents about
                      software testing (or software in general),
                      \textbf{this should be trivial, no?}
            \end{itemize}
            \vspace{5mm}
        \end{block}

        \begin{block}{\fontsize{37}{20}\selectfont In Our Experience\dots}
            \vspace{5mm}
            \begin{center}
                {\fontsize{185}{20}\selectfont \textbf{NO.}}
            \end{center}
            \vspace{4mm}
            \begin{columns}
                \begin{column}{0.7\textwidth}
                    \begin{center}
                        \begin{figure}
                            \centering
                            \includegraphics[width=\textwidth]{test approach choices.png}
                            \vspace{-7mm}
                            \caption{Classification of some ``test approach choices''
                                \cite[p.~22]{IEEE2022}.}
                            \label{Fig:approach-choices}
                        \end{figure}
                    \end{center}
                \end{column}
                \begin{column}{0.25\textwidth}
                    The classification of testing approaches in
                    Figure~\ref{Fig:approach-choices} \emph{appears} logical
                    but contains the following ambiguities:
                    \begin{itemize}
                        \item Experience-based testing is both a test design
                              technique \emph{and} a test practice
                        \item Pairs of terms are not distinguished:
                              \begin{itemize}
                                  \item Disaster/recovery testing and recovery
                                        testing
                                  \item Branch condition testing and branch
                                        condition combination testing
                                  \item Operational acceptance testing and
                                        operational testing \cite[p.~303]{IEEE2017}
                              \end{itemize}
                    \end{itemize}
                \end{column}
            \end{columns}
            \vspace{5mm}
        \end{block}
    \end{textblock}

    \begin{textblock}{7.25}(8.25,2.75)
        \begin{block}{\fontsize{37}{20}\selectfont More Examples}
            \cite{IEEE2022} and \cite{IEEE2021b} are software testing
            standards that leave much unstandardized (see Figure~\ref{Fig:IEEEdefs})
            \begin{itemize}
                \item About 20\% (23 out of 114) of testing approaches from
                      these standards \textbf{do not have a definition}!
                \item Five of these were (at the very least) described in the
                      previous version of this standard \cite{IEEE2013}
                \item Four were present in the same way in another IEEE
                      standard \cite{IEEE2017} before this one was published
            \end{itemize}
            \vspace{5mm}
            Having definitions does not mean they are useful; see
            Figure~\ref{Fig:unhelpful-defs} for some good (bad?) examples
            \vspace{-8mm}
            \begin{columns}
                \begin{column}{0.375\textwidth}
                    \begin{center}
                        \begin{figure}
                            \begin{tikzpicture}[thick, scale=1.7, every node/.style={align=left, scale=0.8}]
                                \pie{79.8/Defined,
                                12.3/{Not defined},
                                4.4/{Present in \cite{IEEE2013}},
                                3.5/{Present in \cite{IEEE2017}}}
                            \end{tikzpicture}
                            \caption{Breakdown of testing\\approach definitions in
                                \cite{IEEE2022} and \cite{IEEE2021b}.}
                            \label{Fig:IEEEdefs}
                        \end{figure}
                    \end{center}
                \end{column}
                \begin{column}{0.525\textwidth}
                    \begin{center}
                        \begin{figure}
                            \vspace{12mm}
                            \includegraphics[width=0.9\textwidth]{software element.png}

                            \vspace{2mm}

                            \includegraphics[height=2.7cm]{per.png}
                            \includegraphics[height=2.7cm]{state of.png}

                            \vspace{2mm}

                            \includegraphics[width=\textwidth]{device.png}

                            \vspace{2mm}

                            \caption{Less-than-helpful definitions\\
                            \cite[pp.~421, 170, 136, 301 (counterclockwise from top)]{IEEE2017}.
                            Note: ``equipment'' is not defined, and
                            ``mechanism'' is only defined as how ``a function
                            \dots\ transform[s] input into output'' [p.~270].}
                            \label{Fig:unhelpful-defs}
                        \end{figure}
                    \end{center}
                \end{column}
            \end{columns}

            \quad\\ % needed for line break

            \begin{center}
                \begin{minipage}{.9\textwidth}
                    \begin{exampleblock}{The \acs{swebok}'s Definition of
                            ``Scalability Testing''}
                        {\fontsize{28}{13}\selectfont
                            ``Scalability testing evaluates the capability to
                            use and learn the system and the user documentation.
                            It also focuses on the system's effectiveness in
                            supporting user tasks and the ability to recover
                            from user errors'' \cite[p.~5-9]{SWEBOK2024}}
                        % \vskip3mm
                        % \hspace*\fill{\small--- The \acs{swebok} V4 \cite[p.~5-9]{SWEBOK2024}}
                    \end{exampleblock}
                \end{minipage}
            \end{center}

            \begin{itemize}
                \item \swebokScalDef{}
                \item The \acs{swebok}'s definition of elasticity testing
                      \cite[p.~5-9]{SWEBOK2024} \swebokElasRef{}
            \end{itemize}

            \vspace{5mm}

            Alpha testing is quite common, but there is disagreement on who
            performs it:
            \begin{enumerate}
                \item ``users within the organization developing the software''
                      \cite[p.~17]{IEEE2017},
                \item ``a small, selected group of potential users''
                      \cite[p.~5-8]{SWEBOK2024}, or
                \item ``roles outside the development organization''
                      \cite{ISTQB_poster}
            \end{enumerate}
            \vspace{5mm}
        \end{block}

        \begin{block}{\fontsize{37}{20}\selectfont Conclusions \& Future Work}
            \begin{itemize}
                \item Current software testing taxonomies are \textbf{incomplete,
                          inconsistent, and/or incorrect}
                \item Ideally, one will be built systematically
                      from a large body of established sources
                \item We will continue investigating, analyzing, and structuring
                      how the literature defines
                      and categorizes software testing approaches
                \item This \textbf{broad and consistent taxonomy} will hopefully
                      grow as the field of testing advances
            \end{itemize}
            \vspace{5mm}
        \end{block}

        \begin{block}{\fontsize{37}{20}\selectfont References}
            \setbeamertemplate{bibliography item}{\insertbiblabel}
            \bibliographystyle{ieeetr}
            {\fontsize{26}{12}\selectfont
                \bibliography{references}}
            \vspace{5mm}
        \end{block}

        \begin{block}{\fontsize{37}{20}\selectfont Acknowledgments}
            We thank the Government of Ontario for OGS funding and Chris
            Schankula for this template.
            \vspace{5mm}
        \end{block}
    \end{textblock}

\end{frame}
\end{document}
