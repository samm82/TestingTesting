\documentclass[22pt]{beamer}
\usepackage[orientation=portrait, size=custom, width=91.44, height=91.44,scale=1.2]{beamerposter} % 36in*2.5 = 90cm
\usepackage[absolute,overlay]{textpos}
\usepackage{bookmark} %pdflatex says to use this to avoid errors...
\usepackage{graphicx} %for including images
\graphicspath{{assets/images/}} %location of images
\usepackage{wrapfig} %wrap text around the images
\usepackage{listingsutf8}    %package for code environment; use this instead of verbatim to get automatic line break; use this instead of listings to get (•)
\usepackage{amsmath}
\usepackage{gensymb}
\usepackage[export]{adjustbox}
\usepackage[skins,theorems]{tcolorbox}
\usepackage{tikz}
\newcommand*\circled[1]{\tikz[baseline=(char.base)]{
            \node[shape=circle,draw,inner sep=2pt] (char) {#1};}}
\usepackage{array}
\usepackage{booktabs,adjustbox}
\usepackage{subcaption}
\usepackage{pgfplots}
%plot options
\pgfplotsset{width=7cm,compat=1.8}
\PassOptionsToPackage{gray}{xcolor}
\usepackage{cite}
\usepackage{fancyvrb}

\usetikzlibrary{shapes,shapes.geometric,arrows,fit,calc,positioning,automata,}
\usepackage{pgf-pie}

\usepackage{wrapfig}

%\mode<presentation>
%this doesn't seem to make any difference; leave for now for trying out
\usetheme{Berlin}
\definecolor{MacBlue}{rgb}{0.10196,0.22353,0.53725}
\definecolor{MacMaroon} {rgb}{0.47843, 0, 0.23137}
\definecolor{MacMaroon2} {rgb}{0.47451, 0, 0}
\definecolor{MacGray}{rgb}{0.50196,0.49804,0.51765}
\definecolor{MacMaroon3}{rgb}{00.47,0.2,0.31}
\definecolor{MacGold}{rgb}{1, 0.75,0.35}
\usecolortheme[named=MacMaroon2]{structure}
\setbeamertemplate{caption}[numbered]
\setbeamertemplate{navigation symbols}{}

% From https://tex.stackexchange.com/questions/201216/beamer-nested-lists-do-not-decrease-the-font-size
\setbeamerfont{itemize/enumerate subbody}{size=\normalsize} %to set sub-itemize size

\title{A New Taxonomy of Software Testing Approaches}
\subtitle{Seeking More Standardized Standards}
  \author[Crawford, Carette, \& Smith]{Samuel Joseph Crawford, Jacques Carette, \& Spencer Smith\newline \small \{crawfs1, carette, smiths\}@mcmaster.ca}
  \institute[McMaster University]{Department of Computing and Software, McMaster University} % $^\dagger$
  \date{April 22, 2023}

\begin{document}
%compile with pdflatex

%there is only one frame, because there is only one page; yeah, it's a poster
%textblock and block seem to work nicely to organize layout
\begin{frame}[fragile]

    \begin{textblock}{2}(0.7,1)
        \includegraphics[height=8.5cm]{eng_logo.png}
    \end{textblock}

    \begin{textblock}{2}(13,0.55)
        \includegraphics[height=12.5cm]{cas_logo.png}
    \end{textblock}

    \begin{textblock}{8}(4,1)
        \titlepage
    \end{textblock}

    \begin{textblock}{7.25}(0.5,3.6)

        %this needs help
        \begin{block}{\fontsize{37}{20}\selectfont Goal}
            The first step to any formal process is \textbf{understanding the
                underlying domain}. Therefore, a systematic and rigorous
            understanding of software testing approaches is needed to develop formal
            tools to test software. In our specific case, our motivation was seeing
            \textbf{which kinds of testing can be generated automatically by Drasil},
            ``a framework for generating all of the software artifacts for
            (well understood) research software'' \cite{carette_drasil_2021}.
            \vspace{5mm}
        \end{block}

        \begin{block}{\fontsize{37}{20}\selectfont Problem}
            Most software testing ontologies seem to focus on the high-level
            testing process rather than the testing techniques themselves. For
            example:
            \begin{itemize}
                \item \cite{TebesEtAl2020a} mainly focuses on parts of the
                      testing process (e.g., test goal, testable entity)
                \item \cite{UnterkalmsteinerEtAl2014} provides a foundation for
                      classification but ``does not aim at providing a systematic
                      and exhaustive state-of-the-art survey of [either domain]''
                      (p.~A:2)
            \end{itemize}
            \vspace{5mm}
        \end{block}

        \begin{block}{\fontsize{37}{20}\selectfont Methodology}
            Since a taxonomy doesn't already exist, we should create one!
            \begin{itemize}
                \item We started with an ad hoc approach, focusing on
                      textbooks trusted at McMaster
                      %   \\\cite{Patton2006,
                      %       PetersAndPedrycz2000, vanVliet2000}
                \item We then realized that this was too arbitrary, so
                      we started from more established sources, such as
                      IEEE and SWEBOK
                      %   \\\cite{IEEE2022, SWEBOK2024, SWEBOK2014,
                      %       IEEE2017, ISO_IEC2023a, ISTQB}
                \item The goal of this approach is to iterate,
                      eventually revisiting the original textbooks,
                      until enough knowledge is built up to encounter
                      diminishing returns (ideally no returns!)
                \item Since there are many standardized documents about software
                      testing (or software in general), this should be trivial, no?
            \end{itemize}
            \vspace{5mm}
        \end{block}

        \begin{block}{\fontsize{37}{20}\selectfont In Our Experience}
            \vspace{5mm}
            \begin{center}
                {\fontsize{185}{20}\selectfont NO.}
            \end{center}
            \vspace{5mm}
        \end{block}
    \end{textblock}

    \begin{textblock}{7.25}(8.25,3.6)
        \begin{block}{\fontsize{37}{20}\selectfont Examples}
            \cite{IEEE2022} is a standard for general concepts related to
            software testing. However, it is not comprehensive. For example, as
            shown in Figure \ref{Fig:IEEEdefs}, most (55 out of 99) testing
            approaches mentioned in this standard do not have an accompanying
            definition! Eight of these were present in the previous version of
            this standard \cite{IEEE2013}, and nine were present in another
            IEEE standard \cite{IEEE2017} that would have been available
            upon publication of this one. However, the presence of a
            definition does not guarantee that it is useful! See Figure
            \ref{Fig:unhelpful-defs} for some examples.

            \begin{columns}
                \begin{column}{0.4\textwidth}
                    \begin{center}
                        \begin{figure}
                            \begin{tikzpicture}[thick, scale=1.5, every node/.style={align=left, scale=0.8}]
                                \pie{44.4/Defined,
                                38.4/Undefined,
                                9.1/{Present in\\another standard},
                                8.1/{Present in\\previous version}}
                            \end{tikzpicture}
                            \label{Fig:IEEEdefs}
                            \caption{Breakdown of testing approach definitions from \cite{IEEE2022}.}
                        \end{figure}
                    \end{center}
                \end{column}
                \begin{column}{0.4\textwidth}
                    \begin{center}
                        \begin{figure}
                            \includegraphics[height=4cm]{software element.png}

                            \vspace{2mm}

                            \includegraphics[height=3.7cm]{per.png}
                            \includegraphics[height=3.5cm]{state of.png}
                            \label{Fig:unhelpful-defs}
                            \caption{Some less-than-helpful definitions from \cite{IEEE2017}.}
                        \end{figure}
                    \end{center}
                \end{column}
            \end{columns}
        \end{block}

        \begin{block}{\fontsize{37}{20}\selectfont Conclusions \& Future Work}
            \begin{itemize}
                \item Current software testing taxonomies are incomplete,
                      inconsistent, and/or incorrect
                \item For one to be useful, it needs to be built systematically
                      from a large body of established sources
                \item We will continue investigating how the literature defines
                      and categorizes software testing approaches to analyze any
                      discrepancies and structure these ideas coherently
            \end{itemize}
        \end{block}

        \begin{block}{\fontsize{37}{20}\selectfont References}
            \setbeamertemplate{bibliography item}{\insertbiblabel}
            \bibliographystyle{ieeetr}
            {\fontsize{14}{15.75}\selectfont
                \bibliography{references}}
        \end{block}

        \begin{block}{\fontsize{37}{20}\selectfont Acknowledgments}
            We thank the Government of Ontario for OGS funding.
        \end{block}
    \end{textblock}

\end{frame}
\end{document}
